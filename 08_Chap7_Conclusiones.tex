\chapter{Conclusiones}
\noindent Como primer acercamiento a la fenomenología de los procesos estudiados en esta tesis, se realizaron una serie de simulaciones Monte Carlo que lograron reproducir los valores experimentales del factor de Fano. En particular, poner en evidencia las razones por las cuales este se aleja de la unidad, centradas principalmente en: que parte de la energía se disipa en fonones y que toda la energía de la radiación incidente se deposita en el material.

Por otro lado, se realizó un análisis cualitativo de las imágenes, pudiendo determinar características importantes en ellas que provocan efectos indeseados en las mediciones, como ser \textit{hot pixels} y \textit{hot columns} entre otras, lo cual permitió elegir los cuadrantes del sensor más aptos para el análisis.

Gracias a la aplicación de un corte que elimina píxeles con un solo electrón, se pudo aumentar casi tres veces la estadística de eventos disponible para el análisis. Además se desarrolló un método para la corrección del sesgo agregado por este corte y el sesgo existente en las imágenes debido a fondo.

Se obtuvo que el factor de Fano para los rayos $X$ del aluminio es $F = 0.1583 \pm 0.0083 $ con un error relativo del $5.2\,\%$ y para los rayos $X$ del flúor se obtuvo $F = 0.1694 \pm 0.0120$ con un error relativo del $7.1\,\%$, valores que no solapan entre sí con su error. Esto parecería ser un indicio de que el factor de Fano no es, en efecto, independiente de la energía de las fuentes, siendo mayor cuando menor es la energía.

Por otra parte, se obtuvo que la energía de creación electrón-hueco es $\varepsilon_{\eh}=(3.7501\pm0.0006)\,\si{eV}$ para el aluminio, mientras que para el flúor fue de $\varepsilon_{\eh}=(3.8222\pm0.0048)\,\si{eV}$.

Se logró determinar el ancho de la región de colección parcial de carga del sensor mediante la utilización del nuevo modelo de ajuste de los picos, que fue de $\tau_{\scaleto{CCE}{4pt}}=(0.0978\pm 0.0051)\,\si{\mu m}$ cuyo error solapa con el de $\tau_{\scaleto{CCE}{4pt}}=(0.0914\pm 0.0027)\,\si{\mu m} $ obtenido mediante otro método\cite{PCC-CCE}.

Finalmente, cabe destacar que los resultados obtenidos para el factor de Fano podrían indicar que existen razones fundamentales, aún inexploradas, que hacen que esta magnitud aumente para energías cada vez menores. A su vez, el modelo presentado para describir la PCC ofrece una forma cerrada para la densidad de probabilidad que describe la distribución de carga alrededor de los picos y habilita la caracterización de la zona de colección parcial de carga de los sensores utilizando espectros medidos. 
