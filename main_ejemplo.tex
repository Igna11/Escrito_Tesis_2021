\documentclass[a4paper,12pt]{report}
%Inicia preámbulo
\usepackage[utf8]{inputenc}
\usepackage[spanish]{babel}
%\usepackage[T1]{fontenc}
\usepackage{titlesec}
\usepackage{setspace}
\setlength{\parskip}{1em}
\usepackage{mathptmx}%tipografía 'Time' en texto y ecuaciones
\usepackage[a4paper, hmargin=2.54 cm, vmargin= 2.54 cm]{geometry}
\usepackage[draft]{graphicx}
%\usepackage[final]{graphicx}
% \usepackage{subfigure}
\usepackage{subcaption}
\usepackage{amsmath,amssymb}%escritura de ecuaciones
\usepackage{afterpage}
%\usepackage[labelfont=bf,centerlast,font=normalsize]{caption}
\usepackage{color}
\usepackage{enumerate}
%\usepackage{enumitem}
\usepackage{lscape}
\usepackage{multicol}
\usepackage{verbatim}
\usepackage{appendix}
\usepackage{array}
\newcommand{\dr}[1]{\textcolor{red}{[DR: #1]}}
\newcommand{\ka}[1]{\textcolor{blue}{[KA: #1]}}
\newcommand{\Fe}{$^{55}$Fe}
\renewcommand{\appendixname}{Apéndices}
\renewcommand{\appendixtocname}{Apéndices}
\renewcommand{\appendixpagename}{Apéndices}
\usepackage{hyperref}%hyperlinking; should be loaded at the end of the preamble

\addto\captionsspanish{\renewcommand*\contentsname{Indice}}
\addto\captionsspanish{\renewcommand{\tablename}{Tabla}}
\addto\captionsspanish{\renewcommand*{\bibname}{Referencias}}%cambia el nombre de la 'Bibliografía' por 'Referencias'
\titleformat{\chapter}[hang]{\bfseries\Large}{\thechapter.}{12pt}{\bfseries\Large}
\titleformat{\section}[hang]{\bfseries\large}{\thesection}{12pt}{\bfseries\large}
\graphicspath{ {Graficos/} }
\setlength{\parindent}{10pt}
%Finaliza preámbulo

\begin{document}

%%%%%%%% carátula %%%%%%%%%%%%%%%%%%%%%%%%%%%%%%%%%%%%%%%
\begin{titlepage}
    \begin{center}
    \doublespacing

    \vspace*{150pt}
    \textbf{Calibración absoluta de sensores CCD utilizando la tecnología Skipper}\\

    \vspace*{20pt}
    Kevin Daniel Andersson

    \vspace{230pt}
    Tesis de Licenciatura en Ciencias Físicas\\Facultad de Ciencias Exactas y Naturales\\Universidad de Buenos Aires\\\vspace{30pt}Marzo 2021
    \end{center}

%------PÁGINA VACIA ------------------
\newpage
\thispagestyle{empty} \mbox{}
\thispagestyle{empty}
%-------------------------------
\end{titlepage}


%%%%%%%%%%%%  segunda pagina %%%%%%%%%%%%%%%%%%%%%%%%%%%%%%%%%%
\thispagestyle{empty}
TEMA: Física de partículas - Detectores Skipper CCD.

ALUMNO: L.U. N$^\circ\ 218/15$

LUGAR DE TRABAJO: Silicon Detector Facility (SiDet) - Fermi National Accelerator Laboratory

DIRECTOR: Dr. Darío P. Rodrigues F. Maltéz

CODIRECTOR: Juan Cruz Estrada Vigil

FECHA DE INICIACIÓN: Septiembre de 2019

FECHA DE FINALIZACIÓN:  2020 

FECHA DE EXAMEN:  11 de Marzo de 2021

INFORME FINAL APROBADO POR:
\vfill
\begin{multicols}{2}    
\rule[0pt]{2.5in}{0.5pt}\\
Autor: Kevin D. Andersson
\vspace{3em}

\rule[0pt]{2.5in}{0.5pt}\\
Director: Dr. Dario Rodrigues
\vspace{3em}
   
\rule[0pt]{2.5in}{0.5pt}\\
Profesora de Tesis de Licenciatura
    
\columnbreak
\rule[0pt]{2.5in}{0.5pt}\\
Jurado 
\vspace{3em}
    
\rule[0pt]{2.5in}{0.5pt}\\
Jurado 
\vspace{3em}
    
\rule[0pt]{2.5in}{0.5pt}\\
Jurado 
\end{multicols}  
%------PÁGINA VACIA ------------------
\newpage
\thispagestyle{empty} \mbox{}
\thispagestyle{empty}
%%%%%%% fin segunda hoja %%%%%%%%%%%%%%%%%%%%%%%%%%%%%%%%%%%%%

%%%%%%%%%%%%%%%% dedicatoria %%%%%%%%%%%%%%%%%%%%%%%%%%%%%%%%
\newpage
\thispagestyle{empty}
 \vspace*{150pt}
\begin{flushright}
\textit{A mis viejos}
\end{flushright}
%------PÁGINA VACIA ------------------
\newpage
\thispagestyle{empty} \mbox{}
\thispagestyle{empty}
%%%% fin dedicatoria %%%%%%%%%%%%%%%%%%%%%%%%%%%%%%%%%%%%%%%%%


%%%% resumen %%%%%%%%%%%%%%%%%%%%%%%%%%%%%%%%%%%%%%%%%%%%%%%%%
\newpage
\thispagestyle{empty}
\begin{center}\large{\textbf{Resumen}}\end{center}
\spacing{1.30}
\chapter*{Resumen}
\noindent ACÁ VA EL RESUMEN
\begin{equation}
    \sqrt{12382434198312389123012939123091}
\end{equation}

%------PÁGINA VACIA ------------------
\newpage
\thispagestyle{empty} \mbox{}
\thispagestyle{empty}
%-------------------------------
%%%%%%%%%%%%%%%%%%%%%%%%%%%%%%%%%%%%%%%%%%%%%%%%%%%%%%%%%%%%%%%%%%%%%%%%%%%%%%%%%

%%%%% indice %%%%%%%%%%%%%%%%%%%%%%%%%%%%%%%%%%%%%%%%%%%%%%%%%%%%%%%%%%%%%%%%%%%%
\newpage
\pagenumbering{roman} %numero de pagina en números romanos
\setcounter{page}{1}  %comienza a contar desde 1
\tableofcontents % Indice
%%% La siguiente página vacía se agregó para el
% el siguiente capítulo comience en hoja impar.
%------PÁGINA VACIA ------------------
\newpage
\thispagestyle{empty} \mbox{}
\thispagestyle{empty}
%-------------------------------

%%%%%%%%%%%%%%%%%%%%%%%%%%%%%%%%%%%%%%%%%%%%%%%%%%%%%%%%%%%%%%%%%%%%%%%%%%%%%%%%%
\newpage
\pagenumbering{arabic}  %numero de pagina numeración arábiga
\setcounter{page}{1}    %comienza a contar desde 1
%------------------------------------------------------------------------------
\singlespace
\chapter{Introducción}
\spacing{1.30}
\input{CapIntroduccion.tex}

%%%%%%%%%%%%%%%%%%%%%%%%%%%%%%%%%%%%%%%%%%%%%%%%%%%%%%%%%%%%%%%%%%%%%%%%%%%%%%%%%%%%%%%%%%%%%%%%%%%%%%%%%%%%%%%%%%%%%%%%%%%%%%%%%%%%%%%%%%%%%%%%%%%%%%%%%%%%%%%%%%%%%%%%%%%%%%%%%%%%%%%%%%%%%%%%%%%%%%%%%%%%%%%%%%%%%%%%%
\singlespace
\chapter{Detectores Skipper-CCD}
\label{cap:detectores}
\spacing{1.30}
\input{CapDetectores.tex}
%%%%%%%%%%%%%%%%%%%%%%%%%%%%%%%%%%%%%%%%%%%%%%%%%%%%%%%%%%%%%%%%%%%%%%%%%%%%%%%%%%%%%%%%%%%%%%%%%%%%%%%%%%%%%%%%%%%%%%%%%%%%%%%%%%%%%%%%%%%%%%%%%%%%%%%%%%%%%%%%%%%%%%%%%%%%%%%%%%%%%%%%%%%%%%%%%%%%%%%%%%%%%%%%
%%% La siguiente página vacía se agregó para el
% el siguiente capítulo comience en hoja impar.
%------PÁGINA VACIA ------------------
% \newpage
% \thispagestyle{empty} \mbox{}
% \thispagestyle{empty}
% %-------------------------------

%%%%%%%%%%%%%%%%%%%%%%%%%%%%%%%%%%%%%%%%%%%%%%%%%%%%%%%%%%%%%%%%%%%%%%%%%%%%%%%%%%%%%%%%%%%%%%%%%%%%%%%%%%%%%%%%%%%%%%%%%%%%%%%%%%%%%%%%%%%%%%%%%%%%%%%%%%%%%%%%%%%%%%%%%%%%%%%%%%%%%%%%%%%%%%%%%%%%%%%%%%%%%%%%%%%%%%%%%%%%%%%%%
\singlespace
\chapter{Calibración absoluta de la relación ADU-electrón} \label{cap: calibracion}
\spacing{1.3}
\input{CapCalibracion.tex}
%%%%%%%%%%%%%%%%%%%%%%%%%%%%%%%%%%%%%%%%%%%%%%%%%%%%%%%%%%%%%%%%%%%%%%%%%%%%%%%%%%%%%%%%%%%%%%%%%%%%%%%%%%%%%%%%%%%%%%%%%%%%%%%%%%%%%%%%%%%%%%%%%%%%%%%%%%%%%%%%%%%%%%%%%%%%%%%%%%%%%%%%%%%%%%%%%%%%%%%%%%%%%%%%%%%%%%%%%%%%%%
\singlespace
\chapter{Determinación absoluta del factor de Fano y la energía de creación e-h}
\label{cap:fano}
\spacing{1.3}
\input{CapFano.tex}
%%%%%%%%%%%%%%%%%%%%%%%%%%%%%%%%%%%%%%%%%%%%%%%%%%%%%%%%%%%%%%%%%%%%%%%%%%%%%%%%%%%%%%%%%%%%%%%%%%%%%%%%%%%%%%%%%%%%%%%%%%%%%%%%%%%%%%%%%%%%%%%%%%%%%%%%%%%%%%%%%%%%%%%%%%%%%%%%%%%%%%%%%%%%%%%%%%%%%
%%% La siguiente página vacía se agregó para el
% el siguiente capítulo comience en hoja impar.
%------PÁGINA VACIA ------------------
%\newpage
%\thispagestyle{empty} \mbox{}
%\thispagestyle{empty}
%-------------------------------
\singlespace
\chapter{Factor de Fano en energías por debajo de 2 keV} \label{cap:bajaenergia}
\spacing{1.3}
\input{CapBajaEnergia.tex}

%%%%%%%%%%%%%%%%%%%%%%%%%%%%%%%%%%%%%%%%%%%%%%%%%%%%%%%%%%%%%%%%%%%%%%%%%%%%%%%%%%%%%%%%%%%%%%%%%%%%%%%%%%%%%%%%%%%%%%%%%%%%%%%%%%%%%%%%%%%%%%%%%%%%%%%%%%%%%%%%%%%%%%%%%%%%%%%%%%%%%%%%%%
%%% La siguiente página vacía se agregó para el
% el siguiente capítulo comience en hoja impar.
%------PÁGINA VACIA ------------------
% \newpage
% \thispagestyle{empty} \mbox{}
% \thispagestyle{empty}
%-------------------------------
\chapter{Conclusiones}\label{conclusiones}
\spacing{1.3}
\input{Conclusiones.tex}
%%%%%%%%%%%%%%%%%%%%%%%%%%%%%%%%%%%%%%%%%%%%%%%%%%%%%%%%%%%%%%%%%%%%%%%%%%%%%%%%%%%%%%%%%%%%%%%%%%%%%%%%%%%%%%%%%%%%%%%%%%%%%%%%%%%%%%%%%%%%%%%%%%%%%%%%%%%%%%%%%%%%%%%%%%%%%%%%%%%%%%%%%%
\begin{thebibliography}{99}
\addcontentsline{toc}{chapter}{Referencias}%\addcontentsline:agrega una entrada a la lista especificada{extensión del archivo en el que se escribirá información: toc =table of contents}{nombre de la sección}{entrada}

%Introducción:
\bibitem{connie}
A. Aguilar-Arevalo, X. Bertou, C. Bonifazi, G. Cancelo, A. Castañeda, B. Cervantes Vergara, C. Chavez, J. C. D’Olivo, J. C. dos Anjos, J. Estrada, and et al., \textit{Exploring low-energy neutrino physics with the coherent neutrino nucleus interaction experiment}, Physical Review D \textbf{100} (2019), 10.1103/physrevd.100.092005.
\bibitem{sensei}
O. Abramoff, L. Barak, I. M. Bloch, L. Chaplinsky, M. Crisler, Dawa, A. Drlica-Wagner, R. Essig, J. Estrada, E. Etzion, and et al., \textit{Sensei: Direct-detection constraints on sub-GeV dark matter from a shallow underground run using a prototype skipper ccd}, Physical Review Letters \textbf{122} (2019), 10.1103/physrevlett.122.161801.
\bibitem{DESI}
C.J. Bebek, J.H. Emes, D.E. Groom, S. Haque, S.E. Holland, A. Karcher et al., \textit{CCD development for the Dark Energy Spectroscopic Instrument}, 2015 JINST \textbf{10} C05026.
\bibitem{DECam}
DES collaboration, B. Flaugher et al., \textit{The Dark Energy Camera}, Astron. J. \textbf{150} (2015) 150
\bibitem{e-h_1}
R. D. Ryan, \textit{Precision measurements of the ionization energy and its temperature variation in high purity silicon radiation detectors}, IEEE Transactions on Nuclear Science \textbf{20}, 473–480 (1973).
\bibitem{e-h_2}
P. Lechner, R. Hartmann, H. Soltau, and L. Strüder, \textit{Pair creation energy and fano factor of silicon in the energy range of soft x-rays}, Nuclear Instruments and Methods in Physics Research Section A: Accelerators, Spectrometers, Detectors and Associated Equipment \textbf{377}, 206 – 208 (1996), proceedings of the Seventh European Symposium on Semiconductor.
\bibitem{e-h_3}
F. Scholze, H. Rabus, and G. Ulm, \textit{Mean energy required to produce an electron-hole pair in silicon for photons of energies between 50 and 1500 ev}, Journal of Applied Physics \textbf{84}, 2926–2939 (1998),
https://doi.org/10.1063/1.368398.
\bibitem{skipper}
James R. Janesick. \textit{Scientific Charge-Couple Devices}. Press Monographs. Society of Photo Optical, 2001.
\bibitem{Fano_original}
U. Fano, \textit{Ionization yield of radiations. ii. the fluctuations of the number of ions}, Phys. Rev. \textbf{72}, 26–29 (1947).
%\bibitem{Si_band_gap}
%http://hyperphysics.phy-astr.gsu.edu
\bibitem{sensei2020}
L. Barak, I. M. Bloch, M. Cababie, G. Cancelo, L. Chaplinsky, F. Chierchie, M. Crisler, A. Drlica-Wagner, R. Essig, J. Estrada, E. Etzion, G. Fernandez Moroni, D. Gift, S. Munagavalasa, A. Orly, D. Rodrigues, A. Singal, M. Sofo Haro, L. Stefanazzi, J. Tiffenberg, S. Uemura, T. Volansky, T. T. Yu, \textit{Direct-Detection Results on sub-GeV Dark Matter from a New Skipper-CCD} arXiv:2004.11378 (2020).

\bibitem{moroni_2012}
Moroni, G. F., Estrada, J., Cancelo, G., Holland, S. E., Paolini, E. E., Diehl, H. T. \textit{Subelectron readout noise in a skipper ccd fabricated on high resistivity silicon}. Experimental Astronomy, \textbf{34} (1), 43–64, 2012.
%Detectores:
\bibitem{MOS}
Chenming Calvin Hu, \textit{Modern Semiconductor Devices for Integrated Circuits}, Ch. 5 2010
\bibitem{correlated-double-sampling}
White, M. H., Lampe, D. R., Blaha, F. C., Mack, I. A. \textit{Characterization of surface channel ccd image arrays at low light levels}. Solid-State Circuits, IEEE Journal of, \textbf{9} (1), 1–12, 1974.
\bibitem{fully-depleted-CCD}
Holland, S., Bebek, C., Kolbe, W., Lee, J. \textit{Physics of fully depleted ccds}. Journal of Instrumentation, \textbf{9} (03), C03057, 2014

\bibitem{skipper_rn}
J. Tiffenberg, M. Sofo-Haro, A. Drlica-Wagner, R. Essig, Y. Guardincerri, S. Holland, T. Volansky, and T.-T. Yu, \textit{Single-Electron and Single-Photon Sensitivity with a Silicon Skipper CCD}, Phys. Rev. Lett. \textbf{119}, 131802 (2017).

%Calibracion:
\bibitem{tesis_andre}
Tesis de Licenciatura de André Donadon Servelle. \textit{Determinación del factor de Fano en Si utilizando un detector skipper CCD}. (2019). 
\bibitem{linealidad_convencional}
G. M. Bernstein, T. M. C. Abbott, S. Desai, D. Gruen, R. A. Gruendl, M. D. Johnson, H. Lin, F. Menanteau, E. Morganson, E. Neilsen, K. Paech, A. R. Walker, W. Wester, and B. Y. and, \textit{Instrumental response model and detrending for the dark energy camera}, Publications of the Astronomical Society of the Pacific \textbf{129}, 114502 (2017).

%Fano:
\bibitem{Fe_decaimiento}
http://www.nucleide.org/
\bibitem{Fe_table}
J. W. Fowler, B. K. Alpert, D. A. Bennett, W. B. Doriese, J. D. Gard,
G. C. Hilton, L. T. Hudson, Y.-I. Joe, K. M. Morgan, G. C. O’Neil, C. D.Reintsema, D. R. Schmidt, D. S. Swetz, C. I. Szabo, and J. N. Ullom, \textit{A reassessment of absolute energies of the x-ray l lines of lanthanide metals}, Metrologia \textbf{54}, 494–511 (2017).
\bibitem{difusion}
Guillermo Fernandez Moroni, Miguel Sofo Haro, Javier Tiffenberg, Gustavo Cancelo, Eduardo E. Paolini, Juan Estrada, Xavier Bertou, \textit{Mathematical model of point events in CCD images}. 2015 XVI Workshop on Information Processing and Control (publicado el 23 de Junio del 2016 en IEEE Xplore)
\bibitem{difusion_eq}
S. Holland, D. Groom, N. Palaio, R. Stover, and M. Wei, IEEE Trans. Electron Devices \textbf{50}, 225 (2003).
\bibitem{antecedentes_25}
B. Lowe and R. Sareen, \textit{A measurement of the electron–hole pair creation energy and the fano factor in silicon for 5.9kev x-rays and their temperature dependence in the range 80–270k}, Nuclear Instruments and Methods in Physics Research Section A: Accelerators, Spectrometers, Detectors and Associated Equipment \textbf{576}, 367 – 370 (2007).
\bibitem{antecedentes_26}
K. J. McCarthy, A. Owens, A. Holland,  A. A. Wells, \textit{Modelling the X-ray response of charge coupled devices
Author links open overlay panel}, Nuclear Instruments and Methods in Physics Research Section A: Accelerators, Spectrometers, Detectors and Associated Equipment \textbf{362}, 538 (1995).
\bibitem{antecedentes_27}
A. Owens, G. Fraser, and K. J. McCarthy, \textit{On the experimental determination of the Fano factor in Si at soft X-ray wavelengths}, Nuclear Instruments and Methods in Physics Research Section A: Accelerators, Spectrometers, Detectors and Associated Equipment \textbf{491}, 437 (2002).
\bibitem{antecedentes_28}
K. Ramanathan et al., \textit{Measurement of low energy ionization signals from Compton scattering in a charge-coupled device dark matter detector}, Phys. Rev. D \textbf{96}, 042002 (2017)
\bibitem{paper_fano}
D. Rodrigues, K. Andersson, M. Cababie, A. Donadon, G. Cancelo, J. Estrada, G. Fernandez-Moroni, R. Piegaia, M. Senger, M. Sofo Haro, L. Stefanazzi, J. Tiffenberg, S. Uemura, \textit{Absolute measurement of the Fano factor using a Skipper-CCD}, arXiv:2004.1149 (2020). Actualmente en revisión por la revista Nuclear Instrument and Method A.
\bibitem{PCC}
Guillermo Fernandez-Moroni, Kevin Andersson, Ana Botti, Juan Estrada, Dario Rodrigues, and Javier Tiffenberg. \textit{Charge collection efficiency in back-illuminated charge-coupled devices}, arXiv:2007.04201, (2020).
\bibitem{Rodrigues2021}
Rodrigues et al. \textit{On the pdf charge distribution in X-Ray measurements using CCD}. Comunicación interna.

%Bajas Energías
\bibitem{Kotov_2018}
Kotov, Ivan V., Neal, H., and O'Connor, P. Thu. \textit{Pair creation energy and Fano factor of silicon measured at 185 K using 55Fe X-rays}. United States. doi:10.1016/j.nima.2018.06.022. %https://www.osti.gov/servlets/purl/1456908.
\bibitem{kapton}
https://physics.nist.gov/



\end{thebibliography}
%%%%%%%%%%%%%%%%%%%%%%%%%%%%%%%%%%%%%%%%%%%%%%%%%%%%%%%%%%%%%%%%%%%%%%%%%%%%%%%%%%%%%%%%%%%%%%%%%%%%%%%%%%%%%%%%%%%%%%%%%%%%%%%%%%%%%%%%%%%%%%%%%%%%%%%%%%%%%%%%%%%%%%%%%%%%%%%%%%%%%%%%%%%%%%%%%%%%%%%%%%%%%%%%%

\singlespace
\chapter*{Agradecimientos}
\addcontentsline{toc}{chapter}{Agradecimientos}
\spacing{1.00}
\singlespace
\chapter*{Agradecimientos}
\addcontentsline{toc}{chapter}{Agradecimientos}
%\spacing{1.00}
\noindent La carrera la arranqué allá por el 2013, así que ya pasaron casi 10 años. Hoy, eso equivale más o menos a un tercio de mi vida.
Es imposible que en todo este tiempo no haya conocido gente maravillosa a la que quisiera agradecer. 

En primer lugar, a Paui por ser mi compañera eterna de labo y de vida, que si no fuera por ella es muy probable que no hubiera llegado hasta acá, siempre apoyándome y ayudándome en todo lo que necesito.

A los amigos que hice en el camino, Andi, Tomi, Fede, con los que tuve la suerte de compartir tanto estudio como salidas, al igual que con Anita, Lufa, Lukp. También Agus, Sol y Mai, Tomi Noten, Pili y Manu.

A todo el grupo hermoso de gente que conocí gracias a labo 6 y 7 en el LEC, grupo realmente inigualable, Tinchooooooo, Manu, Agus, Fabri, Juan, Les, Hilario, Nacho, Nati, Nico y Lau. Realmente dejaron la vara altísima.

Injusto sería no agradecer a la gente que me dio todo para que yo pueda estar donde estoy hoy, que son mis viejos y que siempre me bancaron con todo. A mi hermano que, a pesar de ser ingeniero, la tiene muy clara en muchas cosas y siempre lo usé y lo uso de material de consulta.

A mis amigos de toda la vida, que se bancaron infinidad de veces mis \textit{no puedo, tengo que estudiar}, en especial a Marian que es el que siempre está ahí presente.

Al lector, gracias por haber llegado hasta acá.

Para ir terminando, no puedo dejar de agradecer a Darío, quien tuvo una paciencia infinita en todos todos los aspectos posibles a lo largo del desarrollo de esta tesis.

Finalmente, quiero agradecer por formar parte de este hermoso mundillo que es Exactas. Mentiría si dijera que no lo sufrí ni un poco, pero no miento al decir que no me arrepiento.
%%%%%%%%%%%%%%%%%%%%%%%%%%%%%%%%%%%%%%%%%%%%%%%%%%%%%%%%%%%%%%%%%%%%%%%%%%%%%%%%%%%%%%%%%%%%%%%%%%%%%%%%%%%%%%%%%%%%%%%%%%%%%%%%%%%%%%%%%%%%%%%%%%%%%%%%%%%%%%%%%%%%%%%%%%%%%%%%%%%%%%%%%%%%%%%%%%%%%%%%%%%%%%%%%

\end{document}