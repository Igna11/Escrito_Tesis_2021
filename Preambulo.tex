\documentclass[12pt]{book}
\usepackage[utf8]{inputenc}
\usepackage[spanish, es-tabla]{babel}
\usepackage{anysize}
\usepackage{amsmath}
\usepackage{amssymb}
\usepackage{physics}

\usepackage{multirow}
\usepackage{graphicx}
\usepackage{float}

\usepackage{subcaption}
\usepackage{fancybox}
\usepackage{titlesec}
\usepackage{letltxmacro}
\usepackage{xcolor} % 14/12/2021 texto en color
% (raiz) Para hacer la raiz cuadrada con el cierre al final
\makeatletter
\let\oldr@@t\r@@t
\def\r@@t#1#2{%
\setbox0=\hbox{$\oldr@@t#1{#2\,}$}\dimen0=\ht0
\advance\dimen0-0.2\ht0
\setbox2=\hbox{\vrule height\ht0 depth -\dimen0}%
{\box0\lower0.4pt\box2}}
\LetLtxMacro{\oldsqrt}{\sqrt}
\renewcommand*{\sqrt}[2][\ ]{\oldsqrt[#1]{#2}}
\makeatother
% (raiz) nacho: 31/03/2018 17:35
\usepackage{xargs}
\usepackage[hidelinks]{hyperref}
% (citas)
\usepackage[superscript,biblabel,nomove]{cite}
% (citas) hace las citas superíndices
\usepackage{mathtools} % 02/02/2022 para tener el prescript{}{} índice y supraíndice del lado izq a misma altura
\usepackage{mathrsfs} %03/02/2022 Para la L del lagrangiano bien cheta 14/10/2021

\makeatletter 
\renewcommand{\@citess}[1]{\textsuperscript{\,[#1]}}

\usepackage{titlesec}% http://ctan.org/pkg/titlesec
\usepackage{fancyhdr}

\usepackage{siunitx} %03/01/2022 <- Para unidades 

\newcommand{\eh}{e\mbox{-}h} % 03/01/2022 <- mi forma de escribir la energía de creación electrón hueco

\usepackage{enumitem,kantlipsum} %05/01/2022 <- para indentar dentro del enumerate

\usepackage{booktabs} %13/01/2022 Para tener tablas de libros

% 03/02/2022 Command for a cheta L for the lagrangian partir 
\newcommand{\Lagr}{\mathscr{L}}