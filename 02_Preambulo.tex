%%%%%%%%%%%%%%%%%%%%%%%% DOCUMENTCLASS %%%%%%%%%%%%%%%%%%%%%%%%

\documentclass[a4paper,12pt]{report}

%%%%%%%%%%%%%%%%%%%%%%%% PACKAGES %%%%%%%%%%%%%%%%%%%%%%%%

\usepackage[utf8]{inputenc}
\usepackage[spanish, es-tabla]{babel}
\usepackage{anysize}
\usepackage{titlesec} % Para customizar títulos entiendo que es (template darío)
\usepackage{setspace} % Para setear el espacio entre lineas (template darío)

\usepackage{graphicx}%[draft]{graphicx} % Draft hace que no se redeneé la imagen al compilar entonces compilar x1000 más rápido. Ta buenardo (template darío)
\usepackage{float}

\usepackage{afterpage} % Entiendo que con este package y usando el comando \afterpage{\clearpage} cuando te queda una figura muy arriba sin texto abajo y media hoja vacía, te completa el espacio con texto. Está bueno porque latex se chotea seguido en ese sentido.

\usepackage{amsmath,amssymb}
\usepackage{xcolor} % 14/12/2021 texto en color

\usepackage{enumerate} % template dario
\usepackage{lscape} % rota la hoja si por ejemplo hay una tabla o figura muy grande/larga (dudo usarlo) (template darío)
\usepackage{verbatim} % (template darío)
\usepackage{appendix} % (template darío)
\usepackage{array} % (template darío)
\usepackage{physics}

\usepackage{multirow}
\usepackage{multicol}

\usepackage[font=small]{caption} %10/3/2022 para achicar el tamaño de las palabras "Fig" o "Tabla" en los captions
\usepackage{subcaption}
\usepackage{fancybox}
\usepackage{letltxmacro}

\usepackage{xargs}
\usepackage[hidelinks]{hyperref}
% (citas)
\usepackage[superscript,biblabel,nomove]{cite}
% (citas) hace las citas superíndices

\usepackage{mathtools} % 02/02/2022 para tener el prescript{}{} índice y supraíndice del lado izq a misma altura
\usepackage{mathrsfs} %03/02/2022 Para la L del lagrangiano bien cheta 14/10/2021

\usepackage{titlesec}% http://ctan.org/pkg/titlesec
\usepackage{fancyhdr}

\usepackage{siunitx} %03/01/2022 <- Para unidades

\usepackage{enumitem,kantlipsum} %05/01/2022 <- para indentar dentro del enumerate

\usepackage{booktabs} %13/01/2022 Para tener tablas de libros


%%%%%%%%%%%%%%%%%%%%%%%% NEW COMMANDS %%%%%%%%%%%%%%%%%%%%%%%%
\makeatletter 
\renewcommand{\@citess}[1]{\textsuperscript{\,[#1]}}

\newcommand{\eh}{e\mbox{-}h} % 03/01/2022 <- mi forma de escribir la energía de creación electrón hueco

% 03/02/2022 Command for a cheta L for the lagrangian partir 
\newcommand{\Lagr}{\mathscr{L}}

% (raiz) Para hacer la raiz cuadrada con el cierre al final
\makeatletter
\let\oldr@@t\r@@t
\def\r@@t#1#2%
    {%
        \setbox0=\hbox{$\oldr@@t#1{#2\,}$}\dimen0=\ht0
        \advance\dimen0-0.2\ht0
        \setbox2=\hbox{\vrule height\ht0 depth -\dimen0}%
        {\box0\lower0.4pt\box2}%
    }
\LetLtxMacro{\oldsqrt}{\sqrt}
\renewcommand*{\sqrt}[2][\ ]{\oldsqrt[#1]{#2}}
\makeatother
% (raiz) nacho: 31/03/2018 17:35

%%%%%%%%%%%%%%%%%%%%%%%% MODIFICACIONES %%%%%%%%%%%%%%%%%%%%%%%%

% modifican el nombre que muestra3
\addto\captionsspanish{\renewcommand*\contentsname{Índice}}
\addto\captionsspanish{\renewcommand*{\bibname}{Referencias}} % cambia el nombre de la 'Bibliografía' por 'Referencias'

%\titleformat{\chapter}[hang]{\bfseries\Large}{\thechapter.}{12pt}{\bfseries\Large} % tamaño de la letra de chapter. Queda igual que la de section. No sé si me gusta
%\titleformat{\section}[hang]{\bfseries\large}{\thesection}{12pt}{\bfseries\large}


%%%%%%%%%%%%%%%%%%%%%%%% SETEOS %%%%%%%%%%%%%%%%%%%%%%%%

%\setlength{\parskip}{1em} % Por lo que vi, me genera un espaciado en el índice (no me gusta como queda 3jj3j3, negociar) (template darío)

%\setlength{\parindent}{10pt} % modifica la indentación (10 pt es menos que la por defecto y no sé si me gusta. Negociar)
