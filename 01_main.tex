%%%%%%%%%%%%%%%%%%%%%%%% DOCUMENTCLASS %%%%%%%%%%%%%%%%%%%%%%%%

\documentclass[a4paper,12pt]{report}

%%%%%%%%%%%%%%%%%%%%%%%% PACKAGES %%%%%%%%%%%%%%%%%%%%%%%%

\usepackage[utf8]{inputenc}
\usepackage[spanish, es-tabla]{babel}
\usepackage{anysize}
\usepackage{titlesec} % Para customizar títulos entiendo que es (template darío)
\usepackage{setspace} % Para setear el espacio entre lineas (template darío)

\usepackage{graphicx}%[draft]{graphicx} % Draft hace que no se redeneé la imagen al compilar entonces compilar x1000 más rápido. Ta buenardo (template darío)
\usepackage{float}

\usepackage{afterpage} % Entiendo que con este package y usando el comando \afterpage{\clearpage} cuando te queda una figura muy arriba sin texto abajo y media hoja vacía, te completa el espacio con texto. Está bueno porque latex se chotea seguido en ese sentido.

\usepackage{amsmath,amssymb}
\usepackage{xcolor} % 14/12/2021 texto en color

\usepackage{enumerate} % template dario
\usepackage{lscape} % rota la hoja si por ejemplo hay una tabla o figura muy grande/larga (dudo usarlo) (template darío)
\usepackage{verbatim} % (template darío)
\usepackage{appendix} % (template darío)
\usepackage{array} % (template darío)
\usepackage{physics}

\usepackage{multirow}
\usepackage{multicol}

\usepackage[font=small]{caption} %10/3/2022 para achicar el tamaño de las palabras "Fig" o "Tabla" en los captions
\usepackage{subcaption}
\usepackage{fancybox}
\usepackage{letltxmacro}

\usepackage{xargs}
\usepackage[hidelinks]{hyperref}
% (citas)
\usepackage[superscript,biblabel,nomove]{cite}
% (citas) hace las citas superíndices

\usepackage{mathtools} % 02/02/2022 para tener el prescript{}{} índice y supraíndice del lado izq a misma altura
\usepackage{mathrsfs} %03/02/2022 Para la L del lagrangiano bien cheta 14/10/2021

\usepackage{titlesec}% http://ctan.org/pkg/titlesec
\usepackage{fancyhdr}

\usepackage{siunitx} %03/01/2022 <- Para unidades

\usepackage{enumitem,kantlipsum} %05/01/2022 <- para indentar dentro del enumerate

\usepackage{booktabs} %13/01/2022 Para tener tablas de libros


%%%%%%%%%%%%%%%%%%%%%%%% NEW COMMANDS %%%%%%%%%%%%%%%%%%%%%%%%
\makeatletter 
\renewcommand{\@citess}[1]{\textsuperscript{\,[#1]}}

\newcommand{\eh}{e\mbox{-}h} % 03/01/2022 <- mi forma de escribir la energía de creación electrón hueco

% 03/02/2022 Command for a cheta L for the lagrangian partir 
\newcommand{\Lagr}{\mathscr{L}}

% (raiz) Para hacer la raiz cuadrada con el cierre al final
\makeatletter
\let\oldr@@t\r@@t
\def\r@@t#1#2{%
\setbox0=\hbox{$\oldr@@t#1{#2\,}$}\dimen0=\ht0
\advance\dimen0-0.2\ht0
\setbox2=\hbox{\vrule height\ht0 depth -\dimen0}%
{\box0\lower0.4pt\box2}}
\LetLtxMacro{\oldsqrt}{\sqrt}
\renewcommand*{\sqrt}[2][\ ]{\oldsqrt[#1]{#2}}
\makeatother
% (raiz) nacho: 31/03/2018 17:35

%%%%%%%%%%%%%%%%%%%%%%%% MODIFICACIONES %%%%%%%%%%%%%%%%%%%%%%%%

% modifican el nombre que muestra3
\addto\captionsspanish{\renewcommand*\contentsname{Índice}}
\addto\captionsspanish{\renewcommand*{\bibname}{Referencias}} % cambia el nombre de la 'Bibliografía' por 'Referencias'

%\titleformat{\chapter}[hang]{\bfseries\Large}{\thechapter.}{12pt}{\bfseries\Large} % tamaño de la letra de chapter. Queda igual que la de section. No sé si me gusta
%\titleformat{\section}[hang]{\bfseries\large}{\thesection}{12pt}{\bfseries\large}


%%%%%%%%%%%%%%%%%%%%%%%% SETEOS %%%%%%%%%%%%%%%%%%%%%%%%

%\setlength{\parskip}{1em} % Por lo que vi, me genera un espaciado en el índice (no me gusta como queda 3jj3j3, negociar) (template darío)

%\setlength{\parindent}{10pt} % modifica la indentación (10 pt es menos que la por defecto y no sé si me gusta. Negociar)


\pagestyle{fancy}
\lhead{{}}
%%% Dario: Te gusta como queda lo que hace la linea anterior? aparece tu nombre en todas las hojas, es medio mucho no? no lo vi en otras tesis creo. Me gusta más que aparezca el nombre de la sección en la que estas, que es lo que hace si lo dejas comentado.
%\fancyhead[L]{\nouppercase{\slshape \leftmark}} % Parte izquierda del encabezado
%\fancyhead[C]{\nouppercase{\slshape \leftmark}} % Medio del encabezado
\fancyhead[R]{\nouppercase{\slshape \leftmark}} % Parte derecha del encabezado

\setlength{\headheight}{16pt} %Soluciona el warning de " fancyhdr \headheight is to small. use at least 14.9999 o algo así decía << No de nuevo decía >>

\marginsize{2cm}{2cm}{2cm}{2cm}
\renewcommand{\baselinestretch}{1.3} % separa los renglones

\title{\textbf{\Huge Propiedades de los sensores de silicio más allá del ruido de lectura}}
\author{Ignacio Martín Gómez Florenciano}

%\date{Abril 2022}
\date{
\bigskip
\parbox{\linewidth}
    {\centering%
        \endgraf
        \bigskip
        \vspace{8cm}
            Tesis de Licenciatura en Ciencias Físicas
        \endgraf
        \bigskip
            Facultad de Ciencias Exactas y Naturales
        \endgraf
        \bigskip
            Universidad de Buenos Aires
        \endgraf
        \vspace{2cm}
    }
Mayo 2022
\newpage
\thispagestyle{empty} \mbox{}
\thispagestyle{empty}
}

\begin{document}
    
    \maketitle

    %%%%%%%%%%%%  segunda pagina %%%%%%%%%%%%%%%%%%%%%%%%%%%%%%%%%%
\thispagestyle{empty}
\noindent TEMA: Propiedades de los sensores de silicio más allá del ruido de lectura.\\
\\
\noindent ALUMNO: L.U. N$^\circ\ 116/13$\\
\\
\noindent LUGAR DE TRABAJO: Facultad de Ciencias Exactas y Naturales - Universidad de Buenos Aires\\
\\
\noindent DIRECTOR: Dr. Darío P. Rodrigues F. Maltéz\\
\\
\noindent COLABORADOR: Lic. Mariano Cababié\\
\\
\noindent FECHA DE INICIACIÓN: Marzo 2021\\
\\
\noindent FECHA DE FINALIZACIÓN:  Mayo 2022\\
\\
\noindent FECHA DE EXAMEN:\\
\\
\noindent INFORME FINAL APROBADO POR:\\
\vfill
\begin{multicols}{2}    
\rule[0pt]{2.5in}{0.5pt}\\
Autor: Ignacio M. Gómez Florenciano
\vspace{3em}

\rule[0pt]{2.5in}{0.5pt}\\
Director: Dr. Dario Rodrigues
\vspace{3em}
   
\rule[0pt]{2.5in}{0.5pt}\\
Profesora de Tesis de Licenciatura
    
\columnbreak
\rule[0pt]{2.5in}{0.5pt}\\
Jurado 
\vspace{3em}
    
\rule[0pt]{2.5in}{0.5pt}\\
Jurado 
\vspace{3em}
    
\rule[0pt]{2.5in}{0.5pt}\\
Jurado 
\end{multicols}  
%------PÁGINA VACIA ------------------
\newpage
\thispagestyle{empty} \mbox{}
\thispagestyle{empty}
%%%%%%% fin segunda hoja %%%%%%%%%%%%%%%%%%%%%%%%%%%%%%%%%%%%%
    
     %%%%%%%%%%%%%%%%%%%%%%%%%%%%%%%% AGRADECIMIENTOS %%%%%%%%%%%%%%%%%%%%%%%%%%%%%%%%
\newpage
\thispagestyle{empty}
 \vspace*{150pt}
\begin{flushright}
\textit{A mis viejos}
\end{flushright}
%------PÁGINA VACIA ------------------
\newpage
\thispagestyle{empty} \mbox{}
\thispagestyle{empty}
 %%%%%%%%%%%%%%%%%%%%%%%%%%%%%%%% FIN AGRADECIMIENTOS %%%%%%%%%%%%%%%%%%%%%%%%%%%%%%%%

	\newpage
%\begin{center}
%    {\Large \textbf{Resumen}}
%\end{center}
\chapter*{Resumen}
\thispagestyle{empty}
%<<Breve introducción>>
\noindent El ruido de lectura en los sensores CCDs ha sido una limitación inherente a la electrónica de estos dispositivos, imponiendo límites en la resolución con que podía medirse la carga en ellos generada. Los sensores \textit{Skipper}-CCDs, por su parte, han logrado superar esta barrera, reduciendo el ruido de lectura a niveles subelectrónicos. Sin embargo, el ruido de lectura no es el único factor que está en juego al momento de procesar mediciones realizadas con este tipo de sensores: los eventos provenientes de fuentes que no son de interés y los inducidos por fenómenos intrínsecos a la naturaleza de las interacciones, deben ser también considerados.
%también se encuentran los eventos provenientes de fuentes que no son de interés. Estos son de diferente origen, tanto externos al detector como intrínsecos. En esta última categoría puede incluirse los eventos inducidos por deficiencia en la colección de carga.


%\textbf{<<Objetivo de la tesis>>}
%En esta tesis se estudió el fondo presente en un conjunto de más de $900$ imágenes tomadas con un sensor \textit{Skipper}-CCD de forma de caracterizarlo con precisión en sus valores medios.%\textcolor{red}{no queda claro que serían en este punto los valores medidos ...} 
%Una vez logrado esto, se utilizó esta información junto con un novedoso modelo para el ajuste de los datos con el fin de corregir el sesgo que el fondo \textcolor{red}{el modelo de ajuste (el del beta) no corrige el sesgo que mete el fondo} supone en la determinación de magnitudes como el factor de Fano y la energía de creación electrón-hueco del silicio, para energía inferiores a los $2\,\si{keV}$.
En esta tesis se estudió el fondo presente en un conjunto de más de $900$ imágenes tomadas con un sensor \textit{Skipper}-CCD de forma de caracterizarlo con precisión en sus valores medios. Se probaron tres diferentes umbrales de detección de eventos sobre las imágenes, de forma de elegir el que maximice la estadística y minimice el sesgo. Una vez logrado esto, se utilizó esta información con el fin de corregir el sesgo que el fondo supone en la determinación de magnitudes como el factor de Fano y la energía de creación electrón-hueco del silicio, junto con un novedoso modelo de ajuste de los datos, para energías inferiores a los $2\,\si{keV}$.



%\textbf{<<Trabajo realizado>>}
% Durante el desarrollo de este trabajo se estudió por medio de simulaciones Monte Carlo el fenómeno de ionización, partiendo de un modelo muy sencillo para luego adoptar un modelo más sofisticado. De estas simulaciones se obtuvieron resultados para el factor de Fano semejantes a los obtenidos experimentalmente. En lo que respecta al tratamiento de datos, se probaron diferentes cortes de calidad sobre las imágenes, de forma de elegir el que maximice la estadística y minimice el sesgo \textcolor{red}{a que cortes de calidad se refiere?}. Además se calcularon las correcciones a aplicar sobre las magnitudes de interés, mejorando su incerteza por el aumento en la estadística.
%Durante el desarrollo de este trabajo se estudió por medio de simulaciones Monte Carlo el fenómeno de ionización, con el fin de entender por qué el factor de Fano medido experimentalmente es un orden de magnitud menor a la unidad.

En paralelo, se estudió por medio de simulaciones Monte Carlo el fenómeno de ionización, partiendo de un modelo muy sencillo para luego adoptar un modelo más sofisticado, con el fin de poder entender mejor por qué el factor de Fano medido experimentalmente es un órden de magnitud menor a la unidad. De estas simulaciones se obtuvieron resultados para el factor de Fano semejantes a los obtenidos experimentalmente.
%Durante el desarrollo de este trabajo se estudió por medio de simulaciones Monte Carlo el fenómeno de ionización, partiendo de un modelo muy sencillo para luego adoptar un modelo más sofisticado. De estas simulaciones se obtuvieron resultados para el factor de Fano semejantes a los obtenidos experimentalmente. En lo que respecta al tratamiento de datos, se probaron tres diferentes umbrales de detección de eventos sobre las imágenes, de forma de elegir el que maximice la estadística y minimice el sesgo. Además se calcularon las correcciones a aplicar sobre las magnitudes de interés.




%\textbf{<<Resultados>>}
%Gracias a estos análisis, se determinaron magnitudes como \textcolor{red}{al decir magnitudes como pareciera que son algunas de un conjunto mas grande que tambien se midieron}\textcolor{green}{No entendí} el factor de Fano con una mejora en su incerteza del $N\,\%$ y la energía de creación electrón-hueco con una mejora del $M\,\%$.


%\textbf{<<Perspectivas a futuro?>>}
% Los resultados de este trabajo presentan una mejora en la determinación de las magnitudes antes mencionadas \textcolor{red}{yo no hablaría de mejoras, sino de haber medido en un rango antes inaccesible}, lo cual es de vital importancia en la caracterización de estos dispositivos en el rango de bajas energías, donde los CCDs convencionales presentan importantes barreras debido al ruido de lectura. Es por esto que la importancia de este trabajo radica en la combinación de la tecnología \textit{Skipper} a la vez del uso de técnicas de análisis que permitieron mejorar las incertezas de estas magnitudes que, hasta el momento, no habían podido medirse con esta precisión.
Los resultados de este trabajo presentan la determinación de las magnitudes antes mencionadas para un rango de energías inaccesible hasta el momento, lo cual es de vital importancia en la caracterización de estos dispositivos en el rango de bajas energías, donde los CCDs convencionales presentan importantes barreras debido al ruido de lectura. Es por esto que la importancia de este trabajo radica en la combinación de la tecnología \textit{Skipper} con el uso de técnicas de análisis que permitieron corregir sesgos en mediciones que, hasta el momento, no habían podido realizarse.



%------PÁGINA VACIA ------------------
\newpage
\thispagestyle{empty} \mbox{}
\thispagestyle{empty}
%-------------------------------
    
    \newpage
\pagenumbering{roman} %numero de pagina en números romanos
\setcounter{page}{1}  %comienza a contar desde 1

\tableofcontents

\newpage
\thispagestyle{empty} \mbox{}
\thispagestyle{empty}

    \newpage
\pagenumbering{arabic}  %numero de pagina numeración arábiga
\setcounter{page}{1}    %comienza a contar desde 1

\chapter{Introducción y objetivos}

\noindent Como parte de este trabajo se propuso el uso de técnicas de análisis de imágenes y procesamiento de datos propias de la física de partículas experimental. %, lo que llevó al fortalecimiento de conocimientos de estadística y la familiarización con el trabajo científico en un tema de creciente interés. 
Las metas propuestas en este trabajo son:
\begin{enumerate}
    \item Analizar las mediciones con luz LED para obtener una calibración absoluta de carga del detector a baja ocupancia y a diferentes temperaturas.
    \item Analizar las mediciones de rayos X producidos por desexcitación del flúor y aluminio, con el objetivo de determinar la energía de creación electrón hueco y el factor de Fano a $677\,\si{eV}$ y $1486\,\si{eV}$, respectivamente.
    \item Estudiar de la dependencia con la temperatura en el rango de $123$ a $160\,\si{K}$ de las cantidades anteriormente mencionadas.
    \item Estudiar la influencia de otras fuentes de fotones en la construcción de eventos y en la determinación del factor de Fano y energía de creación electrón hueco.
    \item Corregir el efecto que la luz espuria tiene sobre el valor medio de carga y su distribución.
\end{enumerate}

%%%%%%%%%%%%%%%%%%%%%%%%%%%%%%%%%%%%%%%%%%%%%%%%%%%%%%%%%%%%%%%%%%
%%%%%%%%%%%%%%%%%%%%%%%%%%%%%%%%%%%%%%%%%%%%%%%%%%%%%%%%%%%%%%%%%%
\section{Factor de Fano y energía de creación electrón-hueco}
\noindent El factor de Fano mide la dispersión de una distribución de carga producida en un detector y se define como
\begin{equation*}
    F = \frac{\sigma^{2}}{\mu}
\end{equation*}
donde $\sigma^{2}$ es la varianza de la distribución y $\mu$ es la media. 
Para el caso particular de una distribución de Poisson, la varianza y la esperanza coinciden, de forma que el factor de Fano equivale a $1$. Estas distribuciones de carga son el producto de la interacción de fotones de cierta energía con el detector, entre otros factores, que van depositando energía en el material ionizando cargas a su paso. El origen de dicha distribución radica en que la energía transferida en cada interacción no es constante y, por lo tanto, para una dada energía inicial de la partícula incidente, tampoco será constante el número de cargas generadas.\\
\indent Por otro lado, la energía de creación electrón-hueco $\varepsilon_{\eh}$ es, en valor medio, la energía necesaria para poder producir un par electrón-hueco en el interior del detector de Silicio. Así, esta puede ser calculada a través del cociente entre la energía entregada al detector y la carga producida en él. Además, esta está relacionada con la energía del gap del silicio, entre la banda de valencia y la banda de conducción, que es $E_{g}\sim 1.1\,\si{eV}$\cite{Janesick}. Sin embargo, debido a que durante el proceso de interacción parte de la energía que entregada material puede disiparse emitiendo fonones, la energía de creación electrón-hueco resulta ser mayor, en promedio, que la energía del gap.\\
\indent La estimación precisa de ambas magnitudes es de vital importancia en la caracterización de este tipo de detectores, debido a que, por ejemplo, parámetros como la eficiencia cuántica dependen fuertemente de ellos. Además es importante determinar la dependencia de estas magnitudes con la energía ya que, en particular, el factor de Fano a energías por debajo de $1\,\si{keV}$ es clave para el cálculo de sensibilidad en experimentos de materia oscura liviana, como es caso del experimento SENSEI (\textit{Sub-Electron Noise Skipper-CCD Experiment Instrument})\cite{barak}.
%%%%%%%%%%%%%%%%%%%%%%%%%%%%%%%%%%%%%%%%%%%%%%%%%%%%%%%%%%%%%%%%%%
%%%%%%%%%%%%%%%%%%%%%%%%%%%%%%%%%%%%%%%%%%%%%%%%%%%%%%%%%%%%%%%%%%
\section{CCD y Skipper CCD}
\noindent Los dispositivos CCD (\textit{Charge Coupled Devices}) fueron inventados en 1969 en los Laboratorios Bell, por Willard Boyle y George Smith, en su búsqueda por fabricar dispositivos de memoria. Finalmente, los CCD's no cumplieron este objetivo pero sí demostraron un gran potencial como sensores de luz y partículas. Tal es así que en el año 2010 sus inventores recibieron el premio Nobel de física\cite{Boyle, Smith}.\\
\indent Estos dispositivos están hechos esencialmente de Silicio y sus elementos constitutivos fundamentales son capacitores MOS (por \textit{metal-oxide-semiconductor}). Estos conforman los píxeles del detector, siendo por lo general millones y ocupando casi la totalidad de la superficie del sensor. Los capacitores MOS se componen generalmente de un sustrato semiconductor dopado, sobre el cual se deposita una delgada capa de óxido y a su vez sobre esta se coloca un metal de contacto. Este contacto metálico se encuentra a un voltaje $V_{G}$ y debajo del semiconductor se encuentra otro contacto que se encuentra a tierra. Dependiendo del valor de $V_{G}$ se obtienen distintos regímenes del MOS\cite{Chenming} donde, en particular, uno de ellos genera una región de depleción cerca del óxido, el cual permite acumular carga minoritaria.\\
\begin{figure}%[H]
    \centering
        \includegraphics[scale=.35]{Figs/PixelCrossSection.pdf}
    \caption{\footnotesize{Esquema lateral de un capacitor MOS (píxel).}}
    \label{fig:PixelCrossSection}
\end{figure}
\indent El principio de operación de un CCD se puede dividir en cuatro etapas, que a grandes rasgos son:
\begin{itemize}
    \item Exposición del detector: El tiempo de exposición es variable y depende del tipo de medición que se desee utilizar. Durante la exposición, la radiación incidente interactúa con el detector, generalmente generando pares electrón-hueco. 
    \item Colección: Los electrones son luego arrastrados por el campo eléctrico del detector presente en su volumen hacia los pozos de potencial de los píxeles donde son colectados.
    \item Transferencia: Dado que la medición de los píxeles se realiza de forma secuencial, la carga en cada uno de ellos debe ser transferida de un píxel a otro.
    \item Medición de la carga: A medida que se desplaza la carga, esta es llevada hacia el nodo de sensado donde finalmente es medida.
\end{itemize}
Los CCD's convencionales son capaces de alcanzar ruidos de lectura del orden de los $2\,e^{-}\si{rms/pix}$, gracias a la técnica de muestreo doblemente correlacionado\cite{Tiffenberg}. Sin embargo, en aplicaciones de bajas energías, el ruido electrónico de lectura presupone una barrera al límite de energías que pueden medirse con estos sensores para mantener la precisión deseada. La imagen \ref{fig:Fano_y_ruido} pretende mostrar este efecto. La linea punteada representa el ruido constante de lectura presente en estos sensores y la recta representa un valor constante del factor de Fano para todo el rango de energías. Sobre la curva se ven los puntos que representan los valores del factor de Fano que se medirían si se utilizaran estos sensores debido a la suma de contribuciones del ruido sobre un factor de Fano constante de $0.1$.
\begin{figure}[H]
    \centering
        \includegraphics[scale=.5]{Figs/fano_y_ruido.pdf}
    \caption{\footnotesize{Valores que se obtendrían de medir el factor de Fano con un sensor CCD convencional (puntos negros), debido a la contribución del ruido de lectura constante de $30\,\si{eV}$ (linea punteada horizontal). La linea recta continua representa un valor de factor de Fano constante.}}
    \label{fig:Fano_y_ruido}
\end{figure}
Los sensores \textit{Skipper}-CCD's, por otro lado, permiten disminuir el ruido electrónico de lectura a niveles subelectrónicos gracias a que son capaces de medir la carga en los píxeles de forma no destructiva. Esto permite tomar tantas mediciones como sean necesarias de la carga y estimar la carga real a partir de un promedio sobre el número de mediciones tomadas. El prácticamente ausente ruido de lectura que hace posible la capacidad de medir repetidas veces y de forma no destructiva la carga en cada píxel de un sensor \textit{Skipper}-CCD, tiene un fuerte impacto a energías por debajo de los $5\,\si{keV}$. En particular, por debajo de los $2\,\si{keV}$ la contribución del ruido de lectura de un CCD convencional a la determinación de estas cantidades puede superar el $30\,\%$. 
Esta es la principal razón que motiva este trabajo, donde se realiza un estudio sistemático del Factor de Fano a bajas energías, entre $1486\,\si{eV}$ (rayos X del Al) y $677\,\si{eV}$ (rayos X del flúor).\\


%%%%%%%%%%%%%%%%%%%%%%%%%%%%%%%%%%%%%%%%%%%%%%%%%%%%%%%%%%%%%%%%%%
%%%%%%%%%%%%%%%%%%%%%%%%%%%%%%%%%%%%%%%%%%%%%%%%%%%%%%%%%%%%%%%%%%
\section{Antecedentes}

\noindent En trabajos previos se han estudiado las ventajas de la utilización de la novedosa tecnología \textit{Skipper} en los CCDs, para lograr medir con precisión subelectrónica en regímenes de energía donde los sensores CCD convencionales más precisos sólo podrían alcanzar resoluciones del orden de los $2$ electrones. Por primera vez fue usada para poder medir el factor de Fano y la energía de creación electrón-hueco en el Silicio a una energía de $5.9\,\si{keV}$ a $123\,\si{K}$\cite{Rodrigues}.\\
\indent Para lograr esto, se implementó un método de calibración absoluta que determinó la relación entre el número de electrones en cada píxel y la lectura del valor en ADUs (\textit{Analog Digital Unit} o Unidades analógico-digitales). El procedimiento para la calibración consistió en la utilización de un LED %instalado en el \textit{dewar}, 
que emitía fotones en $405\,\si{nm}$ de longitud de onda, para poblar los píxeles del sensor. Así, lograron poblar los píxeles con un amplio rango de cargas realizando un barrido en el tiempo de exposición del sensor a la luz LED. La medición de carga se realizó tomando $300$ lecturas por cada píxel, gracias a la tecnología \textit{Skipper} que permite realizar múltiples muestreos de forma no destructiva, que luego fueron promediadas logrando reducir el ruido de lectura en un factor $\sqrt{300}$. Como resultado, se obtuvieron distribuciones de carga Gaussianas en los posibles niveles de ocupación de carga, con suficiente resolución para cada nivel, haciendo posible distinguir perfectamente entre picos consecutivos, como se puede ver en la figura \ref{fig:Calibracion}. De esta forma, mediante un ajuste gaussiano se pudo establecer el valor medio en ADUs para cada uno de estos picos, estableciendo como valor de carga el número de pico correspondiente. Así es que se obtuvo una relación $1$ a $1$ entre cantidad de carga por píxel y ADUs. Cabe destacar que para comenzar a numerar los picos, primero es necesario establecer el valor de $0$ carga, es decir, píxel vacío, lo cual no corresponde, a priori, a un valor nulo en ADUs. Para ello, las imágenes tomadas con \textit{Skipper} cuentan con una región denominada \textit{overscan} que se utiliza para calcular la linea de base y poder substraerla luego a la carga de cada píxel, logrando así que la media de los píxeles vacíos quede en cero ADUs.
\begin{figure}[H]
%Como reproducir este gráfico: correr el script NivelesOcupacionCarga.py ubicado en /Escritorio/Tesis2021/Figs/pys_para_plots y buscar la imagen en /home/igna/Escritorio/Tesis2021/Figs/
    \centering
        \includegraphics[scale=.5]{Figs/ajuste_gaussiano_calibracion.pdf}
    \caption{\footnotesize{Histograma de los datos obtenidos al iluminar el CCD con LED, correspondiente a una región con poca ocupación, donde los picos de los ajustes entre electrones se distinguen a la perfección.}}
    \label{fig:Calibracion}
\end{figure}
Gracias a esto, por primera vez se midieron el factor de Fano $F$ de forma absoluta y la energía de creación electrón-hueco $\varepsilon_{\eh}$ utilizando esta tecnología. Esto se realizó midiendo rayos $X$ de $5.9\,\si{\mbox{keV}}$ emitidos por una fuente de $\prescript{55}{}{\mbox{Fe}}$. Más precisamente, rayos $X_{K}$, cuyas energías son los de la tabla \ref{tab:EnergiasXk}.
\begin{table}[h]
\centering
\begin{tabular}{@{}ccc@{}}
\toprule
$X_{K}$         &   Energía [keV]   &   Intensidad relativa \\ \hline \hline
$\alpha_{2}$    &   $5887.6$        &   $8.5 (4)$           \\
$\alpha_{1}$    &   $5898.8$        &   $16.9 (8)$          \\
$\beta_{3}$     &   $6490.4$        &   $4.1 (11)$          \\ \bottomrule
\end{tabular}
\caption{\footnotesize{Energías e intensidades relativas de los fotones X emitidos tras el decaimiento de $\prescript{55}{}{\mbox{Fe}}$}}
\label{tab:EnergiasXk}
\end{table}
Sobre los resultados de las mediciones realizadas para estos rayos $X$, se realizó un ajuste de los picos del espectro utilizando la verosimilitud, cuya expresión es la de la ecuación \eqref{ec:verosimilitud}, que surge de la convolución de dos distribuciones exponenciales y una distribución gaussiana
{\small
\begin{align}
    \Lagr(e|\mu_{1},
            \mu_{2},
            \sigma_{1},
            \lambda_{1},
            \lambda_{2},
            \eta_{1} = \eta_{2},
            \eta_{3})
    = &
    \sum\limits_{j=1}^{3} I_{j}
    \left\{
        \eta_{j}\frac{\lambda_{1}}{2}
        \exp
            \left[
                (e-\mu_{j})\lambda_{1} + \frac{\sigma_{j}^{2}\lambda_{1}^{2}}{2}
            \right]
        \mbox{Erfc}
        \left[
            \frac{1}{\sqrt{2}}
            \left(
                \frac{e - \mu_{j}}{\sigma_{j}}
                +\sigma_{j}\lambda_{1}
            \right)
        \right] \right. \nonumber
        \\
        + &
        \left.
        (1-\eta_{j})\frac{\lambda_{2}}{2}
        \exp
            \left[
                 (e - \mu_{j})\lambda_{2}
                 + \frac{\sigma_{j}^{2}\lambda_{2}^{2}}{2}
            \right]
        \mbox{Erfc}
        \left[
            \frac{1}{\sqrt{2}}
            \left(
                \frac{e - \mu_{j}}{\sigma_{j}}
                +\sigma_{j}\lambda_{2}
            \right)
        \right]
    \right\}
        \label{ec:verosimilitud}
\end{align}
}
donde $\mu_{j}$, $\sigma_{j}$ y $I_{j}$ representan el valor medio de carga, la desviación estándar del valor medio de carga y la intensidad relativa del pico $j$-ésimo con energía $E_{j}$, respectivamente. Además, $\lambda_{1}$ y $\lambda_{2}$ son parámetros de la distribuciones exponenciales y $\eta_{j}$ es el peso relativo entre ellas.\\
\indent La figura \ref{fig:AjusteNoBineado} representa el ajuste mediante la verosimilitud de los picos de rayos $X$ para un total de $18085$ eventos. Lo resultados para la medición del factor de Fano, la energía de creación electrón-hueco y demás parámetros relevantes se encuentran en la tabla \ref{tab:ParametrosAjusteNoBineado}.
\begin{figure}%[H]
    \centering
        \includegraphics[scale=.15]{Figs/AjusteNoBineado.png}
    \caption{\footnotesize{Ajuste no bineado de los picos de $\prescript{55}{}{\mbox{Fe}}$.}}
    \label{fig:AjusteNoBineado}
\end{figure}
\begin{table}[h]
\centering
\begin{tabular*}{\textwidth}{c @{\extracolsep{\fill}} ccccccccc}%{@{}ccccccccc@{}}
\toprule
$X_{K}$ &
  $\mu\ [e^{-}]$ &
  $\Delta \mu\ [e^{-}]$ &
  $\sigma\ [e^{-}]$ &
  $\Delta \sigma\ [e^{-}]$ &
  $F$ &
  $\Delta F$ &
  $\varepsilon_{\eh}\ [\mbox{eV}/e^{-}]$ &
  $\Delta \varepsilon_{\eh} \ [\mbox{eV}/e^{-}]$ \\ \hline\hline
$\alpha_{2}$ &
  $1570.50$ &
  $0.18$ &
  $13.68$ &
  $0.12$ &
  \multirow{3}{*}{$0.119$} &
  \multirow{3}{*}{$0.002$} &
  \multirow{2}{*}{$3.749$} &
  \multirow{2}{*}{$0.001$} \\
$\alpha_{1}$ & $1573.48$ & $0.18$ & $13.69$ & $0.12$ &  &  &         &         \\
$\beta_{3}$  & $1730.50$ & $0.55$ & $14.36$ & $0.13$ &  &  & $3.751$ & $0.002$ \\ \bottomrule
\end{tabular*}
\caption{tabla}
\label{tab:ParametrosAjusteNoBineado}
\end{table}
Estos constituyen los primeros resultados obtenidos de la utilización de la tecnología \textit{Skipper}-CCD para el cálculo del factor de Fano y de la energía de creación electrón-hueco para una temperatura de $123\,\si{K}$ y se encontraron en buen acuerdo con la bibliografía preexistente~\cite{Ryan, Alig, Kotov}.\\
%Comentarios
\indent Además, se avanzó en las primeras mediciones del factor de Fano y la energía de creación electrón-hueco para energías por debajo de los de $2\,\si{keV}$~\cite{TesisKevin}, más específicamente, para energías de $677\,\si{eV}$ y $1486\,\si{eV}$, que corresponden a los rayos $X$ de fluorescencia del flúor y del aluminio respectivamente. En la tabla \ref{tab:EnergiasFluorescenciaFAl} pueden verse las energías e intensidades relativas para los rayos $X$ de estos elementos. En este caso, utilizando toda la maquinaria desarrollada en el trabajo anterior, realizaron mediciones con una fuente de $\prescript{241}{}{\mbox{Am}}$ que emite partículas alfa con una energía de $\sim 5.6\,\si{MeV}$. A estas partículas luego se las hizo impactar contra una cinta de teflón y una barra de aluminio que por fluorescencia emitían rayos $X$ con las energías antes mencionadas. Cabe destacar que se trató de evitar que las partículas alfa alcanzaran el detector, dado que son partículas muy energéticas y por tanto no son eventos de interés.\\
\indent Uno de los desafíos que surgieron en este otro trabajo fue que solo una fracción de los átomos que son impactados por las partículas alfa se desexcitan emitiendo fotones $X$ en las energías de interés. Es por esto que la tasa de eventos medidos es considerablemente menor en comparación a la que se obtuvo al medir con la fuente de $\prescript{55}{}{\mbox{Fe}}$.
\begin{table}[h]
\centering
\begin{tabular}{@{}cccc@{}}
\toprule
Elemento    &   $X_{K}$         &   Energía [eV]    &   Intensidad relativa \\ \hline \hline
F           &   $\alpha_{1,2}$  &   $676.8$         &   $148$               \\
Al          &   $\alpha_{2}$    &   $1486.3$        &   $50$                \\
Al          &   $\alpha_{1}$    &   $1486.7$        &   $100$               \\
Al          &   $\beta_{1}$     &   $1557.4$        &   $1$                 \\ \bottomrule
\end{tabular}
\caption{\footnotesize{Caption}}
\label{tab:EnergiasFluorescenciaFAl}
\end{table}
Con los datos de estas mediciones se reconstruyeron los espectros para ambos picos y se les realizó un ajuste no bineado, similar al utilizado en los experimentos con $\prescript{55}{}{\mbox{Fe}}$, utilizando la verosimilitud descripta por la expresión \eqref{ec:verosimilitudF-Al}.
\begin{align}
    \Lagr(e|\mu,
            \sigma,
            \lambda_{1},
            \lambda_{2},
            \eta)
    = &
    \eta
    \left\{
        \frac{\lambda_{1}}{2}
        \exp\left[
                (e-\mu)\lambda_{1} + \frac{\sigma^{2}\lambda_{1}^{2}}{2}
            \right]
        \mbox{Erfc}
        \left[
            \frac{1}{\sqrt{2}}
            \left(
                \frac{e - \mu}{\sigma}
                +\sigma\lambda_{1}
            \right)
        \right] \right\} \nonumber
        \\
        + &
        (1-\eta)
        \left\{
        \frac{\lambda_{2}}{2}
        \exp
            \left[
                 (e - \mu)\lambda_{2}
                 + \frac{\sigma^{2}\lambda_{2}^{2}}{2}
            \right]
        \mbox{Erfc}
        \left[
            \frac{1}{\sqrt{2}}
            \left(
                \frac{e - \mu}{\sigma}
                +\sigma\lambda_{2}
            \right)
        \right]
    \right\}
        \label{ec:verosimilitudF-Al}
\end{align}
En este caso, el ajuste resultó mucho más sensible a los cambios en el bineado del histograma, debido a que la tasa de eventos registrados para este experimento fue mucho menor. Los resultados obtenidos en este trabajo a partir de los ajustes de estas mediciones se encuentran en la tabla \ref{tab:ParametrosAjusteNoBineadoF-Al}.\\
\begin{table}[h]
\centering
\begin{tabular*}{\textwidth}{c @{\extracolsep{\fill}} ccccccccc}%{@{}ccccccccc@{}}
\toprule
Elemento&
  $\mu\ [e^{-}]$ &
  $\Delta \mu\ [e^{-}]$ &
  $\sigma\ [e^{-}]$ &
  $\Delta \sigma\ [e^{-}]$ &
  $F$ &
  $\Delta F$ &
  $\varepsilon_{\eh}\ [\mbox{eV}/e^{-}]$ &
  $\Delta \varepsilon_{\eh} \ [\mbox{eV}/e^{-}]$ \\ \hline\hline
  F &   $182.0$ &   $0.8$  &   $7.0$   &   $0.7$   &   $0.27$  &   $0.05$  &   $3.72$ &   $0.02$\\
  Al&   $404.4$ &   $0.4$  &   $8.3$   &   $0.3$   &   $0.17$  &   $0.01$  &   $3.679$ &   $0.004$\\ \bottomrule
\end{tabular*}
\caption{tabla}
\label{tab:ParametrosAjusteNoBineadoF-Al}
\end{table}
\indent Hasta aquí los resultados previos obtenidos con la utilización de esta novedosa tecnología de \textit{Skipper}-CCD para la medición del factor de Fano y la energía de creación electrón hueco. Sin embargo, todavía se pueden mejorar los resultados aumentando la estadística de los eventos para las energías inferiores a $2\,\si{keV}$ caracterizando el fondo presente en las imágenes. Para lograr esto es necesario realizar un tratamiento sobre los datos existentes.
%%%%%%%%%%%%%%%%%%%%%%%%%%%%%%%%%%%%%%%%%%%%%%%%%%%%%%%%%%%%%%%%%%
%%%%%%%%%%%%%%%%%%%%%%%%%%%%%%%%%%%%%%%%%%%%%%%%%%%%%%%%%%%%%%%%%%
\section{Motivación del análisis de imágenes}
%\noindent Este trabajo se centró en la determinación del factor de Fano y de la energía de creación electrón-hueco, para energías por debajo de los $2\,\si{keV}$, utilizando mediciones preexistentes. Más precisamente, para las energías de los rayos $X$ de fluorescencia del aluminio de $1486\,\si{eV}$ y del flúor de $677\,\si{eV}$. Con el fin de que la determinación de estas características esté lo menos sesgada posible debido al fondo presente en las imágenes, fue necesario realizar un análisis profundo de los datos.\\
\noindent El fondo presente en las imágenes consiste principalmente en píxeles cuya carga es producida por fluctuaciones térmicas en la red cristalina del Silicio del sensor (corrientes oscuras), por eventos de dispersión producidos por rebotes dentro del dispositivo de medición y que no deseados, o producidos por fotones infrarrojos emitidos desde los materiales que rodean al detector y por eventos muy penetrantes provenientes del exterior, como por ejemplo, muones.\\
\indent El análisis consistió en estudiar el efecto que produce el fondo de las imágenes sobre los eventos de interés: la aglomeración de píxeles con cargas entre $1$ y $2$ electrones alrededor de los clusters, y la carga extra añadida sobre ellos. En muchos casos resulta que los píxeles con eventos de fondo se aglutinan a los clusters, aumentando su tamaño y su carga, o incluso también haciendo de puente entre dos clusters vecinos. Estos son dos efectos indeseados, primero porque sesga la cantidad de carga real en un evento de interés y segundo porque los algoritmos de clusterizacion podrían ignorarlos al no cumplir con los cortes de calidad impuestos, tanto por forma como por cantidad de carga esperada, disminuyendo así la estadística.\\
\indent Es entonces necesario utilizar un umbral de detección que ignore estos píxeles con carga menor o igual a $2$ electrones, de forma de evitar el aglutinamiento de píxeles con eventos producto de fondo a los clusters de interés y así aumentar la estadística en el conteo de eventos. Pero también es necesario lograr caracterizar este fondo para corregir el sesgo introducido por el nuevo umbral de detección, el cual, también eliminará píxeles con carga genuina. Es por eso que en este trabajo se propuso un análisis de las imágenes que pueda determinar el umbral más conveniente a utilizar y recuperar la mayor cantidad de estadística posible, además de un método para estimar cuántos eventos genuinos son removidos y cuántos eventos espurios hay que remover de los clusters para mejorar así las incertezas del factor de Fano y la energía de creación electrón-hueco a bajas energías.
%%%%%%%%%%%%%%%%%%%%%%%%%%%%%%%%%%%%%%%%%%%%%%%%%%%%%%%%%%%%%%%%%%
%%%%%%%%%%%%%%%%%%%%%%%%%%%%%%%%%%%%%%%%%%%%%%%%%%%%%%%%%%%%%%%%%%
\section{Organización de la tesis}
Esto lo escribo al final y es un resumen de como está armada la tesis.
%%%%%%%%%%%%%%%%%%%%%%%%%%%%%%%%%%%%%%%%%%%%%%%%%%%%%%%%%%%%%%%%%%
%%%%%%%%%%%%%%%%%%%%%%%%%%%%%%%%%%%%%%%%%%%%%%%%%%%%%%%%%%%%%%%%%%
%%%%%%%%%%%%%%%%%%%%%%%%%%%%%%%%%%%%%%%%%%%%%%%%%%%%%%%%%%%%%%%%%%
    
    \chapter{Estudio de la interacción utilizando simulaciones Monte Carlo \label{chap:simulaciones}}
%%%%%%%%%%%%%%%%%%%%%%%%%%%%%%%%%%%%%%%%%%%%%%%%%%%%%%%%%%%%%%%%%%
%%%%%%%%%%%%%%%%%%%%%%%%%%%%%%%%%%%%%%%%%%%%%%%%%%%%%%%%%%%%%%%%%%
\noindent En este capítulo se presentan los resultados de modelar de forma cualitativa el proceso que se estudiará en esta tesis a partir de simulaciones montecarlo.


\section{Modelado del fenómeno físico}
\noindent Uno de los aspectos que se propuso estudiar en este trabajo es el de las razones físicas que hacen que el factor de Fano sea un número mucho menor que la unidad, al contrario de lo que podría esperarse desde una descripción puramente poissoniana de la interacción, donde el factor de Fano debería ser $1$.

Dado que el proceso de ionización de carga es un proceso estocástico en el cual un electrón ionizado (por ejemplo por efecto fotoeléctrico al interactuar con un fotón) puede o no interactuar con otros electrones de la red, puede pensarse como una serie de experimentos de Bernoulli, donde el \textit{éxito} es ionizar y generar otro par electrón hueco y el \textit{fracaso} es no hacerlo. Como la probabilidad de ionización es baja, pero la cantidad de veces que puede darse la interacción es muy alta, en el límite el proceso es poissoniano. Sin embargo, experimentalmente se observa que cuando toda la energía de la partícula incidente es depositada en el material, el factor de Fano resulta casi un orden de magnitud menor\cite{TesisKevin}. Una de las posibles razones por las que esto sucede es que la energía de la partícula no solo es disipada en forma de ionización de carga, sino también en excitación de fonones de la red cristalina del material.

Cuando un fotón de alta energía cinética impacta contra el sensor, este produce una dispersión por ionización y emisión de fonones de la red cristalina del silicio, produciendo así una cascada de pares electrón-hueco. El número de pares producidos puede ser luego medido y el valor de la energía de creación electrón-hueco promediado a partir de este.

Este fenómeno es estudiando en los trabajos de R.C. Alig et al.\cite{Alig}, mediante simulaciones de Monte Carlo y luego por K. Ramanathan \cite{Ramanathan}, compilando resultados y propuestas de diferentes trabajos. En el primero se propone un modelo en el que la partícula incidente interactúa con el material, generando pares electrón-hueco por ionización en forma de cascada y, eventualmente, perdiendo energía por emisión de fonones. La forma en que la partícula incidente pierde energía depende fuertemente de la energía que tiene al momento de ionizar. Esta dependencia está modelada en el trabajo de Ramanathan\cite{Ramanathan} y lo llaman \textit{modelo simplificado}, donde proponen que la energía $E$ que se transfiere para generar pares electrón-hueco se reparte según una distribución Beta, de la forma
\begin{equation*}
    p(x|\alpha) = \frac{2}{B(\alpha)} x^{\alpha - 1}(1-x)^{\alpha - 1}
\end{equation*}
donde $x = \frac{E}{E_{R} - E_{g}}$ es la variable aleatoria, con $E_{r}$ la energía inicial de la partícula en cada ionización, $E_{g}$ es la energía del gap del silicio y $E$ la fracción de energía que va a parar a un nuevo par electrón hueco. Utilizando esta distribución para generar realizaciones de la variable aleatoria $x$, se puede despejar el valor de $E$, que es la energía transferida para generar pares electrón-hueco según este modelo. Por otro lado, $B(\alpha)$ es la función Beta con un único parámetro $\alpha$, y viene dada por
\begin{equation*}
    B(\alpha, \beta) 
    = \frac{\Gamma(\alpha)\Gamma(\beta)}{\Gamma(\alpha) + \Gamma(\beta)}
    \longrightarrow
    B(\alpha)
    = \frac{\Gamma^{2}(\alpha)}{2\Gamma(\alpha)}
\end{equation*}
donde $\alpha$ depende de la energía $E_{r}$ y es el parámetro que determina el régimen de distribución de la energía, o en otras palabras, la forma de la distribución.

La motivación de la utilización de la distribución Beta para modelar cómo se reparte la energía en la generación de pares electrón-hueco por ionización se debe a que esta se adapta muy bien a los tres tipos de regímenes de energía en los que se puede encontrar la partícula incidente, según los trabajos mencionados anteriormente:
\begin{itemize}
    \item \textbf{A bajas energías de la partícula incidente}, se tienen distribuciones muy picudas en los extremos posibles: $E = 0$ y $E = E_{R}+E_{g}$,
    \item \textbf{A energías mucho mayores que la energía del gap}, $E_{R} >> E_{g}$: Se tiene una distribución aproximadamente uniforme,
    \item \textbf{A energías entre $2.2\,\si{eV}$ - $4.2\,\si{eV}$} se tiene una distribución de energía muy picuda en el medio de $x = E/(E_{R} - E_{g})$.
\end{itemize}
Para energías bajas, el parámetro $\alpha$ tiende a cero y se tiene una distribución con máximos en los extremos del intervalo. Para energías entre $2.2\,\si{eV}$ y $4.2\,\si{eV}$ se tiene una distribución con un máximo en el medio del intervalo y el parámetro $\alpha$ puede tender a infinito. Por último, para energías mucho mayores a la energía del gap, el parámetro $\alpha = 1$ y la distribución es uniforme. Estos casos se resumen en el gráfico de la Figura \ref{fig:BetaDist}.
\begin{figure}[h]
% Este gráfico se hace con el script que está acá: /home/igna/Escritorio/Tesis2021/Figs/Figuras_Apendice_Simulaciones/pys_para_plots DistBetaFig.py
    \centering
        \includegraphics[scale=0.5]{Figs/BetaDistFig.pdf}
    \caption{Distribución Beta para diferentes valores del parámetro $\alpha$. Las curvas de trazo punteado son para $\alpha < 1$, la recta punteada es para $\alpha=1$ y las curvas de trazo continuo son para $\alpha>1$.}
    \label{fig:BetaDist}
\end{figure}

El mecanismo de cascada por el cual se producen las ionizaciones consiste en que para una dada energía inicial $E_{R}$, una fracción de esa energía se utiliza para generar un par electrón-hueco y la energía restante vuelve a fraccionarse para generar otros pares electrón-hueco. Estos pares generados, a su vez, utilizan fracciones de esa energía que les fue entregada para generar otros pares, en un proceso que se repite hasta que la energía disponible para repartir en cada rama de la cascada es menor a la energía del gap del Silicio y ya no es suficiente para generar más pares. Durante todo este proceso existe una probabilidad no nula de que parte de la energía se pierda por emisión fonones en la red.

Se define una probabilidad $P_{eh}$ para la cual se produce ionización y una probabilidad $1 - P_{eh}$ para la cual se produce emisión de fonones. Esta probabilidad depende de la energía inicial, al igual que el parámetro $\alpha$ de la distribución Beta, y viene dada por
\begin{equation}
    P_{eh}(E_{R}) = 
    \left[
        1 + \frac{\Gamma_{ph}(E_{R})}{\Gamma_{eh}(E_{R})}
    \right]^{-1}
        \label{ec:ProbabilidadIonizacion}
\end{equation}
donde 
\begin{equation*}
    \frac{\Gamma_{ph}(E_{R})}{\Gamma_{eh}(E_{R})}
    = A\frac{105}{2\pi}\frac{(E_{R} - \hbar \omega_{0})^{1/2}}{(E_{R} - E_{g})^{7/2}}
\end{equation*}
con $A = 5.2\,\si{eV^{3}}$, que es una constante fenomenológica que contiene información microscópica del sistema y que además puede ajustarse para reproducir valores medidos experimentalmente. Por otro lado, $\Gamma_{ph}$ y $\Gamma_{eh}$ son las tasas de producción de fonones y pares electrón-hueco, respectivamente. A partir de estos modelos, se buscó simular este mecanismo de ionización mediante simulaciones Monte Carlo.
%%%%%%%%%%%%%%%%%%%%%%%%%%%%%%%%%%%%%%%%%%%%%%%%%%%%%%%%%%%%%%%%%%
%%%%%%%%%%%%%%%%%%%%%%%%%%%%%%%%%%%%%%%%%%%%%%%%%%%%%%%%%%%%%%%%%%
\section{Simulaciones básicas}
\noindent El modelo más simplificado del mecanismo de ionización puede pensarse como una serie de experimentos de Bernoulli, es decir, donde solo hay dos resultados posibles: éxito-fracaso. La probabilidad de éxito $p$, representa la de ionizar una carga y perder una cantidad de energía equivalente a la energía de creación electrón-hueco, sin tener en cuenta explícitamente la disipación de energía por excitación de fonones.

Dada una energía inicial y un número fijo $N$ de experimentos de Bernoulli, puede suceder que la energía inicial se agote completamente o no. El valor final de la energía depende muy fuertemente del valor de $N$: para una cantidad de experimentos $N$ muy grande, la probabilidad de que la energía se agote completamente tiende a $1$, mientras que para $N$ muy pequeño la probabilidad es mucho menor, pudiendo suceder que la energía no sea cero al final del experimento. No es sorprendente que al simular con este modelo se recupere un factor de Fano que tiende a $1$, en el caso en el que el valor de $N$ es grande pero no suficiente para agotar la energía de la partícula incidente. Esto es debido a que la realización de experimentos consecutivos de Bernoulli conforman una variable aleatoria de distribución binomial, la cual tiende a una distribución de Poisson si además $p$ es pequeño. 
\begin{figure}[h]
%Para hacer este gráfico hay que correr el script que está en esta carpeta /home/igna/Escritorio/Tesis2021/Figs/Figuras_Apendice_Simulaciones/pys_para_plots y se llama Orden0_simu_SI_atraviesa.py con los datos de esta carpeta /home/igna/Escritorio/Tesis2021/Figs/Figuras_Apendice_Simulaciones/txts_para_plots y se llama orden0_simu_SI_atraviesa.txt
    \centering
    \includegraphics[scale=0.5]{Figs/Orden0_fano1.pdf}
    \caption{Histograma, con los correspondientes ajustes, gaussiano y poissoniano, de los resultados de $2\times 10^3$ experimentos, cada uno de los cuales consiste de $N = 5000$ experimentos de Bernoulli. En este caso la energía final de la partícula no es cero, es decir, la partícula incidente logra atravesar el material. Puede verse como el ajuste poissoniano representa muy bien al histograma de carga.}
    \label{fig:SimulacionOrden0Fano1}
\end{figure}
Esto puede verse en la Figura \ref{fig:SimulacionOrden0Fano1}, donde puede observarse como una distribución poissoniana describe correctamente el histograma. El histograma corresponde a la distribución de carga obtenida mediante una simulación que parte de una energía inicial de $677\,\si{eV}$ con un $N = 5000$, que es la cantidad de veces que se repite el experimento de Bernoulli de ionizar o no con probabilidad $p=0.01$ y el factor de Fano resultó $F = 0.9790 \pm 0.0104$ mientras que el valor medio de eventos ionizados fue $\mu = 49.71 \pm 0.06$. A su vez, este experimento se repitió $20000$ veces para tener una buena cantidad de estadística para formar el histograma.

En oposición a lo previamente descripto, para el caso en que $N$ es tal que la gran mayoría de las veces la energía se agota completamente, el factor de Fano se vuelve menor a $1$, como puede verse en la Figura \ref{fig:SimulacionOrden0Fano0}. En este caso, nuevamente se parte de una energía inicial de $677\,\si{eV}$ pero esta vez con un $N = 30000$, cantidad suficiente para agotar completamente la energía y $p=0.01$ nuevamente. Se repitió $20000$ veces el experimento para obtener una buena cantidad de estadística para formar el histograma. En este caso el factor de Fano fue de $F = 0.2952 \pm 0.0025$ y el valor medio de carga ionizada $\mu = 200.97 \pm 0.06$.
\begin{figure}[h]
%Para hacer este gráfico hay que correr el script que está en esta carpeta /home/igna/Escritorio/Tesis2021/Figs/Figuras_Apendice_Simulaciones/pys_para_plots y se llama Orden0_simu_NO_atraviesa.py con los datos de esta carpeta /home/igna/Escritorio/Tesis2021/Figs/Figuras_Apendice_Simulaciones/txts_para_plots y se llama orden0_simu_NO_atraviesa.txt
    \centering
    \includegraphics[scale=0.5]{Figs/Orden0_fano0.pdf}
    \caption{Histograma, con los correspondientes ajustes, gaussiano y poissoniano, de los resultados de $2\times 10^3$ experimentos, cada uno de los cuales consiste de $N = 30000$ experimentos de Bernoulli. En este caso la energía final de la partícula es casi siempre cero, es decir, la partícula incidente pierde toda su energía dentro del material. Es en este caso en el que el factor de Fano es menor a la unidad.}
    \label{fig:SimulacionOrden0Fano0}
\end{figure}

Si bien este modelo de juguete es una simplificación del proceso de dispersión real, los resultados soportan la hipótesis de que el factor de Fano es menor a $1$ debido, en parte, a que la partícula incidente deposita toda su energía en el material.
%%%%%%%%%%%%%%%%%%%%%%%%%%%%%%%%%%%%%%%%%%%%%%%%%%%%%%%%%%%%%%%%%%
%%%%%%%%%%%%%%%%%%%%%%%%%%%%%%%%%%%%%%%%%%%%%%%%%%%%%%%%%%%%%%%%%%
\section{Simulación de ionización en cascada}
\noindent Utilizando ideas tomadas de los trabajos anteriormente citados, se modificaron las simulaciones Monte Carlo con la intención de reproducir el mecanismo de creación de pares electrón-hueco por ionización en cascada. Para este caso se tuvo en cuenta la posibilidad de disipación de energía por emisión de fonones, a una energía fija de $\hbar\omega_{0} = 0.063\,\si{eV}$.

El resultado de la simulación es simplemente el número de pares 
ionizados a partir de la energía inicial $E_{R}$. De esta puede verse la distribución de la cantidad de pares generados y además calcular tanto su varianza como su esperanza, para así obtener el factor de Fano y la energía de creación electrón-hueco.
%Otro elemento a tener en cuenta en la simulación es la conservación de la energía durante el proceso de creación de pares. Puede considerarse que la energía transferida al ionizar es utilizada totalmente para ionizar otros pares o puede considerarse que siempre que se dé una ionización habrá una pequeña parte de energía que se pierde y no puede ser utilizada, es decir, que no se conserva la energía. 
Cabe destacar que en este tipo de simulación, dada su implementación particular, solo se puede considerar el caso en que la partícula disipa toda su energía en el interior del material, de modo que se esperan valores para el factor de Fano menores a la unidad.

Se realizaron las simulaciones partiendo de una energía inicial $E_{r} = 677\,\si{eV}$ y $E_{R} = 1500\,\si{eV}$, correspondiente a los rayos $X$ de flúor y del aluminio respectivamente, que son el principal objeto de estudio de este trabajo. Los valores de los parámetros fueron extraídos de la bibliografía\cite{Alig, Ramanathan} y son $A = 5.2\,\si{eV}^{3}$, la energía del gap $E_{g} = 1.1\,\si{eV}$, la energía de creación electrón-hueco promedio $\varepsilon_{eh} = 3.75\,\si{eV}$ y la energía perdida cada vez que se emiten fonones $\hbar \omega = 0.063\,\si{eV}$. La cantidad de repeticiones del experimento es un parámetro configurable y para las simulaciones realizadas no se utilizaron menos de $5000$. %El resto de los parámetros son configurables y se variaron para ver los diferentes resultados de la simulación, como ser la pérdida de energía al ionizar $E_{loss}$ y la cantidad de repeticiones del experimento. La simulaciones se efectuaron con no menos de $5000$ repeticiones y con diferentes valores de $E_{loss}$.

Por otro lado, con el fin de observar como afecta el número de repeticiones del experimento a la dispersión de los valores del factor de Fano, se realizó en primera instancia un barrido en la cantidad de repeticiones, partiendo desde $100$ hasta $100900$ repeticiones. En la Figura \ref{fig:FanoConvergencia} se observa la convergencia de los valores del factor de Fano a medida que aumenta el número de experimentos, para el caso de la energía de los rayos $X$ del flúor, $E_{R} = 677\,\si{eV}$.
\begin{figure}[h]
%Este gráfico se puede hacer con el script GrafFanoConvergencia.py que esta en este directorio /home/igna/Escritorio/Tesis2021/Figs/Figuras_Apendice_Simulaciones/pys_para_plots
    \centering
    \includegraphics[scale=0.5]{Figs/FanoConvergencia.pdf}
    \caption{Valores del factor de Fano para distintas cantidades de repeticiones del experimento, partiendo desde $100$ repeticiones hasta 100900 repeticiones, utilizando $E_{R} = 677\,\si{eV}$.}
    \label{fig:FanoConvergencia}
\end{figure}
Entre $10^{2}$ y $10^{4}$ repeticiones del experimento se tiene que el factor de Fano se encuentra entre $0.096$ y $0.110$ aproximadamente, mientras que entre $10^{4}$ y $10^{5}$ cantidad de repeticiones los valores para el factor de Fano están acotados entre $\sim 0.100$ y $\sim 0.104$ aproximadamente. Se nota claramente la convergencia de los valores con el aumento de repeticiones y que para $10^{4}$ la variabilidad del factor de Fano es como mucho del $6\%$.

%Se simuló el proceso de cascada con con diferente cantidad de repeticiones, partiendo desde $5000$ hasta incluso $100000$ para asegurar la robustez de la estadística

%Además, con el fin de caracterizar las dependencias entre parámetros en esta implementación particular, se realizaron barridos sobre la pérdida de energía $E_{loss}$ y así conocer la dependencia de los resultados respecto de este, partiendo desde $0\,\si{eV}$ de pérdida de energía hasta $7\,\si{eV}$ de pérdida de energía por cada ionización (equivale a perder casi $4$ veces la energía del gap del silicio). Con esto se quiso observar el comportamiento de esta simulación en este rango de valores, por más que la pérdida de energía final sea demasiado elevada para poder considerarse en el fenómeno físico real. 

%En cada uno de los barridos se obtuvo el factor de Fano, el valor medio de carga ionizada y la energía de creación electrón hueco, como puede verse en las figuras \ref{fig:FanoVsEloss}, \ref{fig:ElossVsMu} y \ref{fig:CreacionHuecoVsEloss}.
%\begin{figure}%[h]
%a) Esta figura se puede hacer con los datos de: fano_Eloss_mu_vec.txt que está en el directorio /home/igna/Escritorio/Tesis2021/Figs/Figuras_Apendice_Simulaciones/txts_para_plots usando el .py Barridos_mu_Eloss_fano.py que está en /home/igna/Escritorio/Tesis2021/Figs/Figuras_Apendice_Simulaciones/pys_para_plots
%    \centering
%     \includegraphics[scale=0.5]{Figs/Fano_vs_Eloss_5ktrials_0-7Eloss.pdf}
%     \caption{\footnotesize{Curva del factor de Fano en función de la energía pérdida por cada ionización. El cambio abrupto en la curva ocurre para $E_{loss} = 3.75\,\si{eV}$, coincidente con la energía media de creación electrón-hueco.}}
%     \label{fig:FanoVsEloss}
% \end{figure}
%La curva del factor de Fano en función de la energía $E_{loss}$ (y al igual que la energía de creación electrón-hueco y el valor medio de carga ionizada) presenta un cambio de régimen abrupto cuando se cruza el umbral $E_{loss} = 3.75\,\si{eV}$. En la Figura \ref{fig:FanoVsEloss} puede verse claramente este cambio de régimen. También se observa que cuando hay conservación de la energía, es decir, $E_{loss} = 0$, es cuando se obtiene un factor de Fano más semejante al observado experimentalmente, que está cerca de $0.1$. A medida que aumenta la pérdida de energía, el factor de Fano comienza a decrecer hasta que se alcanza los $3.75\,\si{eV}$ de pérdida de energía, donde se observa el cambio brusco en la curva, y se observa un aumento pronunciado de la pendiente, semejante a un punto crítico.

%De la misma forma, el valor medio $\mu$ de la carga ionizada tiene un cambio de concavidad en la curva a medida que aumenta la cantidad de energía perdida por cada ionización, como se ve en la Figura \ref{fig:ElossVsMu}.
%\begin{figure}%[h]
% b) Esta figura se puede hacer con los datos de: fano_Eloss_mu_vec.txt que está en el directorio /home/igna/Escritorio/Tesis2021/Figs/Figuras_Apendice_Simulaciones/txts_para_plots usando el .py Barridos_mu_Eloss_fano.py que está en /home/igna/Escritorio/Tesis2021/Figs/Figuras_Apendice_Simulaciones/pys_para_plots
%     \centering
%     \includegraphics[scale=0.5]{Figs/ELoss_vs_mu_5ktrials_0-7Eloss.pdf}
%     \caption{\footnotesize{Curva del valor medio de carga $\mu$ en función de la energía perdida por cada ionización. El cambio abrupto en la curva ocurre para $E_{loss} = 3.75\,si{eV}$, coincidente con la energía media de creación electrón-hueco.}}
%     \label{fig:ElossVsMu}
% \end{figure}
%Nuevamente, los valores más cercanos a los medidos experimentalmente son los que corresponden a los casos en los que no hay disipación de energía.

%Por último, para la energía de creación electrón hueco, calculada a partir del valor medio de carga ionizada y la energía inicial $E_{R} = 677\,\si{eV}$, usando que $\mean{\varepsilon_{\eh}} = 677\,\si{eV}/\mu$, es claro que esta tendrá el mismo cambio de régimen en $3.75\,\si{eV}$, como se ve en la Figura \ref{fig:CreacionHuecoVsEloss}.
%\begin{figure}%[h]
%a) Esta figura se puede hacer con los datos de: fano_Eloss_mu_vec.txt que está en el directorio /home/igna/Escritorio/Tesis2021/Figs/Figuras_Apendice_Simulaciones/txts_para_plots usando el .py Barridos_mu_Eloss_fano.py que está en /home/igna/Escritorio/Tesis2021/Figs/Figuras_Apendice_Simulaciones/pys_para_plots
%     \centering
%     \includegraphics[scale=0.5]{Figs/E_eh_vs_Eloss_5ktrials_0-7Eloss.pdf}
%     \caption{\footnotesize{Curva del valor medio de la energía de creación de electrón-hueco en función de la energía perdida por cada ionización. El cambio abrupto en la curva ocurre para $E_{loss} = 3.75\,si{eV}$, coincidente con la energía media de creación electrón-hueco. Esta curva se obtiene utilizando el valor medio de carga ionizada $\mu$ y la energía inicial $677\,\si{eV}$.}}
%     \label{fig:CreacionHuecoVsEloss}
% \end{figure}
%Con lo cual se observa que este Monte Carlo es muy sensible a dos parámetros muy importantes de la física real del sistema: La energía de creación electrón hueco, que es el parámetro de la simulación que determina si hay o no ionización\footnote{Como se verá en el apéndice}; y la conservación de la energía.%Se observa que cuando la energía se conserva en la simulación, se obtienen los resultados más cercanos a los observados experimentalmente y reportados en la bibliografía.
%Con estos barridos se vio la dependencia de tanto del factor de Fano, como de la energía de creación electrón-hueco y del valor medio de carga ionizada al variar el valor de la energía que se pierde con cada ionización. Se observó que estos parámetros son muy sensibles a la pérdida de energía del sistema, obteniéndose resultados con cambios de regímenes muy pronunciados, particularmente, cuando la energía perdida por ionización coincide con el valor $E_{loss} = 3.75\,\si{eV}$, que casualmente es la energía de creación electrón-hueco promedio y que se usó en la simulación como un parámetro fijo dentro del código, como se verá en el apéndice \ref{app:Implementación}.

%En cuanto a los resultados de las simulaciones del factor de Fano, se realizaron simulaciones tanto para la energía de los rayos $X$ del flúor como para la de los rayos $X$ del aluminio, de $677\,\si{eV}$ y $1500\,\si{eV}$ respectivamente.

En cuanto a los resultados de estas simulaciones, para el caso del flúor, se pueden ver los histogramas de carga en los gráficos de las figuras \ref{fig:F_fano_A5.2} y \ref{fig:F_fano_A20}. Ambos gráficos fueron realizados con $100000$ repeticiones del experimento, pero con distintos valores del parámetro $A$. En estos se muestran la distribución de carga con un ajuste gaussiano, del cual se deriva el valor de $\mu$ posteriormente utilizado para graficar una curva poissoniana. En ambos casos se observa como la distribución de carga está lejos de parecerse a una distribución de Poisson y por ello el factor de Fano se aleja de la unidad. En la Figura \ref{fig:F_fano_A5.2} se puede ver que una distribución gaussiana es una buena descripción del histograma. En este caso se utilizó $A = 5.2\,\si{eV}^{3}$, el factor de Fano corresponde a $F = 0.1018 \pm 0.0004$ y el valor medio de carga $\mu = 191.50 \pm 0.01$, un valor significativamente mayor que el valor esperado, cercano a $\mu = 181$.
\begin{figure}[h]
%Los datos para este gráfico están en /home/igna/Escritorio/Tesis2021/Figs/Figuras_Apendice_Simulaciones/txts_para_plot F_E677_A5.2_E_loss0_Trials100000.txt Para modificar el graf hay que correr el .py que están en /home/igna/Escritorio/Tesis2021/Figs/Figuras_Apendice_Simulaciones/pys_para_plots: Al_Fano_E1570_A5.2_trials100k_histcarga.py
    \centering
    \includegraphics[scale=0.5]{Figs/F_Fano_E677_A5.2_Eloss0_100ktrials.pdf}
    \caption{Distribución de carga simulada con el método de Monte Carlo, con parámetro $A = 5.2\,\si{eV}^{3}$ y $100000$ del experimento. Se observa un valor medio $\mu = 191.50 \pm 0.01$, lo cual representa un corrimiento hacia la derecha del valor esperado para el pico de los rayos $X$ del flúor, que es al rededor de $181$ electrones un valor del factor de Fano de $F = 0.1018 \pm 0.0004$.}
    \label{fig:F_fano_A5.2}
\end{figure}
\noindent Para el segundo caso, en la Figura \ref{fig:F_fano_A20}, se modificó el valor del parámetro $A$ hasta que el pico coincida con el valor medio de carga esperado, y su valor fue $\mu = 181.12 \pm 0.02$ electrones. El valor de $A$ que cumple esa condición es $A = 20\,\si{eV}^{3}$, valor $5$ veces mayor al propuesto en la bibliografía para describir macroscópicamente las propiedades del silicio.
\begin{figure}[h]
%Los datos para este gráfico están en /home/igna/Escritorio/Tesis2021/Figs/Figuras_Apendice_Simulaciones/txts_para_plot F_E677_A20_E_loss0_Trials100000.txt Para modificar el graf hay que correr el .py que están en /home/igna/Escritorio/Tesis2021/Figs/Figuras_Apendice_Simulaciones/pys_para_plots: Al_Fano_E1570_A20_trials100k_histcarga.py
    \centering
    \includegraphics[scale=0.5]{Figs/F_Fano_E677_A20_Eloss0_100ktrials.pdf}
    \caption{Distribución de carga simulada con el método de Monte Carlo, forzando el parámetro $A$ para que el pico se encuentre en los $181.12 \pm 0.02$ electrones esperados para los $677\,\si{eV}$ de energía de los rayos X del flúor. El valor necesario para esto suceda fue $A=20\,\si{eV}^{3}$ y el factor de Fano obtenido fue $F = 0.1059 \pm 0.0004$.}
    \label{fig:F_fano_A20}
\end{figure}
El factor de Fano en este caso es de $F = 0.1059 \pm 0.0004$, que está contenido entre las bandas esperadas para la cantidad de estadística utilizada en esta simulación, al igual que para el caso anterior.

De la misma manera que para la energía del flúor, se presentan los resultados para las simulaciones para los rayos $X$ del aluminio en las Figuras \ref{fig:Al_fano_A5.2} y \ref{fig:Al_fano_A22}. Los resultados son muy similares a los del flúor. En ambos gráficos puede verse como las curvas de Poisson correspondientes al $\mu$ obtenido de las simulaciones difiere fuertemente de los histogramas, mientras que los ajustes gaussianos se encuentran en muy buen acuerdo con ellos. 
\begin{figure}[h]
%Los datos para este gráfico están en /home/igna/Escritorio/Tesis2021/Figs/Figuras_Apendice_Simulaciones Para modificar el graf hay que correr el .py que están en /home/igna/Escritorio/Tesis2021/Figs/Figuras_Apendice_Simulaciones/pys_para_plots
    \centering
    \includegraphics[scale=0.5]{Figs/Al_Fano_1500_A5.2_Eloss0_100ktrials.pdf}
    \caption{Distribución de carga simulada con el método de Monte Carlo, con parámetro $A = 5.2\,\si{eV}^{3}$ y $100000$ del experimento. Se observa un valor medio $\mu = 425.53 \pm 0.02$, lo cual representa un corrimiento hacia la derecha del valor esperado para el pico de los rayos $X$ del aluminio, que es al rededor de $400$ electrones. Se obtuvo un factor de Fano $F = 0.1017 \pm 0.0003$}
    \label{fig:Al_fano_A5.2}
\end{figure}

En la Figura \ref{fig:Al_fano_A5.2} se simuló el proceso partiendo de $E_{R} = 1500\,\si{eV}$, con $100000$ repeticiones del experimento y $A = 5.2\,\si{eV}^{3}$ como indica la bibliografía. Sin embargo, el valor medio de carga esperado presenta nuevamente un corrimiento hacia la derecha, teniéndose $\mu = 425.53 \pm 0.02$ y un factor de Fano $F = 0.1017 \pm 0.0003$.

Por último, en la Figura \ref{fig:Al_fano_A22} se repitió la simulación variando $A$ hasta obtener el un valor medio de carga semejante al esperado. Con $A = 22\,\si{eV}^{3}$ se obtuvo $\mu = 400.28 \pm 0.02$. El factor de Fano en este caso resultó $F = 0.1054 \pm 0.0004$, también contenido entre los valores esperados, como se vio en la Figura \ref{fig:FanoConvergencia}.

Tanto para el caso del flúor como para el caso del aluminio se observaron valores del factor de Fano muy semejantes entre sí, con lo cual parecería que para esta implementación de la simulación Monte Carlo, la energía inicial $E_{R}$ no resulta ser un parámetro relevante para esta magnitud.
\begin{figure}[h]
%Los datos para este gráfico están en /home/igna/Escritorio/Tesis2021/Figs/Figuras_Apendice_Simulaciones/txts_para_plots. Para modificar el graf hay que correr el .py que están en /home/igna/Escritorio/Tesis2021/Figs/Figuras_Apendice_Simulaciones/pys_para_plots
    \centering
    \includegraphics[scale=0.5]{Figs/Al_Fano_1500_A22.0_Eloss0_100ktrials.pdf}
    \caption{Distribución de carga simulada con el método de Monte Carlo, forzando el parámetro $A$ para que el pico se encuentre en los $400.28 \pm 0.02 $ electrones esperados para los $1500\,\si{eV}$ de energía de los rayos X del aluminio. El valor necesario para esto suceda fue $A=22\,\si{eV}^{3}$ y el factor de Fano obtenido $F = 0.1054 \pm 0.0004$.}
    \label{fig:Al_fano_A22}
\end{figure}
Además se obtuvieron valores cercanos a los reportados en diferentes trabajos\cite{Ryan, Alig, Janesick2, Fraser, Owens} tanto por medio de simulaciones como por mediciones.

De estas simulaciones se esperaba poder comprender mejor una de las posibles razones de por qué el factor de Fano es menor a la unidad en las mediciones experimentales. La hipótesis que sustentan las simulaciones es que el proceso pierde su carácter poissoniano por depositarse toda la energía en el interior del material y que una fracción de ella no se utiliza para ionizar sino para producir fonones.

    
    \chapter{Mediciones y configuración experimental \label{chap:ConfiguracionExperimental}}
%%%%%%%%%%%%%%%%%%%%%%%%%%%%%%%%%%%%%%%%%%%%%%%%%%%%%%%%%%%%%%%%%%
%%%%%%%%%%%%%%%%%%%%%%%%%%%%%%%%%%%%%%%%%%%%%%%%%%%%%%%%%%%%%%%%%%
\section{Configuración experimental}
\subsection{Cámara de vacío}
\noindent El detector se encontraba colocado dentro de una cámara de vacío fabricada a partir de cubo macizo de aluminio de $20\,\si{cm}$ de lado, denominado \textit{dewar}. Fue necesario que las temperaturas se encontraran por debajo de los $140\,\si{K}$ para disminuir la producción de cargas en el silicio por fluctuaciones térmicas (corrientes oscuras), como así también el fondo producido por fotones infrarrojos emitidos por las superficies interiores del \textit{dewar}, que se encontraban a temperatura ambiente.
\begin{figure}%[h]
    \centering
    \includegraphics[scale=0.5]{Figs/Frontal_Dewar_Sensor.pdf}
    \caption{\footnotesize{Esquema frontal del \textit{dewar} y el posicionamiento del sensor. El sensor se encuentra montado detrás de una placa de cobre frontal con un abertura rectangular por donde la radiación incidente alcanza al sensor. A su vez, se cubren las esquinas laterales del sensor con dos láminas de cobre para evitar la exposición de esas regiones del sensor donde es desplazada la carga para posteriormente ser medida.}}
    \label{fig:FrontalDewarYSensor}
\end{figure}
Debido a las bajas temperaturas y para evitar la condensación de humedad dentro del recinto, se hizo vacío mediante la utilización de una bomba turbo-molecular, capaz de alcanzar una presión del orden de los $10^{-5}\,\si{mbar}$.\\
\indent También fue necesaria la utilización de un calentador para controlar la temperatura del sensor y del \textit{dewar} por dos razones principales: Evitar que la temperatura de operación del sensor sea menor a $110\,\si{K}$, debido a que la eficiencia de la transferencia de carga entre píxeles del sensor se ve disminuida para temperaturas menores a esta; y para regular la velocidad de enfriamiento del sensor, dado que podría comprometerse su integridad estructural si esta superaba $1\,\si{K/s}$.\\
\subsection{Detector utilizado}
\noindent El detector utilizado fue un \textit{fully-depleted} CCD, del tipo \textit{back-iluminated}, es decir, que se expone a la radiación incidente una de sus lados y luego las cargas generadas se migran hacia el lado contrario. La zona muerta en la parte trasera del CCD estaba compuesta por tres capas: Una capa de $\sim 20\,\si{nm}$ de óxido de indio y estaño (ITO, por sus siglas en inglés: \textit{Indium Tin Oxide}), una capa de $\sim 38\,\si{nm}$ de dióxido de circonio (ZrO$_{2}$) y una última capa de $\sim 100\,\si{nm}$ de dióxido de silicio (SiO$_{2}$). El CCD estaba dividido en cuatro cuadrantes, denominados OHDU, con un amplificador en la esquina de cada cuadrante, permitiendo la lectura en simultaneo de todos ellos. Cada cuadrante consiste en $2063$ filas y $500$ columnas, y cada píxel tiene una dimensión de $15\,\si{\mu m} \times 15\,\si{\mu m}$.
%%%%%%%%%%%%%%%%%%%%%%%%%%%%%%%%%%%%%%%%%%%%%%%%%%%%%%%%%%%%%%%%%%
%%%%%%%%%%%%%%%%%%%%%%%%%%%%%%%%%%%%%%%%%%%%%%%%%%%%%%%%%%%%%%%%%%
\subsection{Fuente de \texorpdfstring{$\Am{241}$}{Am241}}
\noindent Para las mediciones estudiadas en este trabajo, se utilizó una fuente radioactiva de $\Am{241}$ electrodepositada, que emitía partículas alfa con energía de $\sim 5.6\,\si{MeV}$, una actividad de $1\,\si{\mu C}$ y un diámetro de $5\,\si{mm}$. Para obtener los rayos $X$ de fluorescencia del flúor y del aluminio, se colocó dentro del \textit{dewar} un caja de cobre enfrentada al sensor y que alberga una cinta de Teflón, material que contiene flúor, y una placa de aluminio. La fuente radioactiva se colocó debajo de esta caja de cobre, a la cual se le hizo un agujero por donde ingresaban las partículas alfa que impactarían los materiales para producir la fluorescencia, como se ve en el esquema de la figura \ref{fig:FrontalAlYF}.
\begin{figure}[h]
    \centering
    \includegraphics[scale=0.7]{Figs/CajaSensor.pdf}
    \caption{\footnotesize{Esquema frontal de la caja de cobre donde se posicionaron la cinta de teflón y la placa de aluminio. Se esquematiza el orificio por donde ingresan las partículas alfa, pero no la barrera allí colocada. El sensor se encuentra montado frente de esta, detrás de la placa de cobre del esquema de la figura \ref{fig:FrontalDewarYSensor}, de forma que los rayos de fluorescencia producidos por el aluminio y el teflón impacten sobre él.}}
    \label{fig:FrontalAlYF}
\end{figure}
Además, se colocó una pequeña barrera de cobre entre el orificio la fuente y el sensor para que las partículas alfa no llegasen directamente hacia este.

En el esquema de la figura \ref{fig:LateralDewar} puede verse un corte lateral del \textit{dewar}. Del lado derecho se encuentra la caja de cobre que contiene el material que se quiere estudiar (flúor o aluminio) y la fuente radioactiva. Del lado izquierdo puede verse la estructura que sostiene el detector y la pieza de cobre destinada a regular la temperatura del mismo.
\begin{figure}%[h]
    \centering
    \includegraphics[scale=0.7]{Figs/LateralDewar.pdf}
    \caption{\footnotesize{Esquema lateral del \textit{dewar}. Del lado izquierda se encuentra la placa de cobre con la abertura para el sensor (vista lateral del esquema \ref{fig:FrontalDewarYSensor}) el cual está en contacto con una pieza de cobre fría para mantenerlo en, por ejemplo, $123\,\si{K}$. Del lado derecho se encuentra la caja de cobre con la pieza de aluminio o de flúor, posicionada en ángulo y por encima del orificio por donde pasan las partículas alfa de la fuente radioactiva (vista lateral del esquema \ref{fig:FrontalAlYF}).}}
    \label{fig:LateralDewar}
\end{figure}
Por otro lado, como los fotones de la fuente radioactiva eran capaces de impactar en cualquier parte de la superficie de la caja de cobre, este podría producir fotones de fluorescencia. Para evitar esto se recubrió el interior de la caja de cobre con cinta \textit{Kapton}, la cual está compuesta por moléculas cuya probabilidad de emisión de fotones de fluorescencia es menor que la del cobre.
%%%%%%%%%%%%%%%%%%%%%%%%%%%%%%%%%%%%%%%%%%%%%%%%%%%%%%%%%%%%%%%%%%
%%%%%%%%%%%%%%%%%%%%%%%%%%%%%%%%%%%%%%%%%%%%%%%%%%%%%%%%%%%%%%%%%%
\section{Mediciones}
\noindent Las mediciones se realizaron exponiendo la mitad central del sensor a la radiación incidente y dejando las mitades laterales donde se encuentran los amplificadores cubiertos por láminas de cobre, como se ve en el esquema de la figura \ref{fig:FrontalDewarYSensor}, para evitar su exposición. Los cuatro cuadrantes del sensor son expuestos por un muy breve período de tiempo a la radiación, para rápidamente desplazar la carga colectada en esta región a la región del sensor cubierta por las láminas y protegerla de la emisión durante el proceso de lectura, el cual llevaba unos minutos. Este procedimiento es importante porque si no se mueve la carga rápidamente a la región cubierta, la actividad de la fuente saturaría de eventos el detector inutilizando estas mediciones.

%Es importante notar que el tiempo que se demora en desplazar la carga de la región expuesta a la región cubierta depende de la cantidad de columnas del sensor que se quieren medir. Si por ejemplo, quieren medirse $50$ columnas, la columna número $50$ abandonará la región expuesta en un tiempo $t$. Pero si lo que se quiere medir son $100$ columnas, entonces en este caso la columna número $100$ abandonará la región expuesta en un tiempo $\sim 2t$. Esto implica que la cantidad de eventos medidos es proporcional a la cantidad de columnas que se quieran medir además de que las últimas columnas tendrán cada vez más eventos.

%\noindent Debido a que la tasa de eventos de fluorescencia de aluminio y flúor producidos fue relativamente baja, las colección de eventos se realizó exponiendo el detector durante $20$ minutos antes de realizar la medición de la carga en los píxeles. 
Para la medición de la carga de los píxeles se realizaron $300$ muestreos por píxel, lo que corresponde a un tiempo de lectura de $\sim 10$ minutos por imagen. De las $500$ columnas que tiene el sensor, se utilizaron solo $50$ para las mediciones del flúor, y de las $2063$ filas se utilizaron $500$, donde las $7$ primeras son de \textit{pre-scan}, $443$ de región activa y las $50$ finales de \textit{over-scan}, con lo cual el área activa total del sensor fue de $22150$ píxeles.

Luego de la lectura, las $300$ mediciones tomadas para cada píxel fueron promediadas y los píxeles vacíos del \textit{over-scan} fueron usados para calcular y extraer la linea de base de cada fila. Este proceso es necesario para la calibración, para poder establecer el valor en ADU's correspondiente al $0$ de carga para cada una de las filas del sensor. Las regiones de \textit{pre-scan} y \textit{over-scan} son regiones del sensor que no están colectando carga junto con la región activa, pero cuando se realiza la medición de los píxeles y las cargas son desplazadas secuencialmente, estos píxeles atraviesan la región del sensor expuesta y en consecuencia tienen una probabilidad no nula, aunque muy baja, de colectar algún evento al desplazarse hacia el nodo de sensado. Esa baja probabilidad de colectar carga durante el proceso de lectura es la razón por la cual esos píxeles son utilizadas para definir la linea de base de cada fila. Al final de cada ciclo de exposición/lectura, toda la carga colectada por el CCD era removida en un rápido proceso que toma aproximadamente un segundo.

Las imágenes resultantes contienen $443 \times 50$ píxeles por cada cuadrante y la carga es medida en unidades electrónico digitales (ADU's), que posteriormente son convertidas en electrones usando la calibración absoluta del sensor.
    
    \chapter{Caracterización y corrección de sesgos \label{chap:Analisis}}

%%%%%%%%%%%%%%%%%%%%%%%%%%%%%%%%%%%%%%%%%%%%%%%%%%%%%%%%%%%%%%%%%%
\section{Procesado de las imágenes \label{sec:ProcesadoDatos}}
\noindent Los datos obtenidos durante el proceso de medición se almacenan en un tipo de imagen de formato \verb|.fits|. Cada píxel de esa imagen corresponde a la carga medida en el nodo de sensado en ADU's. Si se decidió realizar un número $N$ de muestreos por cada píxel del sensor usando el modo Skipper, entonces la imagen \verb|.fits| producida tendrá $N$ píxeles por cada píxel real del sensor. Por ejemplo, si el sensor estuviera conformado por $16$ píxeles, $4$ filas y $4$ columnas y el número de muestreos elegido fuera de $3$, la imagen resultante tendría una dimensión de $4$ filas por $12$ columnas, como se ve en la Figura \ref{fig:Skipper2root_esquema}. 
Por esta razón, resulta necesario procesar la imagen generada, promediando los $N$ muestreos por píxel y así obtener una lectura de carga con muy bajo ruido a la vez de disminuir drásticamente el tamaño y peso de la imagen.

El procesamiento de las imágenes es llevado a cabo por el programa \verb|skipper2root.exe|, el cual genera una nueva imagen de formato \verb|.fits|, ya promediada y con el tamaño correspondiente a la cantidad total de píxeles del sensor que se utilizaron en la medición. 
A su vez, es este programa el que se encarga de restar la linea de base, utilizando las columnas del \textit{over-scan}, de forma de establecer el valor nulo de carga para cada pixel vacío en las filas del sensor.
\begin{figure}[h]
    \centering
    \includegraphics[scale=0.4]{Figs/skipper2root_scheme.pdf}
    \caption{Ejemplo de la imagen resultante de medir con un sensor de $4\times4$ píxeles utilizando un muestreo de $3$ lecturas por píxel. La imagen resultante es de $4\times 12$ píxeles.}
    \label{fig:Skipper2root_esquema}
\end{figure}

%%%%%%%%%%%%%%%%%%%%%%%%%%%%%%%%%%%%%%%
%\textcolor{red}{Lo que sigue no está en el mejor lugar. La calibración absoluta es algo que nosotros hacemos después de correr skExtract, aca pareciera que es antes.}

% Estas nuevas imágenes tienen la información de la carga de los eventos medidos en cada píxel con ruido subelectrónico gracias al muestreo realizado utilizando el modo Skipper. Sin embargo, los datos en ellas aún están en unidades analógico-digitales (ADUs), con lo cual, para estudiarlas en términos de la cantidad de carga es necesario utilizar una calibración ADU-electrones, que puede obtenerse a partir del método descripto en la Sección \ref{sec:Antecedentes}. Esta calibración se obtiene ajustando un polinomio de orden $4$ de la forma
% \begin{equation*}
%     e^{-} 
%     = \alpha \mbox{ADU} 
%     + \beta \mbox{ADU}^{2}
%     + \gamma \mbox{ADU}^{3}
%     + \delta \mbox{ADU}^{4}
% \end{equation*}
% y es distinta para cada uno de los cuadrantes del sensor, ya que cada uno de ellos cuenta con un amplificador diferente.
%%%%%%%%%%%%%%%%%%%%%%%%%%%%%%%%%%%%%%%

%\textcolor{red}{-----------------------------------------------------------------}

Además, es necesario poder reconocer los conjuntos de píxeles contiguos con carga que pertenecen a un único evento. Con este fin, se utiliza otro programa, el cual hace uso de una calibración lineal para transformar de ADUs a electrones además de reconocer los conjuntos de píxeles con carga por medio de un algoritmo de clusterización. Este programa se llama \verb|skeExtract.exe| y procesa todas las imágenes que \verb|skipper2root.exe| generó para así formar un nuevo tipo de archivo, de formato \verb|.root|, con toda la información de los eventos encontrados en las imágenes. 

Estos nuevos archivos \verb|.root| tienen la información de la carga de los eventos medidos en cada píxel, con ruido subelectrónico gracias al muestreo realizado utilizando el modo Skipper, además de muchos otros parámetros de propios de los eventos, como ser la varianza en $x$, varianza en $y$, cantidad de píxeles por cluster, etc.

Algo a tener en cuenta es que la calibración utilizada por \verb|skExtract.exe| es nominal y se realiza solamente para que el archivo \verb|.root| tenga la carga en unidades de electrones, al menos de forma aproximada. Para mejorar la precisión de los análisis que se harán sobre los datos, es necesario volver a calibrar la relación ADU-electrones, y para esto es necesario realizar la calibración absoluta tal como fue descripta en la Sección \ref{sec:Antecedentes}. Esta calibración se obtiene ajustando un polinomio de orden $4$ de la forma
%Sin embargo, los datos en ellas aún están en unidades analógico-digitales (ADUs), con lo cual, para estudiarlas en términos de la cantidad de carga es necesario utilizar una calibración ADU-electrones, que puede obtenerse a partir del método descripto en la Sección \ref{sec:Antecedentes}. Esta calibración se obtiene ajustando un polinomio de orden $4$ de la forma
\begin{equation*}
    e^{-} 
    = \alpha \mbox{ADU} 
    + \beta \mbox{ADU}^{2}
    + \gamma \mbox{ADU}^{3}
    + \delta \mbox{ADU}^{4}
\end{equation*}
sobre los datos y es distinta para cada uno de los cuadrantes del sensor, ya que cada uno de ellos cuenta con un amplificador diferente. Los coeficientes que se obtienen de esta calibración, para cada cuadrante, se guardan en un archivo de formato \verb|.txt|.

Finalmente, se utilizan programas que son los encargados de tomar los archivos \verb|.root| y a partir de estos realizar los análisis estadísticos utilizando \verb|C++| y las librerías para análisis de datos científicos desarrolladas por el CERN llamadas \textit{ROOT}. Esto es, determinar el factor de Fano y la energía de creación electrón-hueco tanto para el flúor como para el aluminio, por medio de un ajuste \textit{no bineado}. Para eso se hicieron dos programas diferentes, uno para cada elemento y se llamaron \verb|Al_Fano_Unbinned_fit.C| y \verb|F_Fano_Unbinned_fit.C|. Estos programas utilizan los coeficientes obtenidos de la calibración absoluta para contar con presición la cantidad de carga de cada evento. También en ellos se implementan los cortes de calidad que filtran los eventos no deseados y dejan los que se deseen estudiar. Estos cortes de calidad son tanto de forma como de carga: 
\begin{itemize}
    \item Se filtran eventos cuyo tamaño supere cierto radio establecido;
    \item Se filtran eventos cuyas longitudes tanto en $x$ como en $y$ no se encuentren acotadas entre un valor mínimo y un valor máximo, es decir, no se quiere que sean muy largos en alguna dada dirección;
    \item Se filtran eventos cuya carga total no se encuentre comprendida entre un valor mínimo y un valor máximo, de forma de mirar eventos con carga cercana a los de los picos de interés.
\end{itemize}
Una vez que se tienen los eventos deseados, el programa realiza los histogramas de carga correspondientes que luego pueden ser usados para realizar un ajuste bineado de los datos. El ajuste bineado consiste en tomar el histograma de carga y ajustar los datos con el modelo utilizado minimizando $\chi^{2}$ para obtener $\mu$, $\sigma$ y $\beta$, donde $\mu$ es el valor medio del pico, $\sigma$ es su dispersión y $\beta$ es un parámetro que da cuenta de la colección parcial de carga, como se verá en el Capítulo \ref{chap:ModeloPCC}. Por otro lado, el ajuste no bineado consiste en tomar los parámetros resultantes del ajuste bineado, inyectarlos en el modelo con el que se desea ajustar y calcular la verosimilitud para todo el conjunto de datos. En este caso, se fija el valor de $\beta$ y se dejan libres $\mu$ y $\sigma$ que son variados hasta que se maximiza la verosimilitud. Los parámetros que maximicen la verosimilitud serán los parámetros óptimos para el ajuste. La necesidad de utilizar un ajuste bineado previamente para calcular los parámetros surge de que favorece la velocidad de convergencia del ajuste no bineado.

A partir del análisis de las imágenes previamente descripto se obtienen las magnitudes de interés: el factor de Fano, la energía de creación electrón hueco y el espesor de la zona de colección parcial de carga.

%%%%%%%%%%%%%%%%%%%%%%%%%%%%%%%%%%%%%%%%%%%%%%%%%%%%%%%%%%%%%%%%%%
\section{Impacto del corte propuesto}
\noindent En este trabajo se propone analizar un conjunto de imágenes existente, eliminando los eventos de un electrón presentes en ellas. Se trabaja sobre la hipótesis de que esto deberá incrementar el número de eventos totales. Se espera entonces que esta mejora en la estadística redunde en mayor precisión en la determinación de las cantidades de interés. Sin embargo, aplicar este corte trae aparejado un sesgo en el conteo de carga de cada evento que debe corregirse.

Es de interés entonces cuantificar el aumento en la estadística al realizar el corte mencionado y así motivar el análisis en busca de una mejora en el cálculo de las incertezas de las cantidades de interés.

El parámetro clave en este caso se llama \verb|EPIX| y es un valor umbral que define a partir de qué cantidad de carga en un píxel se cuenta o no como un píxel vacío. El mismo forma parte del programa de reconocimiento de clusters (\verb|skeExtract.exe|). Por ejemplo, para \verb|EPIX=0.5|, todos los píxeles con carga menor o igual a $0.5$ se cuentan como píxeles vacíos, y los que tengan carga mayor a $0.5$ serán contabilizados normalmente, para \verb|EPIX=1.5|, todos los píxeles con carga menor o igual a $1.5$ se cuentan como píxeles vacíos y los píxeles con carga mayor a $1.5$ se cuentan normalmente. En la Figura \ref{fig:HistogramaEPIX} puede verse un esquema de un histograma de los niveles de carga y cómo se sitúa el umbral \verb|EPIX| para contar la carga.
\begin{figure}[h]
%Como reproducir este gráfico: correr el script NivelesOcupacionCargaEPIX_e.py ubicado en /Escritorio/Tesis2021/Figs/pys_para_plots y buscar la imagen en /home/igna/Escritorio/Tesis2021/Figs/
    \centering
        \includegraphics[scale=0.5]{Figs/EsquemaEPIX_histocarga.pdf}
    \caption{Histograma de los datos obtenidos al iluminar el CCD con LED, correspondiente a una región con poca ocupación, donde se han marcado con dos rectas verticales los umbrales para \texttt{EPIX=0.5} y \texttt{1.5}. Depende de cuál umbral sea usado, toda la carga que se encuentra a la izquierda del umbral se considera nula.}
    \label{fig:HistogramaEPIX}
\end{figure}

En el trabajo predecesor de esta tesis\cite{TesisKevin}, los valores obtenidos para el factor de Fano y energía de creación electrón-hueco, fueron calculados utilizando un valor de \verb|EPIX=0.5|. Se espera que al modificar este parámetro, el número de eventos varíe y que, en particular, aumente cuando el \verb|EPIX| aumenta. 
Esto se debe a que es muy común que se tenga un evento de interés, por ejemplo un cluster de $4$ píxeles de área y con una carga total de $180$ electrones (para el caso del flúor), y alrededor de este se acumulen píxeles con fondo de, por ejemplo, $1$ o muy raramente $2$ electrones. 
En estos casos podría suceder que la conexión entre el cluster de interés y los píxeles con carga espuria se extiendan lo suficiente como para que el algoritmo reconozca un gran cluster con exceso de carga y sea desechado por el programa dado que no cumple con los cortes de calidad impuestos. 

También podría suceder que estos píxeles con carga espuria conecten $2$ clusters de interés, lo cual es un caso más extremo, dado que el algoritmo reconocería un único cluster de $\sim 360$ electrones, de forma que se perderían, no uno, sino dos eventos que podrían aportar positivamente a la estadística. Al aplicar un umbral que elimine los píxeles con carga espuria que se amontonan y/o conectan con clusters, el programa es capaz de diferenciar y contar los eventos correctamente.

En la Figura \ref{fig:ClusterPegoteado} se muestra un ejemplo de un evento de $179$ electrones para una medición con flúor, que es un evento de interés que el programa debería reconocer, y que hasta que no se eliminan los eventos de un electrón de la imagen, el programa lo identifica como un gran cluster con aproximadamente $40$ electrones más de carga y sin una forma definida (imagen central, píxeles pintados de blanco). A la derecha la imagen con el cluster individualizado y reconocido correctamente por el algoritmo al eliminar la carga excedente.
\begin{figure}[H]
%Para modificar este plot hay que ir a Escritorio/Tesis2021/Figs/pys_para_plots y modificar clusters_no_pegoteados.py
    \centering
    \includegraphics[scale=0.4]{Figs/despegoteo_clusters.pdf}
    \caption{Ejemplo del caso de un de un evento cercano a los $180$ electrones de carga, que son los eventos de interés. En la imagen de la izquierda se ve la medición sin alterar (ya convertida a unidades de carga). En la imagen del centro se ve en blanco y en un degrade muy tenue de rojos los diferentes clusters que el algoritmo logra reconocer. Lo importante de esta imagen es notar que el algoritmo reconoce como un único cluster (blanco) a un número de píxeles muy grande debido justamente a que píxeles con una única carga generan la unión entre todos ellos. Por último, la imagen de la derecha contiene el cluster de interés una vez que los eventos de un electrón son desechados del análisis, haciendo que ahora sí se contabilice correctamente.}
    \label{fig:ClusterPegoteado}
\end{figure}

Se llevó a cabo entonces el análisis de las imágenes obtenidas al exponer el CCD a los rayos $X$ del flúor, con diferentes valores de umbral: \verb|EPIX=0.5|, \verb|EPIX=1.5| y \verb|EPIX=2.5| y se comparó con los resultados obtenidos para el conteo total de eventos. 
Debe tenerse presente que estos son resultados preliminares, ya que la aplicación del umbral añade un sesgo extra al conteo de carga que posteriormente será corregido, como se esquematiza en la Tabla \ref{tab:SesgosyEpix}.
\begin{table}[h]
\centering
\begin{tabular*}{\textwidth}{c @{\extracolsep{\fill}}cccc}
\toprule
 &
  \begin{tabular}[c]{@{}c@{}}Sesgo \\ por eventos espurios\\  en la superficies\end{tabular} &
  \begin{tabular}[c]{@{}c@{}}Sesgo\\ por eventos espurios\\  en los bordes\end{tabular} &
  \begin{tabular}[c]{@{}c@{}}Sesgo\\ por eliminación de eventos \\ reales en los bordes\end{tabular} \\ \hline \hline
\verb|EPIX=0.5| &
  $+$ &
  $+$ &
  $-$ \\
\verb|EPIX=1.5| &
  $+$ &
  $-$ &
  $+$ \\
\verb|EPIX=2.5| &
  $+$ &
  $-$ &
  $+$ \\ \bottomrule
\end{tabular*}
\caption{Efectos en el sesgo de los resultado con la aplicación de diferentes umbrales. En todos los casos el sesgo por eventos espurios en el interior de los clusters no puede ser eliminado con este umbral, mientras que el sesgo de los eventos espurios en sus bordes desaparece dado que el umbral elimina los bordes de los clusters. Por otro lado la aplicación de estos umbrales introduce un nuevo sesgo al eliminar carga de los bordes que es genuina.}
\label{tab:SesgosyEpix}
\end{table}
Esto se hizo para tres de los cuatro cuadrantes del sensor y para la suma de estos, dado que el segundo cuadrante no funciona correctamente.

%En las Figuras \ref{fig:EntradasVsEpix}, \ref{fig:EnergiadeCreacionVsEpix} y \ref{fig:FanoVsEpix} puede verse la variación de estos parámetros dependiendo del valor de umbral usado. 
%El gráfico más importante en este punto es el de la Figura 
En el gráfico de la Figura \ref{fig:EntradasVsEpix} se observa un drástico aumento en la cantidad de entradas (eventos contabilizados) cuando se varía el \verb|EPIX|. %Se graficaron en cada figura las modificaciones para tres de los cuatro cuadrantes del sensor y para la suma de los cuadrantes $1$, $3$ y $4$. 
%El cuadrante $2$ no funciona correctamente y por eso sus datos no han sido utilizados.
\begin{figure}[h]
%Para hacer estas figs hay que ir a /home/igna/Escritorio/Tesis2021/Figs/pys_para_plots y correr plots_entries_fano_eh.py que usa los datos que están en /home/igna/Escritorio/Tesis2021/Figs/txts_para_plots y se llaman Entries_count.txt
    \centering
    \includegraphics[scale=0.5]{Figs/Entradas_vs_Epix.pdf}
    \caption{Gráfico de barras para las diferentes cantidad de entradas contabilizadas por el programa, tanto para valores diferentes de EPIX como para los diferentes cuadrantes del sensor. OHDUT hace referencia a la suma de las entradas del resto de los cuadrantes funcionales ($1$, $3$ y $4$). Se observa un aumento de más del doble en la cantidad en la cantidad de entradas para el primer cuadrante, y un aumento importante pero menos pronunciado para el resto de los cuadrantes.}
    \label{fig:EntradasVsEpix}
\end{figure}
Se puede ver como los cuadrantes $1$, $3$ y $4$ tienen un cambio pronunciado en la cantidad de entradas al pasar de \verb|EPIX=0.5| a \verb|EPIX=1.5| como se esperaba, mientras que al pasar de \verb|EPIX=1.5| a \verb|EPIX=2.5| el aumento es mucho menos pronunciado. 
Particularmente, es el primer cuadrante el que registra el mayor incremento en la cantidad de entradas en relación a los otros. 

En la Tabla \ref{tab:EntriesVsEpix} se presentan los valores precisos del cambio en el número de entradas para cada cuadrante y para cada valor de \verb|EPIX|. 
El primer cuadrante pasa de tener $760$ entradas para \verb|EPIX=0.5| a tener $2272$ para un \verb|EPIX=1.5|, casi el triple, es un aumento de $\sim 198\%$. En cambio, los cuadrantes $3$ y $4$ pasan de tener $1571$ y $1503$ entradas a $2229$ y $2320$, un aumento muy similar y en torno al $\sim40\%$ y $\sim50\%$ respectivamente.
\begin{table}[h]
\centering
\begin{tabular*}{\textwidth}{c @{\extracolsep{\fill}}ccccc}%{@{}ccccc@{}}
\toprule
           & OHDU 1 & OHDU 3 & OHDU 4 & OHDU 1 + 3 + 4 \\ \hline\hline
EPIX = 0.5 & 760    & 1571   & 1503   & 3834           \\
EPIX = 1.5 & 2272   & 2229   & 2320   & 6821           \\
EPIX = 2.5 & 2399   & 2261   & 2356   & 7016           \\ \bottomrule
\end{tabular*}
\caption{Diferentes valores para las entradas, para cada uno de los cuadrantes, para los diferentes valores de EPIX utilizados.}
\label{tab:EntriesVsEpix}
\end{table}
%Efectivamente se observa un aumento en el conteo de eventos que reconoce el programa al aumentar el valor del parámetro \verb|EPIX|. También se observa que el aumento más pronunciado es desde $0.5$ a $1.5$. El aumento promedio en el conteo de eventos es de alrededor del $100\%$.

\begin{comment}
\begin{figure}[h]
%Para hacer estas figs hay que ir a /home/igna/Escritorio/Tesis2021/Figs/pys_para_plots y correr plots_entries_fano_eh.py que usa los datos que están en /home/igna/Escritorio/Tesis2021/Figs/txts_para_plots y se llaman Entries_count.txt
    \centering
    \includegraphics[scale=0.5]{Figs/EnergiaCreacion_vs_Epix.pdf}
    \caption{Diferentes valores para la energía de creación electrón-hueco en función del EPIX y del cuadrante del sensor utilizado. OHDU hace referencia a al promedio del resto de los cuadrantes funcionales ($1$, $3$ y $4$). Se observa que en todos los casos hay un aumento de la energía de creación electrón hueco cuando aumenta el EPIX.}
    \label{fig:EnergiadeCreacionVsEpix}
\end{figure}
En cuanto al gráfico de la Figura \ref{fig:EnergiadeCreacionVsEpix}, se ve como el valor preliminar para la energía de creación electrón hueco aumenta en todos los casos al aumentar el umbral \verb|EPIX|. Esto puede deberse al efecto que genera aplicar un umbral y disminuir la carga en los clusters contabilizados, que a su vez son más. Puede suceder que la cantidad de carga en los clusters sufra una disminución respecto del valor real y por esto la energía de creación electrón-hueco aumente al aumentar el \verb|EPIX|: a un mismo valor de energía, una menor carga ionizada implica una mayor energía de creación electrón-hueco.
Finalmente, el caso más irregular corresponde al gráfico de la Figura \ref{fig:FanoVsEpix}, donde cada cuadrante y para cada valor de \verb|EPIX| el factor de Fano dio valores diferentes y no puede definirse una tendencia a partir de estos resultados. 
De todas formas es importante recordar que el factor de Fano y la energía de creación electrón hueco obtenidas de estos análisis presentan un sesgo debido a la aplicación del umbral, además del proveniente de eventos de fondo, que debe ser corregido.
\begin{figure}[h]
%Para hacer estas figs hay que ir a /home/igna/Escritorio/Tesis2021/Figs/pys_para_plots y correr plots_entries_fano_eh.py que usa los datos que están en /home/igna/Escritorio/Tesis2021/Figs/txts_para_plots y se llaman Entries_count.txt
    \centering
    \includegraphics[scale=0.5]{Figs/Fano_vs_Epix.pdf}
    \caption{Variación del factor de Fano en función del EPIX para diferentes cuadrantes del sensor. No se observa un patrón que se repita.}
    \label{fig:FanoVsEpix}
\end{figure}
Entonces, del análisis preliminar modificando el umbral de detección de carga por píxel, se puede observar el aumento esperado en la estadística para los eventos de interés de este trabajo. Como se observa que el cambio de \verb|EPIX = 1.5| a \verb|EPIX = 2.5|, no es tan significativo como el que se obtiene al pasar de \verb|EPIX = 0.5| a \verb|EPIX = 1.5|, será este último el que se utilizará en el análisis. Además, este corregir este corte resultará más sencillo que si se eliminaran también los píxeles con dos electrones.

\end{comment}
Habiendo tomado este rumbo, es necesario poder remover el sesgo producido por la eliminación de carga en los eventos medidos al aplicar este umbral. Si bien este proceso genera un aumento en la estadística, también genera un corrimiento hacia la izquierda en los picos de interés que debe ser corregido. Además, también es necesario remover el exceso de carga que tengan los clusters de debido al fondo presente en las imágenes y que en este caso genera un corrimiento a la derecha de estos picos. En adelante, la idea es intentar comprender este fondo en las imágenes y con ello poder aplicar correcciones a los valores de carga de los clusters, una vez aplicado el umbral y así mejorar la incerteza de los resultados.

%%%%%%%%%%%%%%%%%%%%%%%%%%%%%%%%%%%%%%%%%%%%%%%%%%%%%%%%%%%%%%%%%%
\section{Caracterización de las imágenes}
\noindent Todo el análisis cuantitativo anteriormente descripto se realizó sin la necesidad de inspeccionar visualmente las imágenes de las cuales se extraen los datos. Simplemente se aplicaron diferentes umbrales de prueba y se contabilizó el aumento en la estadística. Sin embargo, ver las imágenes y rápidamente poder reconocer patrones, como exceso de eventos en una misma región para diferentes imágenes o cualquier característica que visualmente sea reconocible pero que al analizar los datos de forma automatizada pueda quedar ofuscada, es un factor importante a la hora del estudio de los datos. Dado que la cantidad de imágenes utilizadas en este trabajo es superior a las $900$, observar una por una es una tarea monumental e impracticable. Por esta razón fue necesario buscar maneras de poder extraer información contenida en todas las imágenes, de forma práctica y realizable, como por ejemplo, generar una imagen \textit{promedio}. Con este fin, se hizo un análisis visual, cualitativo y cuantitativo de las imágenes para comprender mejor los datos, explorar las características del sensor y de cada uno de sus cuadrantes y poder reconocer posibles deficiencias o particularidades relevantes.

Uno de los primeros factores a caracterizar tiene que ver con la carga de los píxeles que no es debida a eventos de interés. Estos pueden ser producto de corrientes oscuras (electrones que sufren excitaciones espontáneas debido a fluctuaciones térmicas del sensor), rebotes de un fotón de baja energía en las paredes de la cámara de vacío donde se encuentra el sensor, etc. No es sencillo y no existe una única manera de estimar el fondo en un sensor, por lo que se ensayaron deferentes maneras de encarar este análisis.

Lo primero que se hizo fue buscar la manera de explorar solamente los píxeles que tuvieran una única carga. Asumiendo que en la gran mayoría de los casos, los píxeles con una única carga que se encuentran aislados de otros píxeles o de clusters de interés, son eventos que forman parte del fondo del sensor, es natural empezar el análisis con estos. Una forma de caracterizar esto es tomar las imágenes y extraer todos los píxeles donde la carga sea mayor que un electrón. De este modo, se obtienen imágenes donde solo hay eventos de un electrón y todo lo demás son píxeles vacíos. En la Figura \ref{fig:ImagenFitsOriginal} se puede ver una típica imagen tomada con el sensor, para el primer cuadrante, en la que claramente pueden observarse algunos eventos muy brillantes y un intenso fondo. En la imagen \ref{fig:ImagenFits1e} en cambio puede verse la imagen resultante de extraer todos los píxeles cuya carga es mayor a un electrón.
\begin{figure}[h]
%Para hacer estas figs hay que ir a /home/igna/Escritorio/Tesis2021/Figs/pys_para_plots y correr imagenes_fits_original_y_filtrada.py
    \centering
    \includegraphics[scale=0.4]{Figs/imagen_fits_original.pdf}
    \caption{Ejemplo de imagen tomada con el sensor, en el primer cuadrante.}
    \label{fig:ImagenFitsOriginal}
\end{figure}

\begin{figure}[h]
%Para hacer estas figs hay que ir a /home/igna/Escritorio/Tesis2021/Figs/pys_para_plots y correr imagenes_fits_original_y_filtrada.py
    \centering
    \includegraphics[scale=0.4]{Figs/imagen_fits_1_e.pdf}
    \caption{Imagen resultante luego de ser extraídos los píxeles con carga mayor a $1$ electrón.}
    \label{fig:ImagenFits1e}
\end{figure}
Una vez que se extraen los píxeles de carga mayor a uno, se promedian todas las imágenes resultantes y se obtiene una una única imagen que condensa la información de todas las anteriores. En este contexto, promediar las imágenes implica tomar el arreglo matricial con los valores de carga de los píxeles que conforman las imágenes, y realizar la suma convencional de matrices para las más de $900$ imágenes. Finalmente, se divide cada elemento de la matriz suma por la cantidad total de imágenes y se obtiene una imagen donde cada píxel es el promedio de carga de ese píxel para todas las imágenes. De esta forma se puede ver si existen píxeles con mayor o menor tendencia a contener este tipo de eventos.

En la Figura \ref{fig:Eventos1e} se tiene una imagen por cada cuadrante del sensor, promediados en las $\sim 900$ imágenes tomadas, donde los píxeles más brillantes son los que tienen mayor promedio de eventos, es decir, en el total de las imágenes esos píxeles son los que más veces tuvieron un electrón de carga. Esto también puede interpretarse como una imagen de la probabilidad por píxel de que haya un único electrón: Píxeles más brillantes son píxeles más propensos a tener carga.

De la Figura \ref{fig:Eventos1e} pueden destacarse algunas características:
\begin{itemize}
    \item La carga prácticamente nula (en promedio) en las regiones del pre-scan (región izquierda de $7$ columnas de píxeles de extensión) y del over-scan (región derecha de 50 columnas de píxeles de extensión), lo cual es totalmente esperable dado que estas son píxeles con muy baja probabilidad de colectar cargas durante la medición, como fue descripto en la Seccion \ref{sec:Mediciones}. Esto se ve para todos los cuadrantes menos el segundo;
    \item El primer cuadrante es en promedio más brillante que el resto, y se observa un ligero gradiente de intensidad entre las filas inferiores y superiores. Esto se repite, pero en menor medida en los demás cuadrantes pero no necesariamente se observa a simple vista. Este efecto se aprecia con mayor claridad en los gráficos de la Figura \ref{fig:GradienteProb};
    \item El segundo cuadrante capta en promedio muy poca carga. Este cuadrante del sensor no funciona correctamente;
    \item En los cuadrantes $3$ y $4$ se pueden ver columnas enteras de píxeles oscurecidas, que captaron muchísima menos carga que podrían deberse a defectos del sensor;
    %\textcolor{red}{es al revés, no están oscurecidas porque capturaron menos carga, sino porque capturaron siempre mas de un electrón y quedaron fuera del conjunto de imágenes con un electrón o vacío a partir de las cuales calculaste la imagen promedio. Es importante decirlo, esas son hot columns y debe ser eliminadas del análisis.}
    \item En todos los cuadrantes (menos el segundo), se observa un único píxel (posición $x = 2$, $y = 0$) donde el promedio de carga es mucho mayor al resto. Además, la primera columna de píxeles luego del pre-scan también tiene tendencia a captar más carga que el resto;
    \item Todos los cuadrantes tienen tendencia a tener \textit{hot píxels} en el interior de la región activa, estos son píxeles aislados que tienden a tener más carga que otros. Además pueden verse lineas verticales de \textit{hot píxeles}, llamadas \textit{hot columns} que se producen debido a que un \textit{hot pixel} está generando carga constantemente mientras estas son desplazadas verticalmente durante la lectura.
\end{itemize}
\begin{figure}[h]
%Para reproducir esta figura hay que ir al directorio /home/igna/Escritorio/Tesis2021/Figs/pys_para_plots y correr Skipper_cuadrantes_plot.py
    \centering
    \includegraphics[scale=0.4]{Figs/1ePromedio.pdf}
    \caption{Imágenes promedio para los $4$ cuadrantes del sensor. Puede verse en la escala de la derecha que los valores más altos que se obtienen rondan el $0.3$, lo cual, interpretado como una probabilidad es un $30\,\%$ de probabilidad de que en ese píxel se encuentre un evento de un electrón. En general se ve que los promedios pueden estar entre $0.1$ y $0.2$ aproximadamente. Es decir, para los cuadrantes funcionales del sensor, cada pixel tiene una probabilidad de tener un único evento que ronda entre el $10\%$ y el $20\%$.}
    \label{fig:Eventos1e}
\end{figure}
Respecto al gradiente de intensidades que se observa entre filas superiores e inferiores de la imagen, implicaría una mayor incidencia de eventos de un electrón, en promedio, en los píxeles de las filas inferiores respecto de las filas superiores. 
%\textcolor{red}{hay que aclarar que significa inferior y superior aca en términos de estar cerca o lejos del sense node. Asi como está pareciera entenderse lo contrario a lo que pasa realmente}
Esto puede observarse en los gráficos de Figura \ref{fig:GradienteProb}, donde se ve el aumento en \textit{la probabilidad} media por fila de que haya un evento de un electrón, a medida que el número de la fila aumenta. El gráfico de arriba a la izquierda corresponde al primer cuadrante del sensor, este es el cuadrante donde más evidente se hace este gradiente, además de ser muy lineal. La probabilidad promedio para la fila $0$ del sensor es $\sim 14.5\,\%$ y crece linealmente hasta $\sim 18\,\%$ para la fila $50$. 
En el gráfico de arriba a la derecha, que corresponde al segundo cuadrante, también se observa un cambio, pero solo entre las primeras 10 filas del sensor, luego la variación de la probabilidad por fila es muy pequeña y parece aproximadamente constante. Además puede verse que los valores son un orden de magnitud menor a los del primer cuadrante. Para los gráficos de abajo a la izquierda y abajo a la derecha, que corresponden a los cuadrantes $3$ y $4$, se observan también variaciones entre las primeras y las últimas filas del sensor y que parecerían tener una tendencia lineal, sin embargo, en comparación a la variación del primer cuadrante, esta es mucho menor. Por esto es difícil verlo a simple vista: la variación para el primer cuadrante es de aproximadamente del $24\,\%$ mientras que la variación de los cuadrantes $3$ y $4$ es aproximadamente del $10\,\%$.

Este gradiente se debe a que, dado que la medición de carga del sensor es secuencial por filas, las filas superiores en la imagen son la filas del sensor que más cerca se encuentran del registro horizontal y del nodo de sensado, con lo cual son las primeras a las que se les mide la carga. Por otro lado, las filas inferiores en la imagen, son las filas del sensor que más lejos se encuentran del registro horizontal y del nodo de sensado, con lo cual permanecen más tiempo expuestas a fuentes de fondo. En la imagen el nodo de sensado se encontraría en la esquina superior izquierda.
\begin{figure}[h]
%Para modificar este plot hay que ir a /home/igna/Escritorio/Tesis2021/Figs/pys_para_plots y correr gradiente_filas_sensor.py Los datos los saca de /home/igna/Escritorio/Tesis2021/Figs/txts_para_plots y del archivo OHDU1/2/3/4_gradiente_filas_sensor.tx
    \centering
    \includegraphics[scale=0.45]{Figs/Gradiente_en_filas_sensor.pdf}
    \caption{Variación de la \textit{probabilidad} promedio por filas del sensor de tener un evento de $1$ electrón, para los diferentes cuadrantes. Se ven aumentos lineales de la probabilidad para los casos de los cuadrantes $1$, $3$ y $4$ y un aumento más pronunciado en relación a los demás para el primer cuadrante.}
    \label{fig:GradienteProb}
\end{figure}

Para los análisis que se realizaron en este trabajo fueron utilizados el primer y tercer cuadrante del sensor, dado que son los cuadrantes que mejor funcionan. El segundo cuadrante es defectuoso y el cuarto cuadrante, si bien funciona, presenta muchas \textit{hot columns} y muchas columnas oscuras, por lo que se decidió omitirlo de los análisis de los histogramas de carga.

%Por otro lado, los análisis que siguen en esta sección se centraron sobre el primer cuadrante dado que es el que mejor funciona de los cuatro.

Teniendo entonces una imagen del promedio de la cantidad de eventos de un electrón por píxel, en la búsqueda por caracterizar el fondo del sensor, lo que se hizo posteriormente fue promediar todos los elementos de esta, de forma de obtener un promedio total y poder interpretarlo como una \textit{probabilidad} general de que en un píxel haya un evento de un electrón. Entonces, para el primer cuadrante y considerando solo la región activa del sensor, se obtuvo una probabilidad $p = 0.1802 \pm 0.0213$, es decir, que con esta primera manera de caracterizar el fondo, hay aproximadamente un $18\,\%$ de probabilidad de que un dado píxel de la región activa del sensor tenga un electrón.

Sin embargo, esta es una forma muy rudimentaria para intentar caracterizar el fondo, además de que no es del todo correcta. Con este camino se asume que todos los eventos de un electrón son fondo, lo cual no es correcto, de forma que la probabilidad de tener un evento de un electrón en un dado píxel, calculada de esta manera, está sobrestimada. 

Un camino más sofisticado para estimar el fondo en el sensor es explotando el hecho de que los eventos medidos en él siguen una distribución poissoniana: si se supone que todo píxel tiene igual probabilidad de tener una carga debido a fondo, que dicha probabilidad es pequeña para mediciones de corto tiempo y que el número de píxeles es muy grande ($22150$ píxeles por cuadrante), entonces es esperable que la distribución que modela estos eventos sea una poissoniana. De esta forma, si se pudiera calcular la esperanza $\mu$ de la distribución, podría saberse la probabilidad de que en un determinado píxel se encuentre un evento de un electrón o, en general, la cantidad de electrones que se desee.

%%%%%%%%%%%%%%%%%%%%%%%%%%%%%%%%%%%%%%%%%%%%%%%%%%%%%%%%%%%%%%%%%%
\section{Estimación del fondo}
\noindent Si se considera una distribución poissoniana para la variable aleatoria \textit{número de electrones de fondo por píxel}, se puede tomar el caso $p = P(k = 1 | \mu) = 0.1802 \pm 0.0213$, que es la probabilidad que se obtuvo previamente para el primer cuadrante. A partir de esta se puede despejar numéricamente el valor $\mu$ que satisface la expresión anterior y resulta ser
%De forma itera partiva \textcolor{red}{qué significa de forma iterativa?}puede hallarse el valor de $\mu$ que satisface la expresión anterior y resulta ser
\begin{equation*}
    \mu = 0.2258 \pm 0.0271
    % PARA VER EL CALCULO DE ESTO:Analisis_imagenes_probabilidades.ipynb
    % PRIMER MÉTODO: PROMEDIO SIMPLE
\end{equation*}
Si bien esta forma de cuantificar el fondo es un poco más general, dado que ahora pueden contemplarse casos más raros, como que un píxel tenga más de una carga, este método sigue teniendo el problema de la sobrestimación de la probabilidad por píxel, al seguir asumiendo que todo píxel con un electrón proviene del fondo.

Siguiendo sobre el mismo camino, todavía bajo la hipótesis de que todo evento de un electrón es debido a fondo, pero evitando el cálculo de los promedios, hay una forma de calcular la esperanza de la distribución y es notando lo siguiente: si se toman las probabilidades de que haya una sola carga y ninguna carga por píxel, es decir, se toman
\begin{equation*}
    p_{0} \equiv P(k = 0 | \mu),
    \quad
    \quad
    p_{1} \equiv P(k = 1 | \mu)
\end{equation*}
y se mira la relación entre ambas, se tiene
\begin{equation*}
    \frac{p_{1}}{p_{0}} = \frac{\mu\,e^{-\mu}}{e^{-\mu}} = \mu
\end{equation*}
y se observa que puede hallarse directamente el valor de la esperanza de la distribución. Entonces, tomando la región activa de una imagen completa, con todos sus eventos, como la de la Figura \ref{fig:ImagenFitsOriginal}, contando la cantidad de píxeles vacíos, la cantidad de píxeles con un electrón y calculando la relación entre ambas, para todas las imágenes, se puede obtener directamente una estimación para el parámetro $\mu$ de la distribución. De esto se obtuvo que el valor es:
\begin{equation*}
    \mu = 0.2311 \pm 0.0001
\end{equation*}
Los resultados de ambos métodos difieren en menos del $5\,\%$ y se solapan sus errores. 
%Si bien esta segunda forma para calcular la esperanza $\mu$ parece un poco más elegante y correcta, el problema sigue estando en los datos que se utilizan para calcularla. 
Sin embargo, el valor seguirá estando sobrestimando respecto del valor real, en tanto se siga considerando a todo píxel con un único electrón como fondo.

No hay que perder de vista que el objetivo de calcular la esperanza de la distribución es poder utilizarla para estimar cuánta carga extra hay sobre los clusters debida a fondo y cuánta carga no espuria fue removida debido al umbral aplicado. 
%
Conociendo la esperanza de la distribución de eventos de fondo y la cantidad de píxeles que ocupa un cluster, puede calcularse la cantidad esperada de carga extra que se halla en cada cluster. Teniendo estos valores, puede corregirse el sesgo introducido en el valor de la carga de cada cluster debido al corte aplicado y con eso hacer una mejor determinación del factor de Fano y la energía de creación electrón-hueco.
%
% Pero también, para poder generar una corrección completa para el conteo de cargas por cluster, hay que tener en cuenta las posibles formas en las que un cluster podría tener más o menos carga:
% \begin{itemize}
%     \item Que sobre la superficie de los clusters hayan eventos de más debido a fondo;
%     \item Dado que en este análisis se está aplicando un umbral que elimina eventos de un electrón y podría suceder entonces que a un cluster se le quiten eventos reales que se encuentran en sus bordes.
% \end{itemize}
% \textcolor{red}{-----------------------------------------------------------------}
% \textcolor{red}{Este itemizado de casos llega tarde, hay que hacerlo antes y completo para que el lector pueda seguir el texto.}
% \textcolor{red}{-----------------------------------------------------------------}
Con lo cual no solo es necesario corregir la carga por exceso en los clusters, sino también por defecto. Por ello, la esperanza que se obtuvo de calcular la relación entre eventos de un electrón y píxeles vacíos contiene tanto información de eventos espurios como información de eventos genuinos. Pero lo que se persigue es poder identificar los eventos espurios y los genuinos por separado. En ese sentido puede decirse que 
\begin{equation*}
    \mu_{T} = \mu_{bkg} + \mu_{g}
\end{equation*}
donde $\mu_{bkg}$ es la esperanza de la distribución de la variable aleatoria \textit{cantidad de eventos espurios por píxel}, mientras que $\mu_{g}$ es la esperanza de la variable aleatoria \textit{cantidad de eventos genuinos por píxel}. Hay que lograr separar ambos efectos para poder aplicar las correcciones correctamente. 
Queda claro que hasta el momento solo se calculó $\mu_{T}$, sin poder discriminar ambas contribuciones. La forma en la que se llevó a cabo la separación entre ellas se detalla a continuación.

%%%%%%%%%%%%%%%%%%%%%%%%%%%%%%%%%%%%%%%%%%%%%%%%%%%%%%%%%%%%%%%%%%
\subsection{Cálculo de las contribuciones de carga}
\noindent Hasta el momento, ambos métodos utilizados para calcular $\mu_{T}$ consistían en analizar los eventos de un electrón en toda el área activa del sensor. Sin embargo, esto traía aparejada una sobre estimación en los cálculos dado que se asumió que todo evento de un electrón era fondo, lo cual no es cierto. Por otro lado, considerando que la distribución de eventos por píxel tiene tanto contribuciones de fondo como genuinas, es necesario poder separar ambas contribuciones y no existe forma de hacerlo al estudiar el área activa del sensor sin tener en cuenta la posición de los clusters.

Para poder separar ambas contribuciones al calcular el $\mu_{T}$, se puede restringir el análisis al entorno cercano de los clusters, donde ahora por clusters se entiende todo conjunto de píxeles donde cada uno tenga como mínimo dos electrones de carga (eventualmente podría ser un único píxel). Es decir, se toma una imagen que tiene eventos de dos o más electrones, y se remueven los píxeles con un sólo electrón. 
En la imagen de la Figura \ref{fig:ImagenFits2omasElectrones} se muestra como luce una imagen luego de aplicar este corte.
\begin{figure}[h]
%Para modificar este plot hay que ir a /home/igna/Escritorio/Tesis2021/Figs/pys_para_plots y correr imagen_fit_2_o_mas_e.py
    \centering
    \includegraphics[scale=0.4]{Figs/imagen_fits_2_o_mas.pdf}
    \caption{Imagen de ejemplo en la que solo hay píxeles que tengan dos o más electrones.}
    \label{fig:ImagenFits2omasElectrones}
\end{figure}
En el entorno cercano de los clusters, más precisamente, en los píxeles inmediatamente contiguos a los píxeles con carga, a partir de ahora  \textit{primer borde}, es la región donde se puede decir con seguridad que coexisten ambas contribuciones: fondo y eventos genuinos. En cambio, la región formada por los píxeles que se encuentran separados por un píxel entre ellos y los eventos, a partir de ahora \textit{segundo borde}, es la región donde la probabilidad de que haya eventos de un electrón que sean genuinos y que por difusión terminaron alejados de su cluster es tan baja que puede considerarse nula. 
Con lo cual, en esta región y toda región más lejana a los clusters puede considerarse que los eventos de un electrón que se encuentren solo pueden deberse a fondo.

Sabiendo que existe una región donde se encuentran ambas contribuciones juntas y otra región donde solo se puede encontrar la contribución de fondo se pueden calcular y obtener ambas contribuciones por separado.

El método utilizado consistió en tomar el primer borde de los clusters para calcular allí el valor de $\mu_{T}$. El procedimiento se basó en formar una máscara del primer borde de los clusters. Para eso se tomaron todos los eventos de dos o más electrones y se los expandió un píxel en todas las direcciones para formar la primera etapa de la máscara. Luego, se vacío el interior de esta dejando solo sus contornos, que coinciden con los píxeles contiguos a los bordes de los clusters. Luego, superponiendo la máscara sobre la imagen original (ahora con todos los eventos), se cuentan los píxeles con eventos de un electrón y los píxeles vacíos cayeron sobre la máscara. 
\begin{figure}[h]
%Para modificar este plot hay que ir a /home/igna/Escritorio/Tesis2021/Figs/pys_para_plots y correr imagen_bordes2.py
    \centering
    \includegraphics[scale=0.7]{Figs/analisis_bordes.pdf}
    \caption{Diferentes partes del proceso de análisis de los bordes de los clusters para una imagen de ejemplo. En cada figura se ve una porción de $25 \times 100$ píxeles de área. En la primera imagen (de arriba a abajo) se tienen los clusters de dos o más electrones. En la segunda imagen se representa la dilatación de los clusters aumentando en un píxel en todas las direcciones. En la tercera imagen se ve la diferencia entre las dos primeras imágenes y se la define como la máscara a utilizar. En la cuarta se ve la máscara y superpuestos todos los eventos de un electrón de esa porción del sensor. Finalmente, en la quinta se ven solo los eventos de un electrón que cayeron encima de los píxeles de máscara. Son estos eventos los que son se cuentan en todas las imágenes, junto con los píxeles vacíos de la máscara para calcular el $\mu_{T}$.}
    \label{fig:AnalisisBordes}
\end{figure}
Nuevamente, calculando la relación entre eventos de un electrón y píxeles vacíos, se obtiene para el primer cuadrante $\mu_{T} = 0.2049 \pm 0.0002$, dado que en el primer borde se encuentran las dos contribuciones. En la Figura \ref{fig:AnalisisBordes} puede verse gráficamente cada uno de los pasos que se llevó a cabo para generar la máscara y contabilizar los eventos de un electrón que se solapan con ella.

Conociendo el valor de $\mu_{T}$, resta obtener la contribución del fondo que da origen a $\mu_{bkg}$. En primer lugar se ensayó utilizar la región conformada por el segundo borde y los píxeles aún más lejanos para calcular el $\mu_{bkg}$, es decir, toda la región restante del sensor donde hay eventos espurios. Para esto, utilizando la máscara previamente obtenida, en vez de observar los eventos de un electrón de la imagen original que solapan con ella, se cuentan los eventos de un electrón y los vacíos que están fuera de ella. De calcular la relación entre ambos, como se hizo previamente, se obtiene que el valor para el valor medio de la contribución del fondo es $\mu_{bkg} = 0.1621 \pm 0.0001$. 

Sin embargo, esta forma de calcular el fondo puede mejorarse un poco más. El objetivo del cálculo de estos valores medios es poder utilizarlos para corregir la carga medida en los clusters, debido al fondo y al umbral utilizado. Es por eso que los valores de las contribuciones que se están buscando deben ser las más representativas para los sesgos de estos eventos. Utilizar eventos de un electrón que se encuentran lejos de los eventos de interés para calcular estas correcciones no sería del todo correcto. Con lo cual, para calcular el valor de $\mu_{bkg}$ más representativo a los clusters, se optó por mirar únicamente el segundo borde y no todo el resto del sensor.

El procedimiento es el mismo que para el primer borde, pero ahora expandiendo los clusters en dos píxeles en todas las direcciones, y quedándose únicamente con el segundo borde, donde no hay píxeles con carga genuina y donde se tienen los eventos de fondo más representativos para los clusters. Este proceso puede verse en la imagen \ref{fig:AnalisisBordesx2}. Nuevamente, de la relación entre los eventos de un electrón y los píxeles vacíos que se solapan con la máscara, se obtiene para el primer cuadrante $\mu_{bkg} = 0.1902 \pm 0.002$. Finalmente, teniendo el valor de $\mu_{T}$ y el valor de $\mu_{bkg}$ queda completamente determinado el valor de $\mu_{g}$.
\begin{figure}[h]
%Para modificar este plot hay que ir a /home/igna/Escritorio/Tesis2021/Figs/pys_para_plots y correr imagen_bordesx2.py
    \centering
    \includegraphics[scale=0.7]{Figs/analisis_bordesx2.pdf}
    \caption{Análoga a la Figura \ref{fig:AnalisisBordes}, pero para el caso de $2$ dilataciones, de forma de generar una máscara en el segundo borde. Los pasos son los mismos antes descriptos. De este proceso se halla la esperanza $\mu_{bkg}$.}
    \label{fig:AnalisisBordesx2}
\end{figure}
De realizar estos análisis se obtuvieron los valores para las esperanzas de ambas contribuciones, calculadas sobre el conjunto de más de $900$ imágenes provenientes de mediciones de los rayos $X$ del flúor y para el primer cuadrante del sensor, que resultaron ser:
\begin{equation*}
    \mu_{T} = 0.2040 \pm 0.0002
\end{equation*}
y el valor de la esperanza para los eventos espurios resultó 
\begin{equation*}
    \mu_{bkg} = 0.1961 \pm 0.0002
\end{equation*}
con lo cual, la esperanza para los eventos genuinos es 
\begin{equation*}
    \mu_{g} = 0.0079 \pm 0.0003   
\end{equation*}
El mismo análisis puede repetirse para los demás cuadrantes. En este caso se decidió repetirlo para el tercer cuadrante que fue el otro cuadrante que se utilizó en los resultados de este trabajo, obteniéndose:
\begin{equation*}
    \mu_{T} = 0.1357 \pm 0.0003
\end{equation*}
\begin{equation*}
    \mu_{bkg} = 0.1186 \pm 0.0002
\end{equation*}
\begin{equation*}
    \mu_{g} = 0.0172 \pm 0.0004   
\end{equation*}

%%%%%%%%%%%%%%%%%%%%%%%%%%%%%%%%%%%%%%%%%%%%%%%%%%%%%%%%%%%%%%%%%%
\subsection{Corrección al sesgo en el conteo de carga}
\noindent El punto del análisis anterior era generar las herramientas para corregir el conteo de carga que hace el programa de reconstrucción de eventos luego de aplicar el umbral que elimina todos los eventos menores a dos electrones

Esta corrección se llevó a cabo modificando el código del programa que es usado por \textit{ROOT} para calcular el factor de Fano, la energía de creación electrón-hueco, y otras variables, por medio del ajuste no bineado de los espectros de carga. Los espectros son reconstruidos utilizando la información de los clusters que está contenida en el archivo \verb|.root| generado por \verb|skExtract.exe| al procesar las imágenes, como se describió en la Sección \ref{sec:ProcesadoDatos}. Al contar la carga de estos clusters y conociendo el área de los mismos (cantidad de píxeles que los conforman), se agrega y se quita carga en función de los valores hallados en la sección anterior.

%Dado que la cantidad de carga por píxel sigue una distribución poissoniana, de esperanza $\mu$, para cada tipo evento (espurio, genuino o total) se tiene una esperanza. 
Para calcular la cantidad de carga que se espera que tenga un cluster de $N$ píxeles en el sensor, se puede hacer uso de las propiedades de la esperanza. Sea $Y = \sum\limits_{i = 1}^{N} X_{i}$, donde $X_{i}$ son distintas realizaciones de la variable aleatoria con distribución poissoniana y $N$ es el número de píxeles del cluster, entonces la esperanza de la nueva variable aleatoria $Y$ se calcula como
\begin{equation*}
     E(Y) = 
     E
     \left(
         \sum\limits_{i=1}^{N} X_{i}
     \right)
     = \sum\limits_{i=1}^{N}E(X_{i})
     = \sum\limits_{i=1}^{N}\mu_{i}
\end{equation*}
pero como $X_{i}$ son distintas realizaciones de la misma variable aleatoria, entonces tienen todas la misma esperanza, es decir $\mu_{i} = \mu\ \forall\ i$, con lo cual
\begin{equation*}
    E(Y) = N\mu
\end{equation*}
es por esto que la cantidad de carga esperada para un cluster viene dada por el producto entre la esperanza de la distribución y la cantidad de píxeles del cluster.
\textcolor{red}{Pero este procedimiento no vale para el borde, solo para los electrones de fondo que cayeron sobre el cluster. Para el borde seria necesario conocer cuántos pixeles de borde hay.} \textcolor{blue}{No habíamos quedado que estaba bien en la última reunion?}
De esta forma, sabiendo que la esperanza se puede escribir como $\mu_{T} = \mu_{bgk} + \mu_{g}$, las correcciones se pueden realizar aplicando esta misma receta a los valores de carga por cluster. Si $n_{e}$ es la cantidad de carga medida en un dado cluster, la corrección de este valor de carga será $n_{c}$ y viene dado por
\begin{equation*}
    n_{c} = n_{e} + N(\mu_{g} - \mu_{bkg})
\end{equation*}
es decir, se agrega la cantidad de carga que se estima se pierde en los bordes por aplicar el umbral \verb|EPIX=1.5| y se quita la carga estimada de fondo en el interior de los clusters. En la práctica este procedimiento es simplemente agregar una línea en el código, justo después de la medición de carga de un cluster, donde se actualiza el valor de carga con la expresión anterior.

El método descripto se utilizará para corregir el sesgo introducido al cambiar \verb|EPIX| y el preexistente por el fondo. Los resultados finales se presentan en el Capítulo \ref{chap:Resultados}.
    
    \chapter{Resultados}
Sección pendiente porque faltan los resultados. Etiam accumsan non nulla porta lacinia. Curabitur auctor neque ac nulla efficitur fringilla. Phasellus euismod ante id elit faucibus, condimentum condimentum dui viverra. Etiam in nunc semper, hendrerit leo at, bibendum orci. Fusce feugiat at velit ut blandit. Maecenas consectetur condimentum elit, vel venenatis diam pharetra ut. Curabitur vel sapien vitae purus placerat rhoncus nec sit amet dui. Cras vehicula dictum dignissim. Donec placerat mauris in nisl feugiat dictum. Nam dui tortor, rhoncus sed metus at, finibus bibendum libero. Sed nec pharetra lacus, ac pretium mauris. Proin a augue commodo, imperdiet ex vitae, aliquet orci. In condimentum elementum metus in bibendum. Sed tempus augue sit amet tellus posuere posuere. In et suscipit eros, nec interdum nulla. Vestibulum blandit, lorem et viverra dapibus, diam lectus ultricies augue, ut gravida justo quam sed ligula.


    
	\chapter{Conclusiones}
faltan las conclusiones Phasellus at pharetra nisl. Morbi eu mi dolor. Maecenas sagittis, eros quis vulputate egestas, velit tellus rutrum metus, a condimentum risus nunc et lorem. Pellentesque ornare placerat accumsan. Donec pulvinar lorem augue, nec pharetra eros condimentum sed. Phasellus nec lacinia sem, at tincidunt lectus. Maecenas non blandit orci. Curabitur ullamcorper, elit id iaculis rutrum, enim leo auctor purus, a porttitor elit leo et libero. Nam pellentesque, orci vitae posuere lacinia, leo turpis iaculis lorem, sit amet fermentum nunc turpis et tortor. Maecenas mollis risus enim, at dignissim magna rhoncus et.


	
	\appendix
\chapter{Implementación del código de la simulación \label{app:Implementación}}
%\section{Implementación del código de la simulación}
\noindent La implementación de los códigos que ejecutan la simulación Monte Carlo se realizó con los lenguajes C y Python. Con C se realiza todo el trabajo de alto costo computacional, mientras que Python cumple un rol de interfaz de entrada para los parámetros de la simulación, de procesamiento de los datos obtenidos de ella y de visualización de resultados por medio de gráficos.

A grandes rasgos, en el código en C están implementadas las funciones que hacen los cálculos antes mencionados: el cálculo de la probabilidad de ionización $P_{eh}$ a partir de la expresión \eqref{ec:ProbabilidadIonizacion}, el cálculo del parámetro $\alpha = \alpha(E_{R})$ para la distribución Beta, a partir de la cual se genera una realización de la variable aleatoria de la que se puede despejar la energía transferida a un par electrón-hueco por ionización. Finalmente, por recursión, se simulan los procesos de ionización en cascada y se cuenta la cantidad final de electrones ionizados.

La parte escrita en Python se encarga de ejecutar el programa en C las veces que sean necesarias y con los parámetros iniciales de interés para obtener el resultado buscado.

Más en detalle, el programa en C consta de un total de $6$ funciones, las cuales se listan a continuación con una breve descripción de su funcionamiento
\begin{enumerate}[label=\arabic*., listparindent=1.5em]
    \item \verb|Random()|: Esta función genera realizaciones \verb|p_rand| de una variable aleatoria de distribución uniforme, entre $0$ y $1$. Se usa para generar una probabilidad de comparación en el Monte Carlo.
    \item \verb|Peh(E_r, A)|: Esta se encarga de realizar el cómputo de la probabilidad de ionización, según la ecuación \eqref{ec:ProbabilidadIonizacion}, a partir de la energía $E_{R}$, que es un argumento inicial en el programa y es ingresado desde Python. Esta probabilidad, \verb|p_eh|, se compara con \verb|p_rand| en el algoritmo de aceptación del Monte Carlo.
    \item \verb|alpha(E_r)|: Calcula el valor del parámetro $\alpha$ en base al valor del parámetro \verb|E_r|. Si bien el parámetro \verb|E_r| es un parámetro inicial que, por ejemplo, para el flúor es $677\,\si{eV}$, a medida que evoluciona el sistema, la energía se va consumiendo en ionizacionar y este parámetro cambia. Con cada actualización se calcula nuevamente el valor de $\alpha$ para la distribución Beta. El valor del parámetro se calcula entonces como
    \begin{equation*}
        \alpha =
        \left\{ \begin{array}{lcc}
             0.1 & \mbox{si} & E_{r} < E_{g}\\
             1 & \mbox{si} & E_{g} < E_{r} < 4.2\,\si{eV}\\
             0.02E_{r} + 0.95 & & \mbox{en otro caso.}
             \end{array}
        \right.
    \end{equation*}
    \item \verb|evolucionar(E_r, A, rand_beta)|: Genera la evolución del sistema mediante el algoritmo de aceptación del Monte Carlo, haciendo uso de las funciones anteriores. Implementa un bucle \verb|while|, cuya condición es que se repita el proceso mientras que la energía \verb|E_r| sea mayor que la energía del gap del silicio \verb|E_g|. Dentro del bucle se calcula la probabilidad \verb|p_eh| de ionización, el parámetro $\alpha$, que luego es usado para generar un número pseudo aleatorio con distribución Beta del cual despejar la fracción de energía transferida (\verb|E_traf|) a un par electrón hueco al ionizar y, por último, genera el número pseudo aleatorio de distribución uniforme con cual comparar la probabilidad de ionización en el Monte Carlo para ver si el se acepta el nuevo estado.\\
    \indent Una vez que se tienen estos valores, siempre y cuando se cumpla que la probabilidad de ionizacion \verb|p_eh| sea mayor que \verb|p_rand| y que al mismo tiempo la fracción de energía \verb|E_tranf| sea mayor que $3.75\,\si{eV}$ (valor medio para la energía de creación electrón hueco $\varepsilon_{\eh}$), entonces se actualiza el valor de la energía inicial \verb|E_r| restándole la fracción de energía transferida. Además, también, se guardan en una lista las energías transferidas \verb|E_r|. Notar que la cantidad de elementos de la lista será la cantidad de pares electrón-hueco generados por una rama de la cascada con energía inicial \verb|E_r|. Luego, cada elemento de la lista se transforma, para otra rama, en \verb|E_r|, generando una nueva lista. Repitiendo con todas las energías de toda la lista y todas las sublistas, se pueden contar los electrones ionizados.\\
    \indent De no cumplirse la condición del Monte Carlo, el sistema pierde energía por emisión de fonones, es decir, la energía \verb|E_r| se actualiza restándole un valor fijo de energía $\hbar \omega = 0.063\,\si{eV}$. El resultado de esta función es una lista con las energías de una sola rama de la cascada.
    \item \verb|recursion(E_r, A, rand_beta)|: Esta función cuenta de manera recursiva la cantidad de electrones ionizados durante la cascada. La recursion, en este caso, tiene la ventaja de que es muy sencilla de implementar, pero tiene como desventaja que no es tan sencillo entender por qué funciona correctamente. Es por esto que fue necesario crear un conjunto de datos de prueba para contrastar que los resultados que arroja la función son los esperados.\\
    \indent Lo primero que hace esta función es generar una lista energías, \verb|Energia|, llamando a la función \verb|evolucionar()| y luego se itera sobre todas estas, contando la cantidad total de elementos que posee. Esas energías son las que fueron usadas para ionizar un par electrón hueco, así que la cantidad de energías que alberga la lista es equivalente a la cantidad de pares generados. Notar que a \verb|recursion()| se le pasa como argumento \verb|E_r|. Esto significa que cuando dentro de \verb|recursion()| se vuelva a llamar a ella misma, pero ahora en vez de usar como argumento \verb|E_r|, se usa el primer elemento de la lista \verb|Energia|, es decir \verb|Energia[0]|, se produce una nueva lista a partir de una energía inicial menor. A esta nueva lista se le cuenta la cantidad de elementos y ese será el número de nuevas ionizaciones. Luego se repite para el elemento \verb|Energia[1]|, se genera otra lista de energías, lo mismo para \verb|Energia[2]|, etc. El proceso se repite hasta que todas las energías de la lista original se agotaron. Luego, las sublistas generadas repiten el proceso para todos sus elementos hasta que eventualmente la energía de las listas no es suficiente para seguir ionizando y el proceso termina.\\
    \indent El valor de salida de la función \verb|recursion()| es un entero y contabiliza la cantidad de elementos encontrados en la lista, es decir, la cantidad de ionizaciones. Durante el proceso de recursión se van sumando todas las cantidades de carga ionizada en cada paso y finalmente se obtiene la carga total generada durante la cascada.
    \item \verb|main()|: Esta se encarga de llevar adelante las repeticiones del experimento con el fin de obtener estadística. Además, en esta se definen los parámetros necesarios para la simulación, como ser la energía inicial \verb|E_r|, el parámetro \verb|A|, la cantidad de experimentos que se quieren realizar (\verb|trials|) y la generación de un archivo \verb|.txt| con los datos obtenidos para ser importados posteriormente con Python.
\end{enumerate}
Cabe destacar que de esta simulación el único resultado que se obtiene es la distribución de carga para una dada energía inicial $E_{r}$. Es decir, no puede conocerse ningún proceso intermedio o la \textit{historia} del proceso, solo la cantidad total de electrones generados.
	
    \addcontentsline{toc}{chapter}{Referencias}
    \bibliography{Bibliografia}% Produces the bibliography via BibTeX.
    \bibliographystyle{unsrt}% Produces the bibliography via BibTeX.

    \chapter*{Agradecimientos}
\thispagestyle{empty} 

agradecimientos

\newpage 

\ % The empty page
\thispagestyle{empty} 

    %\pagestyle{empty}
    
\end{document}
