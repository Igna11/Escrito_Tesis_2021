\chapter{Introducción y objetivos}
\noindent Como parte de este trabajo se propuso el uso de técnicas de análisis de imágenes y procesamiento de datos propias de la física de partículas experimental, lo que llevó al fortalecimiento de conocimientos de estadística y la familiarización con el trabajo científico en un tema de creciente interés. 
Algunas de las metas propuestas en este trabajo fueron:
\begin{enumerate}
    \item Análisis de las mediciones existentes con luz LED para obtener una calibración absoluta del detector a baja ocupancia y a diferentes temperaturas.
    \item Análisis de las mediciones existentes con fluorescencia de rayos X producidos por desexcitación del flúor y aluminio con el objeto de determinar la energía de creación electrón hueco a $677\,\si{eV}$ y $1486\,\si{eV}$ respectivamente. Del mismo análisis se obtendrá también el factor de Fano a dichas energías.
    \item Estudio de la dependencia de las cantidades anteriormente mencionadas con la temperatura en el rango de $123$ a $160\,K$.
    \item Estudio la influencia de otros fuentes de fotones en la construcción de \textit{clusters} y en la determinación del factor de Fano y energía de creación electrón hueco.
    \item De-convolución del efecto que la luz espuria tiene sobre el valor medio de carga y su distribución mediante el desarrollo de simulaciones Montecarlo.
\end{enumerate}
Cabe destacar que la última permitió incrementar la estadística disponible en las imágenes. Esto se debe a que, los cortes de calidad utilizados hasta el momento descartaban eventos que se superponían debido al puente generado por los electrones que induce la luz espuria, transformando así dos o más eventos reales en uno solo. Mitigar este problema generó un mayor aprovechamiento de la estadística disponible y con ello una reducción de las incertezas finales.
%%%%%%%%%%%%%%%%%%%%%%%%%%%%%%%%%%%%%%%%%%%%%%%%%%%%%%%%%%%%%%%%%%
%%%%%%%%%%%%%%%%%%%%%%%%%%%%%%%%%%%%%%%%%%%%%%%%%%%%%%%%%%%%%%%%%%
\section{Factor de Fano y energía de creación electrón-hueco}
\noindent El factor de Fano es una magnitud que mide la dispersión de una distribución de probabilidad para la carga en un detector. Se define como
\begin{equation*}
    F = \frac{\sigma^{2}}{\mu}
\end{equation*}
donde $\sigma^{2}$ es la varianza de la distribución y $\mu$ es la media o la esperanza. Para el caso particular de una distribución de Poisson, la varianza y la esperanza de la distribución coinciden, de forma que el factor de Fano equivale a $1$.\\
\indent Por otro lado, la energía de creación electrón-hueco $\varepsilon_{\eh}$ es, en valor medio, la energía necesaria para poder producir en un par electrón-hueco en el interior del detector de Silicio.\\
\indent La estimación precisa de ambas magnitudes es de vital importancia en la caracterización de este tipo de detectores, debido que a parámetros como la \textit{eficiencia cuántica} dependen fuertemente de ellos.
%%%%%%%%%%%%%%%%%%%%%%%%%%%%%%%%%%%%%%%%%%%%%%%%%%%%%%%%%%%%%%%%%%
%%%%%%%%%%%%%%%%%%%%%%%%%%%%%%%%%%%%%%%%%%%%%%%%%%%%%%%%%%%%%%%%%%
\section{CCD y Skipper CCD}
\noindent Los dispositivos CCD (Charge Coupled Devices) fueron inventados en 1969 en los Laboratoriosm Bell, por Willard Boyle y George Smith, en su búsqueda por fabricar dispositivos de memoria. Finalmente, los CCD's no cumplieron este objetivo pero sí demostraron un gran potencial como sensores de luz y partículas. Tal es así que en el año 2010 sus inventores recibieron el premio Nobel de física\footnote{cita[a]: Boyle, W. S. Nobel lecture: CCD-An extension of man's view. Rev. Mod. Phys., 82, 2305-2306, Aug. 2010}\footnote{cita[b]: Smith, G. E. Nobel lecture: The invention and early history of the CCD. Rev. Mod. Phys., 82, 2307-3212, Aug. 2010}.\\
\indent Estos dispositivos están hechos esencialmente de Silicio y sus elementos constitutivos fundamentales son los capacitores MOS (por metal-oxide-semiconductor). Estos conforman los píxeles del detector, siendo por lo general millones y ocupando casi la totalidad de la superficie del sensor. Los capacitores MOS se componen generalmente por un sustrato semiconductor dopado, sobre el cual se deposita una delgada capa de óxido y un metal de contacto por sobre esta. Este contacto metálico se encuentra a un voltaje $V_{G}$ y debajo del semiconductor se encuentra otro contacto que se encuentra a tierra. Dependiendo del valor de $V_{G}$ se obtienen distintos regímenes del MOS\footnote{citar} donde, en particular, uno de ellos genera una región de depleción cerca del óxido, el cual permite acumular carga minoritaria.\\
\indent El principio de operación de un CCD se puede dividir en $4$ etapas, que a grandes rasgos son:
\begin{itemize}
    \item Exposición del detector: El tiempo de exposición es variable y depende del tipo de medición que se desee realizar. Durante la exposición, la radiación incidente interactúa con el detector, generalmente generando pares electrón-hueco. 
    \item Colección: Los electrones son luego arrastrados por el campo eléctrico del detector presente en el \textit{bulk}\footnote{Buscar aclarar o whatever esto, borrarlo o no, veremox} hacia los pozos de potencial de los píxeles donde son colectados.
    \item Transferencia: Dado que la medición de los píxeles se realiza de forma secuencial, la carga en cada uno de ellos debe ser transferida de un píxel a otro.
    \item Medición de la carga: A medida que se desplaza la carga, esta es llevada hacia el nodo de sensado donde finalmente es medida.
\end{itemize}
Los CCD's convencionales son capaces de alcanzar ruidos de lectura del orden de los $2\,e^{-}\si{rms/pix}$, gracias a la técnica de muestreo doblemente correlacionado. Sin embargo, en aplicaciones de bajas energías, el ruido electrónico de lectura presupone una barrera al límite de energías que pueden medirse con estos sensores para mentener la precisión deseada.\\
\textcolor{red}{Figura de las mediciones del fano a baja energía}\\
\indent Los sensores \textit{Skipper} CCD's, por otro lado, permiten disminuir el ruido electrónico de lectura a niveles subelectrónicos gracias a que son capaces de medir la carga en los píxeles de forma no destructiva. Esto permite tomar tantas mediciones como sean necesarias de la carga y estimar la carga real a partir de un promedio sobre el número de mediciones tomadas. El prácticamente ausente ruido de lectura que hace posible la capacidad de medir repetidas veces y de forma no destructiva la carga en cada píxel de un sensor Skipper-CCD, tiene un fuerte impacto a energías por debajo de los $5\,\si{keV}$. En particular, por debajo de los $2\,\si{keV}$ la contribución del ruido de lectura de un CCD convencional a la determinación de estas cantidades puede superar el $30\,\%$. %Así es que en este trabajo se propone un estudio sistemático del Factor de Fano a bajas energías, entre $1486\,\si{eV}$ (rayos X del Al) y $677\,\si{eV}$ (rayos X del F).\\


Acá introducir a estos dispositivos, explicar las mejoras que implementa el skipper CCD e introducir conceptos importantes como el overscan (maybe prescan también)
%%%%%%%%%%%%%%%%%%%%%%%%%%%%%%%%%%%%%%%%%%%%%%%%%%%%%%%%%%%%%%%%%%
%%%%%%%%%%%%%%%%%%%%%%%%%%%%%%%%%%%%%%%%%%%%%%%%%%%%%%%%%%%%%%%%%%
\section{Antecedentes}

\subsection{paper}
%%%%%%%%%%%%%%%%%%%%%%%%%%%%%%%%%%%%%%%%%%%%%%%%%%%%%%%%%%%%%%%%%%
%%%%%%%%%%%%%%%%%%%%%%%%%%%%%%%%%%%%%%%%%%%%%%%%%%%%%%%%%%%%%%%%%%

\subsubsection{Calibracion y linealidad}
Un procedimiento de auto-calibración fue realizado para determinar la relación entre el número de electrones en cada pixel y la lectura del valor en ADUs. Un LED instalado en el \textit{Dewar} fue usado para problar los píxeles del CCD con electrones, producidos por fotones de $405\,\si{nm}$ de longitud de onda. Para lograr poblar los píxeles con un amplio rango de electrones, se realizaron varias mediciones aumentando el los tiempos de exposición en cada una. Se produjeron diferentes distribuciones poissonianas superpeuestas con un valor medio que se incrementaba en el numero de electrones. Estas mediciones se realizaron tomaron 300 lecturas a cada pixel. Como resultado, el ruido de lectura se redujo en un factor $\sqrt{300} \sim 17.3$, logrando un valor final de $0.2\,e^{-}$. Esto permitio distinguir entre ah mpicos consecutivos en todo el rango entre $0$ y $1900$ electrones. El valor medio en AD para cada uno de esos picos fue determinado por medio un ajuste gaussiano. De esta forma, la auto-calibración consistió en asignar para cada valor medio de ADU al número de pico correspondiente, que coincide con la cantidad de electrones del pico.\\
\indent Respecto a las no linealidades, la figura \textbf{REFERENCIA A LA FIGURA} se observan las diferencias entre el número de electrones calculados a partir de una auto-calibración lineal, respecto al número real de electrones por pixel (normalizado?). En contraste con las mediciones de CCDs convenciales, el skipper CCD permite cuantificar las no linealidades para todas las ocupancias.

\subsubsection{Mediciones}
Los rayos $X_{K}$ emitidos luego de la captura electrónica del $\prescript{55}{}{\mbox{Fe}}$ son ampliamente usados para calibración de CCDs. Sus energías, conocidas con excelente precisión, pueden verse en la tabla \ref{tab:EnergiasXk}. Para el propósito de este trabajo, se utilizó una fuente radioactiva de $\prescript{55}{}{\mbox{Fe}}$ electrodepositado, con un diametro de $\sim 5\,\si{mm}$ y una actividad de $\sim 0.1\,\si{\mu Ci}$
\begin{table}[h]
\centering
\begin{tabular}{@{}ccc@{}}
\toprule
$X_{K}$      & Energía [keV] & Intensidad relativa \\ \hline \hline
$\alpha_{2}$ & $5887.6$      & $8.5 (4)$           \\
$\alpha_{1}$   & $5898.8$      & $16.9 (8)$          \\
$\beta_{3}$  & $6490.4$      & $4.1 (11)$          \\ \bottomrule
\end{tabular}
\caption{\footnotesize{Caption}}
\label{tab:EnergiasXk}
\end{table}
Esta fuente se posicionó enfrentando la parte trasera del sensor y a unos $40\,\si{mm}$ de este.\\
\indent En una fracción importante de los decaimientos del $\prescript{55}{}{\mbox{Fe}}$, la energia es transferida a un orbital en vez de en rayos $X$. Estos electrones \textit{auger} abandonan los átomos con unos pocos $\si{eV}$ menos que los rayos $X$ debido a la energía de ionización y crean un espectro continuo de energía cuando interactuan con el CCD. Para evitar este fondo, se cubrio la fuente de $\prescript{55}{}{\mbox{Fe}}$ con $20\,\si{\mu m}$ de \textit{Mylar}, que logran frenan los electrones con esta energǵia y además tienen una muy baja probabilidad de producir dispersión por efecto compton de los rayos $X$. El ancho total de las capas de material que cubren al sensor es de $160\,\si{\mu m}$, sin embargo, se induce una probabilidad de interacción con los fotos de $5.9\,\si{keV}$ de $\sim 1.4 \%$.\\
\indent \textit{Procedimiento de adquisición de datos} para reduri el impacto de las corrientes oscuras se limitó la exposición  y el tiempo de lectura, al medir simultáneamente en los $4$ cuadrantes del CCD y restringiendo la adquisición a $50$ filas por cuadrante. Cada fila contiene $500$ píxeles ($7$ de prescan, $443$ de región activa y $50$ de overscan). Entonces, se tomaron imagenes con un área activa de $22150$ píxeles. La exposición a los rayos $X$ se realizó moviendo rápidamente las cargas en esos píxeles ($\sim 30\,\si{s}$ de exposición efectiva) de forma de conseguir una frecuencia de impacto de rayos $X$ de $\sim 4\,\si{Hz}$ en la región activa de cada imagen, resultando en $\sim 120$ eventos en promedio. Se realizaron $300$ mediciones por píxel, lo que corresponde a un tiempo de lectura de $\sim 10$ minutos por imagen. Luego de la lectura, las $300$ mediciones tomadas para cada píxel son promediadas y los píxeles vacíos del overscan  son usados para calcular y extraer la linea de base (el offset de la medición o el $0$) de cada fila. La imagen resultante contiene $443 \times 50$ píxeles por cada cuadrante y la carga medidida está en unidades electrónico digitales (ADU's), para posteriormente ser convertidas en electrones usando la auto-calibración.\\
\indent Al final de cada ciclo de exposición/lectura, toda la carga colectada por el CCD es removida en un rápido proceso que toma aproximadamente un segundo.\\
\indent Dado el relativamente alto tasa de fotones de rayos $X$ impactando en el sensor, se cubrio la mitad de cada cuadrante con una delgada capa de cobre en la región cercana a los amplificadores. De esta manera, se expuso el area que no se encuentraba cubierta por el cobre a los rayos $X$ mientras la carga era rapidamente movida al area cubierta por la delgada capa de cobre a la espera por ser medida y al resguardo de los rayos $X$.\\
\textit{Protección contra la radiación de cuerpo negro}: Para minimizar el fondo producido por fotones infrarrojos emitidos por las superficies interiores de la cámara de vacío, la cual se encuentra a temperatura ambiente, se cubrio el detector con una cámara de cobre fría. Esta cámara se encontraba en contacto térmico con la pieza de cobre fria en la cual se encuentra montado el detector y lo aisla de la radiación de cuerpo negro originada por las paredes de alrededor.\\

\subsubsection{Análisis de los datos y resultados}
\textit{Reconstrucción de eventos}: Las imágenes tomadas usando el procedimiento descripto antes contienen eventos producidospor la ionización de rayos $X$ producidos por la fuente de $\prescript{55}{}{\mbox{Fe}}$ y radiación ambiental. Una parte de una imagen tipica tomada por el CCD está en la figura \textbf{referencia}. Como el skipper CCD usado es \textit{back iluminated} por los rayos $X$ de la fuente, las interacciones resultantes ocurrieron mayormente en los primeros $30\,\si{\mu m}$ de la parte trasera del sensor. Debido a la difusión de carga durante el proceso de colección, la carga de los eventos resultantes se distribuyó en varios píxeles, siguiendo una distribución gaussiana $2D$. El número total de electrones generados por cada rayo $X$ se reconstruyó utilizando un algoritmo de clusterización, donde todos los pixeles vecinos no vacios son agrupados juntos y considerados como parte de un único evento.\\
\indent Es importante destacar que el skipper CCD permite que la determinación de la carga en clusters de diferente tamaño, introduciendo un muy pequeño error sistemático en la estimación del numero total de electrones producidos por cada evento. Con un ruido de lectura de $0.2\,e^{-}$, los límites de cada cluster pueden ser determinados con una probabilidad de fallo en la determinación tan baja como $p = 0.062$ por cada pixel exterior. Además, esto también implica qu ela medición es robusta para las ineficiencias en la transferencia de carga que pudieran propagar la carga entre píxeles vecinos. La probabilidad de que algún electron de un evento sea separado de otros electrones por uno o más píxeles vacíos es esencialmente nula para todos los fines prácticos.\\
\indent \textbf{Cortes de calidarks}: Para evitar contar eventos unidos, se filtraron aquellos con varianza relativamente grande o pequeña, tanto en $x$ como en $y$ y se contaron unicamente eventos relativamente circulares, compatibles con la forma de una gaussiana $2D$. Para filtrar eventos por producidos por efecto compton en el bulk del ccd por radiación ambiente de alta energia se establecio un corte de calidad en el tamaño de los clusters, para quedarse unicamente con los eventos producidos en los primeros $30\,\si{\mu m}$ de la parte trasera del CCD. El valor medio del tamaño de los clusters reusltantes luego de estos cortes de calidad fue de $12.4 \pm 2.7$ píxeles.\\
\indent \textit{Ruido de lectura} La probabilidad de contar un electrón más o menos (respecto del numero real de electrones) en los pixeles que constituyen los clústers es $p$. Teniendo en cuenta la distribución de tamaños de clusters,  eventual inner or external mis classification introduce un sesgo tan bajo como $0.1\,e^{-}$ y un ruido de lectura $\sigma_{RN} = 0.5\,e^{-}\si{rms/cluster}$. Sin embargo $\sigma_{RN}$ constituye la principal contribución a la incerteza de la energía de creación electrón-hueco.\\
\indent \textit{Efecto de las corrientes oscuras:} Las corrientes oscuras fueron medidas en las mismas condiciones experimentales pero sin la fuente radioactiva de $\prescript{55}{}{\mbox{Fe}}$. Se calculó como la relación entre los eventos de $1$ electrón y los píxeles vacíos, resultando en $\sim 1 \times 10^{-5}$ electrones por píxel por segundo. Entonces, teniendo en cuenta el valor medio del tamaño de los clusters, se esperan solamente $(0.04 \pm 0.01)$ electrones extra debido a corrientes oscuras durante $10$ minutos  de exposición y lectura de cada imagen. Con lo cual, el efecto de las corrientes oscuras puede ser desestimado sin introducir un sesgo significativo en los resultados.\\
\indent \textit{Eficiencia de la colección de carga:} Hay dos efectos responsables de degradar el CCE: la recombinación y las ineficiencias en la transferencia de carga. Como ya se discutió antes, el segundo efecto es insignificante cuando se utiliza un skipper CCD. La baja temperatura de operación, el silicio muy poco dopado y el alto campo eléctrico ($\sim350\,\si{V/mm}$) en el bulk del CCD, intentan prevenir la pérdida de carga por efecto de recombinación. Como resultado, en el volumen activo de un fully depleted detector CCD, CCE puede ser considerado esencialmente $1$ para todos los fines prácticos. Los CCD back iluminated en astronomía se tratan para tener una pequeña entrada de luz con muy baja reflecitividad\footnote{Acá van dos citas}\\
\indent \textit{trampas:} El CCD puede sufrir la presencia de trampas capaces de introducir irregularidades durante el proceso de lectura. Dado que estas trampas liberan cargas con una distribución exponencial en el tiempo, se espera que esas columnas tengan un número alto de eventos de un electrón (hot columns). Para asegurarse que los resultados no están sesgados por este proceso, se identificaron y enmascararon todas las hot columns. A pesar de seguir esta regla, por demás conservadora, no hubo un impacto significativo en los resultados finales.\\
\indent Más aún, como una forma de asegurarse, se realizó el cálculo del factor de Fano para $20$ intervalos de $20$ columnas cada uno. Como resultado, se observó la fluctuación esperada cercana a los valores globales sin puntos outliers. Esto también prueba que no hay un efecto significativo debido a las trampas en el serial register. Sin embargo, se compararon el número de cargas por píxel en la primera y segunda medición en el nodo de sensado, asegurando que no hayan cargas faltantes durante los skippering steps.\\
\indent \textit{Ajuste no bineado de picos:} Los picos de los rayos $X$ $K_{\alpha}$ y $K_{\beta}$ se ajustaron usando la verosimilitud de la ecuación

\begin{align*}
    \Lagr(e|\mu_{1},
            \mu_{2},
            \sigma_{1},
            \lambda_{1},
            \lambda_{2},
            \eta_{1} = \eta_{2},
            \eta_{3})
    = &
    \sum\limits_{j=1}^{3} I_{j}
    \left\{
        \eta_{j}\frac{\lambda_{1}}{2}
        \exp
            \left[
                (e-\mu_{j})\lambda_{1} + \frac{\sigma_{j}^{2}\lambda_{1}^{2}}{2}
            \right]
        \mbox{Erfc}
        \left[
            \frac{1}{\sqrt{2}}
            \left(
                \frac{e - \mu_{j}}{\sigma_{j}}
                +\sigma_{j}\lambda_{1}
            \right)
        \right] \right.
        \\
        + &
        \left.
        (1-\eta_{j})\frac{\lambda_{2}}{2}
        \exp
            \left[
                 (e - \mu_{j})\lambda_{2}
                 + \frac{\sigma_{j}^{2}\lambda_{2}^{2}}{2}
            \right]
        \mbox{Erfc}
        \left[
            \frac{1}{\sqrt{2}}
            \left(
                \frac{e - \mu_{j}}{\sigma_{j}}
                +\sigma_{j}\lambda_{2}
            \right)
        \right]
    \right\}
\end{align*}
donde $\mu_{j}$, $\sigma_{j}$ y $I_{j}$ representan el valor medio de carga, la desviación estandar del valor medio y la intensidad relativa del pico $j$ con energía $E_{j}$. $\lambda_{1}$ y $\lambda_{2}$ son parámetros de las distribuciones exponenciales en convolución con la gaussiana, donde $\eta_{j}$ define el peso relativo entre las exponenciales.\\
\indent Es el resultado de una convolución de dos exponenciales con una gaussiana para cada uno de los $3$ picos dados en la tabla \ref{tab:EnergiasXk}. Es un ajuste empírico que no está motivado por ninguna razón física de la espectroscopía de partículas alfa.\\
\indent Como la diferencia entre la energía de los picos $K_{\alpha}$ es de apenas $11.1\,\si{eV}$, se puede asumir la misma energía de creación electrón hueco y factor de Fano para ambos picos. Con lo cual, se asume $\mu_{2} = \mu_{1} \times E_{2}/E_{1}$ y $\sigma_{2} = \sigma_{1}\times \sqrt{E_{2}/E_{1}}$. Para el caso del pico $K_{\beta}$ también se asumio el mismo $F$ pero se permitio que la energía de creación electrón hueco tomara otro valor. Estas condiciones se satisfacen dejando $\mu_{3}$ como un parámetro libre y fijando $\sigma_{3} = \sigma_{1}\times \sqrt{\mu_{3}/\mu_{2}}$.\\
\indent La figura 5 presenta el ajuste no bineado por la verosimilitud de los picos de rayos $X$ para un total de $18085$ eventos luego de la selección y cortes de calidarks. Los parámetros relevantes para el ajuste y los valores de F y la energía de creación electrón hueco estan la tabla \ref{tab:ParametrosAjusteNoBineado}

\begin{table}[]
\centering
\begin{tabular}{@{}ccccccccc@{}}
\toprule
$X_{k}$ &
  $\mu$ &
  $\Delta \mu$ &
  $\sigma$ &
  $\Delta \sigma$ &
  $F$ &
  $\Delta F$ &
  $\varepsilon_{\eh}$ &
  $\Delta \varepsilon_{\eh}$ \\ \hline\hline
$\alpha_{2}$ &
  $1570.50$ &
  $0.18$ &
  $13.68$ &
  $0.12$ &
  \multirow{3}{*}{$0.119$} &
  \multirow{3}{*}{$0.002$} &
  \multirow{2}{*}{$3.749$} &
  \multirow{2}{*}{$0.001$} \\
$\alpha_{1}$ & $1573.48$ & $0.18$ & $13.69$ & $0.12$ &  &  &         &         \\
$\beta_{3}$  & $1730.50$ & $0.55$ & $14.36$ & $0.13$ &  &  & $3.751$ & $0.002$ \\ \bottomrule
\end{tabular}
\caption{tabla}
\label{tab:ParametrosAjusteNoBineado}
\end{table}
Los valores ajustados para $\mu_{1}$, $\mu_{3}$ y $\sigma_{1}$ son muy robust frente a cambios en el rango de energía considerado para el ajuste, con lo cual los valores para $\varepsilon_{\eh}$ y $F$ también lo son. Estos cambios solamente afectan a $\lambda_{1}$, $\lambda_{2}$ y $\eta_{j}$, que son esencilamente los que corrigen el ajuste en las colas izquierdas.\\
\indent \textit{Incertezas sistemáticas:} Como se mencionó antes, debido a los errores en la clasificación, un error sistemático se origina en la lectura del ruido de clusters ($\sigma_{RN} =0.5\,e^{-}$). Este error se sumó en cuadratura con $\sigma$, resultando contribuir de forma insignificante. Adicionalmente, hay una contribución de $0.010$ en el factor de Fano debido a los cortes de calidad aplicados en la energía de los eventos para restringir el dominio de ajuste y $0.005$ que se originó en la selección de la mínima y máxima varianza para aceptar un cluster. Estos ultimos dos se sumaron en cuadratura junto con la incerteza obtenida por propagación de $\Delta \sigma$ y $\Delta \mu$, que son las incertezas para $\sigma$ y $\mu$ que salen del ajuste a partir de la ecuación para el factor de Fano.

\subsubsection{Discusión}
Los resultados están en acuerdo con trabajos pioneros como Ryan et al, que casi hace 50 años reportó un valor $\varepsilon_{\eh} = 3.745 \pm 0.003\,\si{eV}$ y Alig et. al., quien hace casi $40$ años, usando montecarlo encontró $F = 0.113 \pm 0.005$. Sin embargo, en trabajos más recientes se pueden encontrar grandes discrepancias entre publicaciones, con valores de $F$ entre $0.14$ y $0.16$.\\
\indent A pesar de que inicialmente $\varepsilon_{\eh}$ y $F$ fueron tratadas como constantes del material, desde entonces, muchos autores han investigado experimentalmente y vía simulaciones Monte Carlo la dependencia de estas magnitudes con la energía y la temperatura. Como resultado, hoy en día se sabe que la energía de creación electrón-hueco decrece cuando la energía o la temperatura crecen, mientras que el factor de Fano $F$ varía en menor medida.\\
\indent Kotov et. al, usando CCD convencial a $185\,\si{K}$ ha reportado $(3.650 \pm 0.009)\,\si{eV}$ y $F = 0.128 \pm 0.001$. De acuerdo con el gradiente informado por Lowe et.al, el desacuerdo con los resultados presentados acá pueden ser solamente parcialmente explicados como consecuencia de las diferencias en la temperatura. Se observó un excelente acuerdo con los valores publicados por Lowe et. al. En su trabajo se midio la energía de creación electrón-hueco $\varepsilon_{\eh}$ como función de la temperatura y, de acuerdo con lo que reportan, $\varepsilon_{\eh} = (3.743 \pm 0.090)\,\si{eV}$ y $F = 0.118 \pm 0.004$ es lo que se debería esperar para $123\,\si{K}$.\\
\indent 


\subsection{chamuyo que escribí yo}
\noindent En trabajos previos se estudiaron las ventajas de la utilización de la novedosa tecnología \textit{skipper} CCD, para lograr medir con precisión subelectrónica en régimenes de energía donde los sensores CCD convencionales más precisos sólo podrían alcanzar resoluciones del orden de los $2$ electrones. Por primera vez fue usada para poder estimar el factor de Fano y la energía de creación electrón-hueco en el Silicio a una energía de $5.9\,\si{keV}$ y para diferentes temperaturas. Además de obtenerse las primeras estimaciones, también se trabajó sobre los desafíos que la utilización de esta incipiente tecnología representa. Por ejemplo, la calibración del detector para la transformación de las unidades analógico digitales (ADU's) a cantidad de carga, primero utilizando una calibración lineal y luego calibraciones no lineales de la forma de
\begin{equation*}
    e = ADU \times \alpha + ADU^{2} \times \beta
\end{equation*}

acá voy a hablar de la calibración absoluta que hizo kevin. Ver paper de Darío
%%%%%%%%%%%%%%%%%%%%%%%%%%%%%%%%%%%%%%%%%%%%%%%%%%%%%%%%%%%%%%%%%%
%%%%%%%%%%%%%%%%%%%%%%%%%%%%%%%%%%%%%%%%%%%%%%%%%%%%%%%%%%%%%%%%%%
\section{Motivación - buscar título}
\noindent Este trabajo se centró en la determinación del factor de Fano y de la energía de creación electrón-hueco, para energías por debajo de los $2\,\si{keV}$, utilizando mediciones preexistentes. Más precisamente, para las energías de los rayos $X$ de fluorescencia del Aluminio de $1486\,\si{eV}$ y del Flúor de $677\,\si{eV}$. Con el fin de que la determinación de estas características esté lo menos sesgada posible debido al fondo presente en las imágenes, tanto por ruido del sensor, como por eventos que no son de interés, fue necesario realizar un análisis profundo de los datos.\\
\indent El fondo presente en las imágenes consiste principalmente en píxeles cuya carga es producida por fluctuaciones térmicas en la red cristalina del Silicio del sensor (corrientes oscuras), por eventos de dispersión producidos por rebotes dentro del dispositivo de medición y que no deseados, y por eventos muy penetrantes provenientes del exterior, como por ejemplo, muones.\\
\indent El análisis consistió estudiar el efecto que produce el fondo de las imágenes sobre los eventos de interés: la aglomeración de píxeles con cargas de entre $1$ y $2$ electrones al rededor de los clústers, y la carga extra añadida sobre ellos. En muchos casos resulta que los píxeles con fondo se aglutinan a los clusters, aumentando su tamaño y su carga, o incluso también haciendo de puente entre dos clusters vecinos. Estos son dos efectos indeseados, primero porque sesga la cantidad de carga real en un evento de interés y segundo porque los algoritmos de clusterizacion podrían ignorarlos al no cumplir con los cortes de calidad impuestos, tanto por forma como por cantidad de carga esperada.\\
\indent Es entonces que es necesario utilizar un umbral de detección que ignore estos píxeles con carga menor o igual a $2$ electrones, de forma de evitar el aglutinamiento de píxeles con fondo a los clústers de interés y así aumentar la estadística en el conteo de eventos. Pero también es necesario lograr caracterizar este fondo para corregir el sesgo introducido por el nuevo umbral de detección, el cual, también eliminará píxeles con carga genuina. Es por eso que en este trabajo se propuso un análisis de las imágenes para poder determinar el umbral más conveniente a utilizar y recuperar la mayor cantidad de estadística posible, además de un método para estimar cuántos eventos genuinos son removidos y cuántos eventos espurios hay que remover de los clústers para mejorar así las incertezas del factor de Fano y la energía de creación electrón-hueco a bajas energías.

%%%%%%%%%%%%%%%%%%%%%%%%%%%%%%%%%%%%%%%%%%%%%%%%%%%%%%%%%%%%%%%%%%
%%%%%%%%%%%%%%%%%%%%%%%%%%%%%%%%%%%%%%%%%%%%%%%%%%%%%%%%%%%%%%%%%%
\section{Organización de la tesis}
Esto lo escribo al final y es un resumen de como está armada la tesis.
%%%%%%%%%%%%%%%%%%%%%%%%%%%%%%%%%%%%%%%%%%%%%%%%%%%%%%%%%%%%%%%%%%
%%%%%%%%%%%%%%%%%%%%%%%%%%%%%%%%%%%%%%%%%%%%%%%%%%%%%%%%%%%%%%%%%%
%%%%%%%%%%%%%%%%%%%%%%%%%%%%%%%%%%%%%%%%%%%%%%%%%%%%%%%%%%%%%%%%%%

\input{SimulacionesMC}

%%%%%%%%%%%%%%%%%%%%%%%%%%%%%%%%%%%%%%%%%%%%%%%%%%%%%%%%%%%%%%%%%%
%%%%%%%%%%%%%%%%%%%%%%%%%%%%%%%%%%%%%%%%%%%%%%%%%%%%%%%%%%%%%%%%%%
%%%%%%%%%%%%%%%%%%%%%%%%%%%%%%%%%%%%%%%%%%%%%%%%%%%%%%%%%%%%%%%%%%
\chapter{Mediciones y configuración\\ experimental}
Acá iria una descripción de cómo se realizaron las mediciones, ver paper de Darío. Configuración experimental, mediciones y resultados crudos.
%%%%%%%%%%%%%%%%%%%%%%%%%%%%%%%%%%%%%%%%%%%%%%%%%%%%%%%%%%%%%%%%%%
%%%%%%%%%%%%%%%%%%%%%%%%%%%%%%%%%%%%%%%%%%%%%%%%%%%%%%%%%%%%%%%%%%
%%%%%%%%%%%%%%%%%%%%%%%%%%%%%%%%%%%%%%%%%%%%%%%%%%%%%%%%%%%%%%%%%%
\chapter{Imágenes y análisis}
\section{Análisis y factibilidad}
\noindent En este trabajo se propone realizar un análisis profundo de las mediciones existentes del sensor, aplicando un corte de calidad a las imágenes que aumenta el número de eventos detectados por el programa, para así aumentar la estadística y mejorar la resolución con la que se calculan tanto el factor de Fano como para la energía de creación electrón-hueco. %Sin embargo, aplicar este corte de calidad trae aparejado un sesgo en el conteo de carga de cada evento, tanto por defecto como por exceso, con lo cual es un efecto que debe corregirse.
En este sentido, es de gran importancia cuantificar el aumento en la estadística al modificar los parámetros usados en el programa de reconocimiento de clusters y así poder establecer la factibilidad de la mejora en el cálculo de las incertezas de los valores antes mencionados.\\
\indent El parámetro clave en este caso se llama \verb|EPIX|, y es un valor umbral, define a partir de qué cantidad de carga se cuenta o no como un píxel vacío. Por ejemplo, para \verb|EPIX = 0.5|, todos los píxeles con carga menor o igual $1$ se cuentan como píxeles vacíos, y los que tengan carga mayor a $1$ serán contabilizados normalmente, para \verb|EPIX = 1.5|, todos los píxeles con carga menor o igual a $2$ se cuentan como píxeles vacíos y los píxeles con carga mayor a $2$ se cuentan normalmente.\\
\indent En trabajos previos\footnote{cite tesis kevin}, los valores obtenidos para el factor de Fano y energía de creación electrón-hueco, fueron calculados con un valor de \verb|EPIX = 0.5|. Se espera que al modificar este parámetro, el conteo de eventos varíe y que, en particular, aumente cuando el \verb|EPIX| aumenta. Esto se debe a que es muy común que se tenga un evento de interés, por ejemplo un clúster de $4$ píxeles de área y con una carga total de $180$ electrones, y alrededor de este se acumulen píxeles con cargas debidas a corrientes oscuras del sensor, de por ejemplo $1$ o $2$ electrones. En estos casos podría suceder que la conexión entre el clúster de interés y los píxeles con carga espuria se extiendan lo suficiente como para que el algoritmo reconozca un gran clúster con exceso de carga y sea filtrado dado que no cumple con los cortes de calidad impuestos. También podría suceder que estos píxeles con carga espuria conecten $2$ clústeres de interés, lo cual es un caso más extremo, dado que el algoritmo reconocería un único clúster de $\sim 360$ electrones, de forma que se perderían, no $1$, sino $2$ eventos que podrían aportar positivamente a la estadística. Al aplicar un umbral que elimine los píxeles con carga espuria que se amontonan y/o conectan con clústeres, el programa es capaz de diferenciar y contar la carga correctamente.\\
\indent En la figura \ref{fig:ClusterPegoteado} se muestra un ejemplo de un evento de $174$ electrones, que es un evento de interés y que el programa debería reconocer, y que hasta que no se eliminan los eventos de hasta $2$ electrones de la imagen, el programa lo identifica como un gran (y amorfo) clúster (imagen central, píxeles pintados de blanco). A la derecha se ve la imagen con el cluster individualizado y reconocido correctamente por el algoritmo al eliminar la carga excedente.
\begin{figure}[H]
%Para modificar este plot hay que ir a /home/igna/Escritorio/Tesis2021/Figs/pys_para_plots y correr gradiente_filas_sensor.py Los datos los saca de /home/igna/Escritorio/Tesis2021/Figs/txts_para_plots y del archivo OHDU1/2/3/4_gradiente_filas_sensor.txt
%Esta imagen corresponde al set de imágenes que se procesaron con el parámetro B, y al primer cuadrante del sensor (que es el que mejor anda) Aclaro porque no lo aclaré en ningún otro lado
    \centering
    \includegraphics[scale=0.4]{Figs/despegoteo_clusters.pdf}
    \caption{\footnotesize{Ejemplo del caso de un de un evento cercano a los $180$ electrones de carga, que son los eventos de interés. En la imagen de la izquierda se ve la imagen completa, es decir, la medición sin alterar (ya convertida a unidades de carga). En la imagen del centro se en blanco y en un degrade muy tenue de rojos los diferentes clusters que el algoritmo logra reconocer. Lo importante de esta imagen es notar que el algoritmo reconoce como un unico cluster (blanco) a un número de píxeles muy grande debidido justamente a que píxeles con una unica carga generan la union entre todos ellos. Por último, la imagen de la derecha es contiene el cluster de interes una vez que los eventos de un electrón son desechados del análisis, haciendo que ahora sí se contabilice correctamente el evento de interés. Este es un evento de $174$ electrones de carga.}}
    \label{fig:ClusterPegoteado}
\end{figure}
De esta forma, se rehicieron los análisis de las imágenes con diferentes valores del corte de calidad, \verb|EPIX = 0.5|, \verb|EPIX = 1.5| y \verb|EPIX = 2.5| y se compararon los resultados obtenidos, tanto para el conteo total de eventos, como los valores finales del factor de Fano y la energía de creación electrón-hueco. Cabe aclarar que estos son resultados preliminares. Esto se hizo para $3$ de los $4$ cuadrantes del sensor, también llamados OHDU's, y se tuvo en cuenta la suma de los 3 cuadrantes funcionales.\\
\indent En las figuras \ref{fig:EntradasVsEpix}, \ref{fig:EnergiadeCreacionVsEpix} y \ref{fig:FanoVsEpix}  puede verse la variación de estos parámetros dependiendo del valor de umbral usado. El gráfico más importante en este punto es el de la figura \ref{fig:EntradasVsEpix}, donde se ve un drástico aumento en la cantidad de entradas (eventos contabilizados) cuando se varía el \verb|EPIX|. Se graficaron en cada figura las modificaciones para $3$ de los $4$ cuadrantes del sensor y para la suma de los cuadrantes $1$, $3$ y $4$. El cuadrante $2$ no funciona correctamente y por eso sus datos no han sido utilizados.
\begin{figure}[h]
%Para hacer estas figs hay que ir a /home/igna/Escritorio/Tesis2021/Figs/pys_para_plots y correr plots_entries_fano_eh.py que usa los datos que están en /home/igna/Escritorio/Tesis2021/Figs/txts_para_plots y se llaman Entries_count.txt
    \centering
    \includegraphics[scale=0.5]{Figs/Entradas_vs_Epix.pdf}
    \caption{\footnotesize{Gráfico de barras para las diferentes cantidad de entradas contabilizadas por el programa, tanto para valores diferentes de EPIX como para los diferentes cuadrantes del sensor. OHDUT hace referencia a la suma de las entradas del resto de los cuadrantes funcionales ($1$, $3$ y $4$). Se observa un aumento de más del doble en la cantidad en la cantidad de entradas para el primer cuadrante, y un aumento importante pero menos pronunciado para el resto de los cuadrantes.}}
    \label{fig:EntradasVsEpix}
\end{figure}
Se puede ver como los cuadrantes $1$, $3$ y $4$ tienen un cambio pronunciado en la cantidad de entradas al pasar de \verb|EPIX = 0.5| a \verb|EPIX = 1.5|, lo cual implica un aumento en la estadística, que era lo que se esperaba. En cambio, al pasar de \verb|EPIX = 1.5| a \verb|EPIX = 2.5| el aumento en el número de entradas es mucho menor. Particularmente, el primer cuadrante tiene un aumento muy importante en la cantidad de entradas en relación a los otros cuadrantes. Si bien el aumento sigue siendo pronunciado, para los cuadrantes $3$ y $4$ el aumento es menor. En la tabla \ref{tab:EntriesVsEpix} están los valores precisos del cambio en el número de entradas para cada cuadrante para cada valor de \verb|EPIX|. El primer cuadrante pasa de tener $760$ entradas para \verb|EPIX = 0.5| a tener $2272$ para un \verb|EPIX = 1.5|, casi el triple, es un aumento de $\sim 198\%$. En cambio, los cuadrantes $3$ y $4$ pasan de tener $1571$ y $1503$ entradas a $2229$ y $2320$, un aumento muy similar y en torno al $\sim40\%$ y $\sim50\%$ respectivamente.
\begin{table}[h]
\centering
\begin{tabular}{@{}c|c|c|c|c@{}}
\toprule
           & OHUD 1 & OHDU 3 & OHDU 4 & OHDU 1 + 3 + 4 \\ \midrule\hline
EPIX = 0.5 & 760    & 1571   & 1503   & 3834           \\
EPIX = 1.5 & 2272   & 2229   & 2320   & 6821           \\
EPIX = 2.5 & 2399   & 2261   & 2356   & 7016           \\ \bottomrule
\end{tabular}
\caption{\footnotesize{Diferentes valores para las entradas, para cada uno de los cuadrantes, para los diferentes valores de EPIX utilizados.}}
\label{tab:EntriesVsEpix}
\end{table}
Efectivamente se observa un aumento en el conteo de eventos que reconoce el programa al aumentar el valor del parámetro \verb|EPIX|. También se observa que el aumento más pronunciado es desde $0.5$ a $1.5$. El aumento promedio en el conteo de eventos es de alrededor del $100\%$.
\begin{figure}[h]
%Para hacer estas figs hay que ir a /home/igna/Escritorio/Tesis2021/Figs/pys_para_plots y correr plots_entries_fano_eh.py que usa los datos que están en /home/igna/Escritorio/Tesis2021/Figs/txts_para_plots y se llaman Entries_count.txt
    \centering
    \includegraphics[scale=0.5]{Figs/EnergiaCreacion_vs_Epix.pdf}
    \caption{\footnotesize{Diferentes valores para la energía de creación electrón-hueco en función del EPIX y del cuadrante del sensor utilizado. OHDUT hace referencia a al promedio del resto de los cuadrantes funcionales ($1$, $3$ y $4$). Se observa que en todos los casos hay un aumento de la energía de creación electrón hueco cuando aumenta el EPIX.}}
    \label{fig:EnergiadeCreacionVsEpix}
\end{figure}
En cuanto al gráfico de la figura \ref{fig:EnergiadeCreacionVsEpix}, se ve como el valor preliminar para la energía de creación electrón hueco aumenta en todos los casos al aumentar el umbral \verb|EPIX|. Esto puede deberse al efecto que genera aplicar un umbral y disminuir la carga en los clusters contabilizados, que a su vez son más. Puede suceder que la cantidad de carga en los clusters sufra un corrimiento a la izquierda del valor medio real y por esto la energía de creación electrón-hueco aumente al aumentar el \verb|EPIX|: A un mismo valor de energía, una menor carga ionizada implica una mayor energía de creación electrón-hueco.
Finalmente, el caso más irregular corresponde al gráfico de la figura \ref{fig:FanoVsEpix}, donde cada cuadrante y para cada valor de \verb|EPIX| el factor de Fano dio valores diferentes y no puede definirse una tendencia a partir de estos resultados.
\begin{figure}[h]
%Para hacer estas figs hay que ir a /home/igna/Escritorio/Tesis2021/Figs/pys_para_plots y correr plots_entries_fano_eh.py que usa los datos que están en /home/igna/Escritorio/Tesis2021/Figs/txts_para_plots y se llaman Entries_count.txt
    \centering
    \includegraphics[scale=0.5]{Figs/Fano_vs_Epix.pdf}
    \caption{\footnotesize{Variación del factor de Fano en función del EPIX para diferentes cuadrantes del sensor. No se observa un patrón que se repita.}}
    \label{fig:FanoVsEpix}
\end{figure}
Entonces, del análisis preliminar modificando el umbral conteo de carga por píxel se puede observar el aumento deseado en la estadística para los eventos de interés de este trabajo. Como se observa que el mayor aumento en la estadística se da para \verb|EPIX = 1.5|, ese es el valor de umbral con el que se realizan los subsiguientes análisis. No está de más recordar que ese valor de umbral implica desechar los píxeles que tengan $2$ o menos electrones de carga.\\
\indent Habiendo tomado este rumbo, es necesario poder remover el sesgo producido por la eliminación de carga en los eventos medidos. Si bien definir un umbral genera un aumento en la estadística, también genera un corrimiento hacia la izquierda en los picos de interés en los espectros que debe ser corregido. Más aún, es necesario remover además el exceso de carga que tengan los clústers de interés debido a corrientes oscuras o eventos indeseados que generen un corrimiento a la derecha de los picos en los espectros. En adelante, la idea es intentar comprender el ruido de fondo en las imágenes y con ello poder corregir los valores de carga de los clústers, una vez aplicado el umbral y así mejorar la incerteza de los resultados.

%%%%%%%%%%%%%%%%%%%%%%%%%%%%%%%%%%%%%%%%%%%%%%%%%%%%%%%%%%%%%%%%%%
\section{Caracterización de las imágenes}
\noindent Todo el análisis cuantitativo anteriormente descripto se realizó sin la necesidad de inspeccionar visualmente las imágenes de las cuales se extraen los datos. Simplemente se aplicaron diferentes umbrales de prueba y se contabilizó el aumento en la estadística. Sin embargo, poder ver las imágenes y rápidamente poder reconocer características que se repiten, como exceso de ruido en algunas imágenes o cualquier característica que visualmente sea reconocible pero que al analizar los datos de forma automatizada pueda quedar ofuscada, es un factor importante a la hora del estudio de los datos. Dado que la cantidad de imágenes utilizadas en este trabajo es superior a las $900$, claramente observar una por una es una tarea monumental y sin sentido. Por esta razón fue necesario buscar maneras de poder extraer información contenida en todas las imágenes, de forma práctica y realizable, como por ejemplo, generar una imagen \textit{promedio} de todas las imágenes.\\
\indent Con este fin, se hizo un análisis visual, cualitativo y cuantitativo de las imágenes para comprender mejor los datos, explorar las características del sensor y de cada uno de sos cuadrantes y poder reconocer posibles deficiencias o cualquier tipo particularidad relevante.\\
\indent Uno de los primeros factores a caracterizar es el ruido en el sensor, donde por ruido se entiende a todos aquellos píxeles carga que no es debida eventos de ionización o eventos de interés. El ruido puede ser producto de corrientes oscuras (electrones que sufren excitaciones espontáneas debido a fluctuaciones térmicas del sensor), rebotes de un haz de baja energía en la cámara donde se encuentra el sensor y producen eventos que no son de interés en un determinado píxel, etc. No es sencillo y no existe una única manera de estimar el ruido en un sensor, por lo que en este trabajo se ensayaron deferentes maneras de encarar este análisis. \\
\indent Lo primero que se hizo en este trabajo fue buscar la manera de explorar solamente los píxeles que tuvieran una única carga. Asumiendo que en la gran mayoría de los casos, los píxeles con una única carga que se encuentran aislados de otros píxeles o de clústers de interés, son debidos a ruido del sensor, es natural empezar el análisis con estos. Una forma de caracterizar esto es tomar las imágenes y extraer todos los píxeles donde la carga sea mayor que un electrón. De este modo, se obtienen imágenes donde solo hay eventos de un electrón y todo lo demás son píxeles vacíos. En la figura \ref{fig:ImagenFitsOriginal} se puede ver una típica imagen tomada con el sensor, para el primer cuadrante, en la que claramente pueden observarse algunos eventos muy brillantes y un gran fondo. En la imagen \ref{fig:ImagenFits1e} en cambio puede verse la imagen resultante de extraer todos los píxeles cuya carga sea mayor a $1$ electrón.
\begin{figure}[h]
%Para hacer estas figs hay que ir a /home/igna/Escritorio/Tesis2021/Figs/pys_para_plots y correr imagenes_fits_original_y_filtrada.py
    \centering
    \includegraphics[scale=0.4]{Figs/imagen_fits_original.pdf}
    \caption{\footnotesize{Ejemplo de imagen tomada con el sensor, en el primer cuadrante.}}
    \label{fig:ImagenFitsOriginal}
\end{figure}

\begin{figure}[h]
%Para hacer estas figs hay que ir a /home/igna/Escritorio/Tesis2021/Figs/pys_para_plots y correr imagenes_fits_original_y_filtrada.py
    \centering
    \includegraphics[scale=0.4]{Figs/imagen_fits_1_e.pdf}
    \caption{\footnotesize{Imagen resultante luego de ser extraídos los píxeles con carga mayor a $1$ electrón.}}
    \label{fig:ImagenFits1e}
\end{figure}
Una vez que se extraen los píxeles de mayor carga, se promedian todas las imágenes resultantes y se obtiene una una única imagen que condensa la información de todas las anteriores. En este contexto, promediar las imágenes implica tomar el arreglo matricial de valores ($0$ y $1$ en este caso) de carga en cada píxel (cada elemento fila y columna de la matriz) para cada imagen, y realizar la suma convencional de matrices para las $\sim 950$ imágenes. Finalmente dividir cada elemento de la matriz por la cantidad total de imágenes. De esta forma se puede ver si existen regiones con mayor o menor tendencia a concentrar este tipo de eventos.\\
\indent En la figura \ref{fig:Eventos1e} se tiene una imagen por cada cuadrante del sensor, promediados en las $925$ imágenes tomadas, donde los píxeles más brillantes son los son los que tienen mayor promedio de eventos, es decir, en el total de las imágenes esos píxeles son los que más veces tuvieron un electrón de carga. Esto también puede interpretarse como una imagen de la probabilidad por píxel de que haya un único electrón: Píxeles más brillantes son píxeles más propensos a tener carga y los píxeles más oscuros los menos propensos.\\
\indent De la figura \ref{fig:Eventos1e} pueden destacarse algunas características: Entre las que más se destacan, se encuentran:
\begin{itemize}
    \item La ausencia de carga (en promedio) en las regiones del pre-scan (región izquierda de 8 columnas de píxeles de extensión) y del over-scan (región derecha de 50 columnas de píxeles de extensión), lo cual es totalmente esperable, para todos los cuadrantes menos el segundo;
    \item El primer cuadrante es en promedio más brillante que el resto, y se observa un ligero gradiente de color entre las filas inferiores y superiores. Esto se repite, pero en menor medida en los demás cuadrantes pero no necesariamente se observa a simple vista;
    \item El segundo cuadrante (OHDU 2) capta en promedio muy poca carga. Este cuadrante del sensor es defectuoso;
    \item En los cuadrantes $3$ y $4$ se pueden ver columnas enteras de píxeles oscurecidas, que captaron muchísima menos carga (defectos del sensor?);
    \item En todos los cuadrantes (menos el segundo), se observa un único píxel (posición $x = 2$, $y = 0$) donde el promedio de carga es mucho mayor al resto. Además, la primera columna de píxeles luego del pre-scan también tiene tendencia a captar más carga que el resto;
    \item Todos los cuadrantes tienen tendencia a tener \textit{hot píxels} en el interior de la región activa, esto es píxeles aislados que tienen tendencia a captar carga más que otros, además pueden verse lineas verticales de \textit{hot píxeles} que no pueden explicarse completamente.
\end{itemize}

\begin{figure}[h]
%Para reproducir esta figura hay que ir al directorio /home/igna/Escritorio/Tesis2021/Figs/pys_para_plots y correr skipper_cuadrantes_plot.py
    \centering
    \includegraphics[scale=0.4]{Figs/1ePromedio.pdf}
    \caption{\footnotesize{Imágenes promedio para los $4$ cuadrantes del sensor. Puede verse en la escala de la derecha que los valores más altos que se obtienen rondan el $0.3$, lo cual, interpretado como una probabilidad es un $30\,\%$ de probabilidad de que en ese píxel se encuentre un evento de un electrón. En general se ve que los promedios pueden estar entre $0.1$ y $0.2$ aproximadamente. Es decir, para los cuadrantes funcionales del sensor, cada pixel tiene una probabilidad de tener un único evento que ronda entre el $10\%$ y el $20\%$.}}
    \label{fig:Eventos1e}
\end{figure}
Respecto al gradiente de color que se observa entre filas superiores e inferiores, implicaría una mayor incidencia de eventos de un electrón, en promedio, en los píxeles de las filas inferiores respecto de las filas superiores. Esto puede observarse en los gráficos de figura \ref{fig:GradienteProb}, donde se ve el aumento en \textit{la probabilidad} media por fila de que haya un evento de un electrón, a medida que el número de la fila aumenta. El gráfico de arriba a la izquierda corresponde al primer cuadrante del sensor, este es el cuadrante donde más evidente se hace este gradiente, además de ser muy lineal. La probabilidad promedio para la fila $0$ del sensor es $\sim 14.5\,\%$ y crece linealmente hasta $\sim 18\,\%$ para la fila $50$. En el gráfico de arriba a la derecha, que corresponde al segundo cuadrante, también se observa un cambio, pero solo entre las primeras 10 filas del sensor, luego la variación de la probabilidad por fila es muy pequeña y parece aproximadamente constante. Además puede ver los valores son un orden de magnitud menor a los del primer cuadrante. Para los gráficos de abajo a la izquierda y abajo a la derecha, que corresponden a los cuadrantes $3$ y $4$ del sensor respectivamente, se observan también variaciones entre las primeras filas del sensor y las últimas que parecerían tener una tendencia lineal, sin embargo, en comparación a la variación del primer cuadrante, esta es mucho menor. Por eso es difícil verlo a simple vista: La variación para el primer cuadrante es de aproximadamente del $24\,\%$ mientras que la variación de los cuadrantes $3$ y $4$ es aproximadamente del $10\,\%$.\\
\indent Esto puede deberse a que las filas superiores del sensor son las primeras a las que se les mide la carga, generando que las filas inferiores permanezcan más tiempo expuestas a fuentes de eventos. De todas maneras esta diferencia de tiempo en la lectura de las diferentes filas del sensor es muy pequeña, haciendo que estas diferencias entre filas sea pequeña, aunque apreciable en algunos casos.\\
\begin{figure}[h]
%Para modificar este plot hay que ir a /home/igna/Escritorio/Tesis2021/Figs/pys_para_plots y correr gradiente_filas_sensor.py Los datos los saca de /home/igna/Escritorio/Tesis2021/Figs/txts_para_plots y del archivo OHDU1/2/3/4_gradiente_filas_sensor.tx
    \centering
    \includegraphics[scale=0.45]{Figs/Gradiente_en_filas_sensor.pdf}
    \caption{\footnotesize{Variación de la \textit{probabilidad} promedio por filas del sensor de tener un evento de $1$ electrón, para los diferentes cuadrantes. Se ven aumentos lineales de la probabilidad para los casos de los cuadrantes $1$, $3$ y $4$ y un aumento más pronunciado en relación a los demás para el primer cuadrante.}}
    \label{fig:GradienteProb}
\end{figure}
\indent Todos los análisis subsiguientes fueron realizados principalmente con el primer cuadrante del sensor, dado que es el cuadrante que funciona mejor.\\
\indent Teniendo entonces una imagen del promedio de la cantidad de eventos de un electrón, en la búsqueda por caracterizar el ruido en el sensor, lo que se hizo posteriormente fue promediar la imagen promedio de forma de obtener un promedio total y poder interpretarlo como una \textit{probabilidad} general de que en un píxel haya un evento de un electrón. Entonces, para el primer cuadrante y considerando solo la región activa del sensor, realizando este procedimiento, se obtuvo una probabilidad $p = 0.1762 \pm 0.021$, es decir, que con esta primera manera de caracterizar el ruido, hay aproximadamente un $17\,\%$ de probabilidad de que un dado píxel de la región activa del sensor tengo un electrón.\\
\indent Sin embargo, esta es una forma muy rudimentaria para intentar caracterizar el ruido del sensor, además de que no es del todo correcta. Con este camino se asume que todos los eventos de un electrón son ruido, lo cual claramente no es correcto, de forma que la probabilidad de tener un evento de un electrón en un dado píxel, calculada de esta manera, está sobrestimada. Un camino un poco más sofisticado para estimar el ruido en el sensor es explotando el hecho de que los eventos medidos en él siguen una distribución Poissoniana: bajo la suposición de que todo píxel tiene igual probabilidad de tener una carga por ruido, que dicha probabilidad es pequeña para mediciones de corto tiempo y que el número de píxeles es muy grande ($24650$ píxeles por cuadrante), entonces es esperable que la distribución que modela estos eventos sea una Poissoniana. De esta forma, si se pudiera calcular la esperanza $\mu$ de la distribución, podría saberse la probabilidad de que en un determinado píxel se encuentre un evento de un electrón o, en general, la cantidad de electrones que se deseé.\\

\section{Estimación de la esperanza de la distribución}
\noindent Si se considera una distribución Poissoniana para la variable aleatoria \textit{número de electrones por píxel}\footnote{De ruido}, se puede tomar el caso $p = P(k = 1 | \mu) = 0.1762 \pm 0.0210$, que es la probabilidad que se obtuvo previamente. De forma iterativa puede hallarse el valor de $\mu$ que satisface la expresión anterior y resulta:
\begin{equation*}
    \mu = 0.2194 \pm 0.0001
\end{equation*}
Si bien esta forma de cuantificar el ruido en los píxeles del sensor es un poco más general, dado que ahora pueden contemplarse los casos más raros, como que un píxel tenga más de una carga producto de ruido de corrientes oscuras u otros factores, este método sigue teniendo el problema de la sobrestimación de la probabilidad por píxel, al seguir asumiendo que todo píxel con un electrón es debido a ruido. Por otro lado, esta forma también presenta el inconveniente que para calcular la probabilidad hubo que hacer dos promedios, primero un promedio sobre todas las imágenes y luego un promedio sobre todos los píxeles.\\
\indent Siguiendo sobre el mismo camino, todavía bajo la hipótesis de que todo evento de un electrón es de ruido, pero evitando el cálculo de los promedios, hay una forma de calcular la esperanza de la distribución y es notando lo siguiente: Si se toman las probabilidades de que haya una sola carga por píxel y que no haya ninguna carga por píxel, es decir, se toman
\begin{equation*}
    p_{0} \equiv P(k = 0 | \mu),
    \quad
    \quad
    p_{1} \equiv P(k = 1 | \mu)
\end{equation*}
y se mira la relación entre ambas, se tiene
\begin{equation*}
    \frac{p_{1}}{p_{0}} = \frac{\mu\,e^{-\mu}}{e^{-\mu}} = \mu
\end{equation*}
y se ve que puede hallarse directamente el valor de la esperanza de la distribución. Entonces, tomando la región activa de una imagen original (sin separar los eventos de $1$ o más electrones), como la de la figura \ref{fig:ImagenFitsOriginal}, contando la cantidad de píxeles vacíos, la cantidad de píxeles con $1$ electrón y calculando la relación entre ambas, para todas las imágenes, se puede obtener directamente la esperanza $\mu$ de la distribución. De esto se obtuvo que el valor de la esperanza es :
\begin{equation*}
    \mu = 0.2245 \pm 0.0001
\end{equation*}
Los resultados de ambos métodos son parecidos, pero distintos y no se solapan sus errores. Si bien esta segunda forma para calcular la esperanza $\mu$ parece un poco más elegante y correcta, el problema sigue estando en los datos que se utilizan. El valor seguirá estando sobrestimando en tanto se considere a todo píxel con un único electrón como ruido.\\
\indent No hay que perder de vista que el objetivo de calcular la esperanza de la distribución es poder utilizarla para estimar cuánta carga extra hay sobre los clústers debida a ruido y cuánta carga no espuria fue removida debido al umbral aplicado. Conociendo la esperanza de la distribución del ruido y la cantidad de píxeles que ocupa un clúster, puede calcularse la cantidad esperada de carga extra debido a ruido que se halla en cada clúster. Teniendo estos valores, puede corregirse el valor de la carga y con eso hacer mejores estimaciones del factor de Fano y la energía de creación electrón-hueco.\\
\indent Pero también, para poder generar una corrección en el conteo de cargas por clúster y que esté lo menos sesgada posible, hay que tener en cuenta las posibles formas en las que un clúster podría tener más o menos carga. Podría suceder que sobre la superficie de los clústers hayan eventos de más, debido a corrientes oscuras u cualquier forma de ruido del sensor. Pero también, dado que en este análisis se está aplicando un umbral que elimina eventos de $2$ o menos electrones, podría suceder que a un clúster se le quiten eventos reales que se encuentran en sus bordes. Con lo cual no solo es necesario corregir la carga por exceso en los clústers, sino también por defecto.\\
\indent Con lo cual, la esperanza que se obtuvo de calcular la relación entre eventos de $1$ electrón y píxeles vacíos contiene tanto información de eventos espúrios como información de eventos genuinos. Pero lo que se persigue es poder identificar los eventos de ruido y los genuinos por separado. En ese sentido puede decirse que 
\begin{equation*}
    \mu_{T} = \mu_{bkg} + \mu_{g}
\end{equation*}
donde $\mu_{bkg}$ es la esperanza de la distribución de la variable aleatoria \textit{cantidad de eventos espurios por píxel}, mientras que $\mu_{g}$ es la esperanza de la variable aleatoria \textit{cantidad de eventos genuinos por píxel}. Hay que lograr separar ambos efectos para poder aplicar las correcciones correctamente. Queda claro que hasta el momento solo se calculó $\mu_{T}$, sin poder discriminar las contribuciones del ruido y de los eventos genuinos. La forma en la que se llevó a cabo la separación en ambas contribuciones se detalla a continuación.

\subsection{Análisis de los bordes de los clústers}
\noindent Para poder separar ambas contribuciones al calcular el $\mu_{T}$ de la distribución, la idea fue analizar los bordes de los clústers, donde ahora por clústers se entiende todo píxel individual $2$ o más electrones de carga, o conjunto de píxeles donde cada píxel tenga como mínimo $2$ electrones de carga. Es decir, se toma una imagen que tiene eventos de $2$ o más electrones, y los demás eventos se hacen $0$, como se ve en la figura \ref{fig:ImagenFits2omasElectrones}.
\begin{figure}[h]
%Para modificar este plot hay que ir a /home/igna/Escritorio/Tesis2021/Figs/pys_para_plots y correr imagen_fit_2_o_mas_e.py
    \centering
    \includegraphics[scale=0.4]{Figs/imagen_fits_2_o_mas.pdf}
    \caption{\footnotesize{Imagen de ejemplo en la que solo hay píxeles que tenga más de $2$ electrones.}}
    \label{fig:ImagenFits2omasElectrones}
\end{figure}
Usando como referencia los clústers de esta imagen, se genera una máscara sobre de los bordes de estos, es decir, los píxeles inmediatamente contiguos a los píxeles con carga. Como la idea es corregir la carga en los clústers, tiene sentido mirar en el entorno de estos y no en todo el sensor.\\
\indent El procedimiento consiste en tomar los clústers y hacer una máscara de su contorno, es decir, la máscara es el conjunto de píxeles que rodea al clúster, sin incluir los píxeles cargados. Luego, superponiendo la máscara sobre la imagen original (con todos los eventos), se cuenta cuántos eventos cayeron dentro de la máscara. 
\begin{figure}[h]
%Para modificar este plot hay que ir a /home/igna/Escritorio/Tesis2021/Figs/pys_para_plots y correr imagen_bordes.py
    \centering
    \includegraphics[scale=0.4]{Figs/analisis_bordes.pdf}
    \caption{\footnotesize{Diferentes partes del proceso de análisis de los bordes de los clusters para una imagen de ejemplo. En cada figura se ve una porción de $25 \times 25$ píxeles de área. En la primera imagen (contando de izquierda a derecha) se tiene los clústers de $2$ o más electrones. En la segunda imagen se representa la dilatación de los clústers aumentando en $1$ píxel en todas las direcciones. En la tercera imagen se ve la diferencia entre las dos primeras imágenes y se la define como la máscara a utilizar. En la cuarta se ve la máscara y superpuestos todos los eventos de $1$ electrón de esa porción del sensor. Finalmente, en la quinta se se ven solo los eventos de $1$ electrón que cayeron dentro de los píxeles de máscara. Son estos eventos los que son se cuentan en todas las imágenes, junto con los píxeles vacíos de la máscara para calcular el $\mu_{T}$.}}
    \label{fig:AnalisisBordes}
\end{figure}
Nuevamente, haciendo la relación entre eventos de un electrón y píxeles vacíos, se calcula el $\mu_{T}$. En la figura \ref{fig:AnalisisBordes} puede verse gráficamente cada uno de los pasos que se lleva a cabo para contabilizar los eventos de $1$ electrón que caen sobre la máscara. Se estima que en el borde inmediato a los clústers existe contribución de ambos efectos: eventos espurios y eventos genuinos.\\
\indent Por otro lado, para calcular la contribución de los eventos espurios, se hace el mismo procedimiento pero expandiendo los clústers en dos píxeles en todas las direcciones, y quedándose únicamente con el segundo borde, donde claramente no puede haber contribución de eventos genuinos de un cluster. Este proceso puede verse en la imagen \ref{fig:AnalisisBordesx2}. Nuevamente, de la relación entre los eventos de $1$ electrón y los píxeles vacíos, se obtiene $\mu_{bkg}$ y con este, puede despejarse el valor $\mu_{g}$.\\
\begin{figure}[h]
%Para modificar este plot hay que ir a /home/igna/Escritorio/Tesis2021/Figs/pys_para_plots y correr imagen_bordes.py
    \centering
    \includegraphics[scale=0.4]{Figs/analisis_bordesx2.pdf}
    \caption{\footnotesize{Análoga a la figura \ref{fig:AnalisisBordes}, pero para el caso de $2$ dilataciones, de forma de generar una máscara en el segundo borde. Los pasos son los mismos antes descriptos. De este proceso se halla la esperanza $\mu_{bkg}$.}}
    \label{fig:AnalisisBordesx2}
\end{figure}
La razón por la cual se usan los eventos inmediatamente contiguos a los bordes de los clústers para calcular el $\mu_{T}$, es porque se puede decir con seguridad que es la única región del sensor donde coexisten ambas contribuciones: $\mu_{bkg}$ y $\mu_{g}$. Tanto los eventos de un electrón que realmente pertenecen a los clústers como los eventos de un electrón que son espurios. Por otro lado, la razón por la cual se estima $\mu_{bkg}$ del siguiente cordón ($2$ píxeles de distancia al borde del clúster) se debe a que en esa zona es seguro que no pueden haber eventos genuinos de un clúster (o al menos la probabilidad de que eso suceda ronda los $10\times 10^{-14}$).\\
\indent Finalmente, de realizar estos análisis se obtuvieron los valores para las esperanzas de ambas contribuciones, que resultaron ser:
\begin{equation*}
    \mu_{T} = 0.19735 \pm 0.00024
\end{equation*} 
y el valor de la esperanza para los eventos espurios resultó 
\begin{equation*}
    \mu_{bkg} = 0.18583 \pm 0.00021
\end{equation*}
con lo cual, la esperanza para los eventos genuinos es 
\begin{equation*}
    \mu_{g} = 0.011524 \pm 0.00003
\end{equation*}
Teniendo estos valores puede corregirse el conteo de carga en los clústeres, tanto por exceso como por defecto y así tener una medición más precisa del factor de Fano y la energía de creación electrón-hueco.

\subsection{Corrección del conteo de carga}
\noindent El punto de todo el análisis anterior era poder generar las herramientas para corregir el conteo de carga que hace el programa de detección de clusters, luego de aplicar un umbral que podría eliminar carga genuina y además poder estimar cuánta carga excedente por corrientes oscuras tiene cada clúster.\\
\indent Esta corrección se llevó a cabo modificando el código del programa que con \textit{root} calcula el factor de Fano, la energía de creación electrón-hueco, y otras variables, por medio del ajuste de los espectros de carga. El programa se encarga de procesar todas las imágenes, desechar eventos que tengan carga menor a $2$ electrones (\verb|EPIX = 1.5|), filtrar eventos que no cumplan ciertos criterios (cortes de calidad), buscar los clusters, quedarse únicamente con aquellos que tengan cargas en el entorno de los $180$ electrones y calcular esos valores. Al contar la carga de estos clústers y conociendo el área de los mismos (cantidad de píxeles que los conforman), se agrega y se quita carga en función los valores hallados antes.\\
\indent Dado que la distribución de la variable aleatoria \textit{cantidad de eventos por píxel} sigue una distribución poissoniana, de esperanza $\mu$, para cada tipo evento (espurio, genuino o total) se tiene una esperanza. Para calcular la cantidad de carga que se espera que tenga un clúster de $N$ píxeles en el sensor, se puede hacer uso de las propiedades de la esperanza. Sea $Y = \sum\limits_{i = 1}^{N} X_{i}$, donde $X_{i}$ son distintas realizaciones de la variable aleatoria con distribución Poissoniana y $N$ es el número de píxeles del clúster, entonces la esperanza se calcula como
\begin{equation*}
    E(Y) = 
    E
    \left(
        \sum\limits_{i=1}^{N} X_{i}
    \right)
    = \sum\limits_{i=1}^{N}E(X_{i})
    = \sum\limits_{i=1}^{N}\mu_{i}
\end{equation*}
pero como $X_{i}$ son distintas realizaciones de la misma variable aleatoria, entonces tienen todas la misma esperanza, es decir $\mu_{i} = \mu\ \forall\ i$, con lo cual
\begin{equation*}
    E(Y) = N\mu
\end{equation*}
con lo cual, la cantidad de carga esperada para un cluster viene dada por la esperanza de la distribución, por la cantidad de píxeles. De esta forma, teniendo una esperanza total $\mu_{T} = \mu_{bgk} + \mu_{g}$, aplicando esta misma receta pueden corregirse los valores de carga por cluster. Si $n_{e}$ es la cantidad de carga medida en un dado cluster, la corrección de este valor de carga será $n_{c}$ y viene dado por
\begin{equation*}
    n_{c} = n_{e} + N\mu_{g} - N\mu_{bkg}
\end{equation*}
es decir, se agrega la cantidad de carga que se estima se pierde en los bordes por aplicar el umbral \verb|EPIX = 1.5| y se quita la carga estimada por ruido en el interior de los clústers. \textcolor{red}{Acá no me queda del todo claro como era el tema del redondeo: $N\mu_{g}$ y $N\mu_{bkg}$ claramente son float, pero la carga tiene que ser entera. El redondeo donde se hacía?}

%%%%%%%%%%%%%%%%%%%%%%%%%%%%%%%%%%%%%%%%%%%%%%%%%%%%%%%%%%%%%%%%%%

\chapter{Resultados}
Sección pendiente porque faltan los resultados.

%%%%%%%%%%%%%%%%%%%%%%%%%%%%%%%%%%%%%%%%%%%%%%%%%%%%%%%%%%%%%%%%%%

\chapter{Conclusiones}
faltan las conclusiones