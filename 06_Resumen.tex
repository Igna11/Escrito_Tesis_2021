\newpage
%\begin{center}
%    {\Large \textbf{Resumen}}
%\end{center}
\chapter*{Resumen}
\thispagestyle{empty}
%<<Breve introducción>>
\noindent El ruido de lectura en los sensores CCDs ha sido una limitación inherente a la electrónica de estos dispositivos, imponiendo límites en la resolución con que podía medirse la carga en ellos generada. Los sensores \textit{Skipper}-CCDs, por su parte, han logrado superar esta barrera, reduciendo el ruido de lectura a niveles subelectrónicos. Sin embargo, el ruido de lectura no es el único factor que está en juego al momento de procesar mediciones realizadas con este tipo de sensores: los eventos provenientes de fuentes que no son de interés y los inducidos por fenómenos intrínsecos a la naturaleza de las interacciones, deben ser también considerados.


%\textbf{<<Objetivo de la tesis>>}
En esta tesis se estudió el fondo presente en un conjunto de más de $900$ imágenes tomadas con un sensor \textit{Skipper}-CCD, de forma de caracterizarlo con precisión en sus valores medios. A su vez se probaron tres diferentes umbrales de detección de eventos, de forma tal de elegir aquel que aumente lo más posible la estadística introduciendo el menor sesgo posible en los datos. Una vez logrado esto, se utilizó la información obtenida de los análisis para corregir los sesgos presentes en la determinación del factor de Fano y la energía de creación electrón-hueco, junto con un modelo de ajuste de los datos, especialmente diseñado para reproducir los efectos introducidos por el detector.


%\textbf{<<Trabajo realizado>>}
En paralelo, se estudió el fenómeno de ionización por medio de simulaciones Monte Carlo. Para ello se partió de un modelo muy sencillo para luego adoptar uno más sofisticado, con el fin de poner en evidencia las razones por las cuales el factor de Fano medido experimentalmente es un orden de magnitud menor a la unidad. De estas simulaciones se obtuvieron resultados para el factor de Fano semejantes a los obtenidos experimentalmente.


%\textbf{<<Resultados>>}
Los resultados de este trabajo presentan la determinación de las magnitudes antes mencionadas para un rango de energías inaccesible hasta el momento, lo cual es de vital importancia en la caracterización de estos dispositivos en el rango de bajas energías, donde los CCDs convencionales presentan importantes barreras debido al ruido de lectura. Es por esto que la importancia de este trabajo radica en la combinación de la tecnología \textit{Skipper} con el uso de técnicas de análisis que permitieron corregir sesgos en mediciones que, hasta el momento, no habían podido realizarse.


%\textbf{<<Perspectivas a futuro?>>}
%------PÁGINA VACIA ------------------
\newpage
\thispagestyle{empty} \mbox{}
\thispagestyle{empty}
%-------------------------------