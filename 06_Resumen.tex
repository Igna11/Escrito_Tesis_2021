\newpage
%\begin{center}
%    {\Large \textbf{Resumen}}
%\end{center}
\chapter*{Resumen}
\thispagestyle{empty}
\vspace{-10mm}
%<<Breve introducción>>
\noindent El ruido de lectura en los sensores CCDs ha sido una limitación inherente a la electrónica de estos dispositivos, imponiendo límites en la resolución con que podía medirse la carga en ellos generada. Los sensores \textit{Skipper}-CCDs, por su parte, han logrado superar esta barrera, reduciendo el ruido de lectura a niveles subelectrónicos. Sin embargo, el ruido de lectura no es el único factor que afecta las mediciones: los eventos provenientes de fuentes que no son de interés, los efectos inducidos por fenómenos intrínsecos a la naturaleza de las interacciones y los introducidos por el mismo detector deben ser también considerados.


%\textbf{<<Objetivo de la tesis>>}
En esta tesis se estudió y caracterizó el fondo presente en un conjunto de más de $900$ imágenes tomadas con un sensor \textit{Skipper}-CCD. A su vez se probaron tres umbrales de detección de eventos, de forma tal de elegir aquel que maximice la estadística disponible introduciendo el menor sesgo posible en los datos. Luego, se utilizó la información obtenida para corregir los sesgos presentes en la determinación del factor de Fano y la energía de creación electrón-hueco. El análisis se realizó mediante un modelo de ajuste de los datos, especialmente diseñado para reproducir los efectos introducidos por la región de colección parcial de carga presente en los sensores.


%\textbf{<<Trabajo realizado>>}
En paralelo, se estudió el fenómeno de ionización por medio de simulaciones Monte Carlo. Para ello se partió de un modelo muy sencillo para luego adoptar uno más sofisticado, con el fin de poner en evidencia las razones por las cuales el factor de Fano es un orden de magnitud menor a la unidad. De estas simulaciones se obtuvieron resultados en acuerdo con los observados experimentalmente.


%\textbf{<<Resultados>>}
Los resultados de este trabajo presentan la determinación de las magnitudes mencionadas para energías inaccesibles hasta el momento (677 eV y 1486 eV), lo cual es de vital importancia para conocer el desempeño de estos dispositivos en el rango de bajas energías, donde los CCDs convencionales presentan importantes barreras debido al ruido de lectura. Así es que este trabajo explora propiedades de los sensores de silicio mediante la tecnología \textit{Skipper}, que habilita ruidos de lectura sub-electrónicos. Esto último, junto con técnicas de análisis que hicieron posible corregir sesgos preexistentes e inducidos y modelar efectos del sensor, permitió determinar cantidades de interés en una región de particular importancia como es el de bajas energías.


%------PÁGINA VACIA ------------------
\newpage
\thispagestyle{empty} \mbox{}
\thispagestyle{empty}
%-------------------------------