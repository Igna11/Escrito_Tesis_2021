\newpage
%\begin{center}
%    {\Large \textbf{Resumen}}
%\end{center}
\chapter*{Resumen}
\thispagestyle{empty}
%<<Breve introducción>>
\noindent El ruido de lectura en los sensores CCDs ha sido una limitación inherente a la electrónica de estos dispositivos, imponiendo límites en la resolución con que podía medirse la carga en ellos generada. Los sensores \textit{Skipper}-CCDs, por su parte, han logrado superar esta barrera al poder reducir el ruido de lectura a niveles subelectrónicos.

%\textbf{<<Motivación>>}
Sin embargo, el ruido de lectura no es el único factor que está en juego al momento de procesar mediciones realizadas con este tipo de sensores. Los eventos provenientes de fuentes que son de interés y los inducidos por fenómenos intrínsecos a la naturaleza de las interacciones, deben ser también considerados.
%también se encuentran los eventos provenientes de fuentes que no son de interés. Estos son de diferente origen, tanto externos al detector como intrínsecos. En esta última categoría puede incluirse los eventos inducidos por deficiencia en la colección de carga.

%\textbf{<<Objetivo de la tesis>>}
En esta tesis se estudió el fondo presente en un conjunto de más de $900$ imágenes tomadas con un sensor \textit{Skipper}-CCD de forma de caracterizarlo con precisión en sus valores medios. Una vez logrado esto, se utilizó esta información junto con un novedoso modelo para el ajuste de los datos con el fin de corregir el sesgo que el fondo supone en la determinación de magnitudes como el factor de Fano y la energía de creación electrón-hueco del silicio, para energía inferiores a los $2\,\si{keV}$.

%\textbf{<<Trabajo realizado>>}
Durante el desarrollo de este trabajo se estudió por medio de simulaciones Monte Carlo el fenómeno de ionización, partiendo de un modelo muy sencillo para luego adoptar un modelo más sofisticado. De estas simulaciones se obtuvieron resultados para el factor de Fano semejantes a los medidos experimentalmente. En lo que respecta al tratamiento de datos, se probaron diferentes cortes de calidad sobre las imágenes, de forma de elegir el que maximice la estadística y minimice el sesgo. Además se calcularon las correcciones a aplicar sobre las magnitudes de interés, mejorando su incerteza por el aumento en la estadística.

%\textbf{<<Resultados>>}
Gracias a estos análisis, se determinaron magnitudes como el factor de Fano con una mejora en su incerteza del $N\,\%$ y la energía de creación electrón-hueco con una mejora del $M\,\%$.

%\textbf{<<Perspectivas a futuro?>>}
Los resultados de este trabajo presentan una mejora en la determinación de las magnitudes antes mencionadas, lo cual es de vital importancia en la caracterización de estos dispositivos en el rango de bajas energías, donde los CCDs convencionales presentan importantes barreras debido al ruido de lectura. Es por esto que la importancia de este trabajo radica en la combinación de la tecnología \textit{Skipper} a la vez del uso de técnicas de análisis que permitieron mejorar las incertezas de estas magnitudes que, hasta el momento, no habían podido medirse con esta precisión.

%------PÁGINA VACIA ------------------
\newpage
\thispagestyle{empty} \mbox{}
\thispagestyle{empty}
%-------------------------------